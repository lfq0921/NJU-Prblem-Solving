% 2-10-data-structures.tex

%%%%%%%%%%%%%%%%%%%%
\documentclass[a4paper, justified]{tufte-handout}

% hw-preamble.tex

% geometry for A4 paper
% See https://tex.stackexchange.com/a/119912/23098
\geometry{
  left=20.0mm,
  top=20.0mm,
  bottom=20.0mm,
  textwidth=130mm, % main text block
  marginparsep=5.0mm, % gutter between main text block and margin notes
  marginparwidth=50.0mm % width of margin notes
}

% for colors
\usepackage{xcolor} % usage: \color{red}{text}
% predefined colors
\newcommand{\red}[1]{\textcolor{red}{#1}} % usage: \red{text}
\newcommand{\blue}[1]{\textcolor{blue}{#1}}
\newcommand{\teal}[1]{\textcolor{teal}{#1}}

\usepackage{todonotes}

% heading
\usepackage{sectsty}
\setcounter{secnumdepth}{2}
\allsectionsfont{\centering\huge\rmfamily}

% for Chinese
\usepackage{xeCJK}
\usepackage{zhnumber}
\setCJKmainfont[BoldFont=FandolSong-Bold.otf]{FandolSong-Regular.otf}

% for fonts
\usepackage{fontspec}
\newcommand{\song}{\CJKfamily{song}} 
\newcommand{\kai}{\CJKfamily{kai}} 

% To fix the ``MakeTextLowerCase'' bug:
% See https://github.com/Tufte-LaTeX/tufte-latex/issues/64#issuecomment-78572017
% Set up the spacing using fontspec features
\renewcommand\allcapsspacing[1]{{\addfontfeature{LetterSpace=15}#1}}
\renewcommand\smallcapsspacing[1]{{\addfontfeature{LetterSpace=10}#1}}

% for url
\usepackage{hyperref}
\hypersetup{colorlinks = true, 
  linkcolor = teal,
  urlcolor  = teal,
  citecolor = blue,
  anchorcolor = blue}

\newcommand{\me}[4]{
    \author{
      {\bfseries 姓名:}\underline{#1}\hspace{2em}
      {\bfseries 学号:}\underline{#2}\hspace{2em}\\[10pt]
      {\bfseries 评分:}\underline{#3\hspace{3em}}\hspace{2em}
      {\bfseries 评阅:}\underline{#4\hspace{3em}}
  }
}

% Please ALWAYS Keep This.
\newcommand{\noplagiarism}{
  \begin{center}
    \fbox{\begin{tabular}{@{}c@{}}
      请独立完成作业,不得抄袭。\\
      若得到他人帮助, 请致谢。\\
      若参考了其它资料,请给出引用。\\
      鼓励讨论,但需独立书写解题过程。
    \end{tabular}}
  \end{center}
}

\newcommand{\goal}[1]{
  \begin{center}{\fcolorbox{blue}{yellow!60}{\parbox{0.50\textwidth}{\large 
    \begin{itemize}
      \item 体会``思维的乐趣''
      \item 初步了解递归与数学归纳法 
      \item 初步接触算法概念与问题下界概念
    \end{itemize}}}}
  \end{center}
}

% Each hw consists of four parts:
\newcommand{\beginrequired}{\hspace{5em}\section{作业 (必做部分)}}
\newcommand{\beginoptional}{\section{作业 (选做部分)}}
\newcommand{\beginot}{\section{Open Topics}}
\newcommand{\begincorrection}{\section{订正}}
\newcommand{\beginfb}{\section{反馈}}

% for math
\usepackage{amsmath, mathtools, amsfonts, amssymb}
\newcommand{\set}[1]{\{#1\}}

% define theorem-like environments
\usepackage[amsmath, thmmarks]{ntheorem}

\theoremstyle{break}
\theorempreskip{2.0\topsep}
\theorembodyfont{\song}
\theoremseparator{}
\newtheorem{problem}{题目}[subsection]
\renewcommand{\theproblem}{\arabic{problem}}
\newtheorem{ot}{Open Topics}

\theorempreskip{3.0\topsep}
\theoremheaderfont{\kai\bfseries}
\theoremseparator{:}
\theorempostwork{\bigskip\hrule}
\newtheorem*{solution}{解答}
\theorempostwork{\bigskip\hrule}
\newtheorem*{revision}{订正}

\theoremstyle{plain}
\newtheorem*{cause}{错因分析}
\newtheorem*{remark}{注}

\theoremstyle{break}
\theorempostwork{\bigskip\hrule}
\theoremsymbol{\ensuremath{\Box}}
\newtheorem*{proof}{证明}

% \newcommand{\ot}{\blue{\bf [OT]}}

% for figs
\renewcommand\figurename{图}
\renewcommand\tablename{表}

% for fig without caption: #1: width/size; #2: fig file
\newcommand{\fig}[2]{
  \begin{figure}[htbp]
    \centering
    \includegraphics[#1]{#2}
  \end{figure}
}
% for fig with caption: #1: width/size; #2: fig file; #3: caption
\newcommand{\figcap}[3]{
  \begin{figure}[htbp]
    \centering
    \includegraphics[#1]{#2}
    \caption{#3}
  \end{figure}
}
% for fig with both caption and label: #1: width/size; #2: fig file; #3: caption; #4: label
\newcommand{\figcaplbl}[4]{
  \begin{figure}[htbp]
    \centering
    \includegraphics[#1]{#2}
    \caption{#3}
    \label{#4}
  \end{figure}
}
% for margin fig without caption: #1: width/size; #2: fig file
\newcommand{\mfig}[2]{
  \begin{marginfigure}
    \centering
    \includegraphics[#1]{#2}
  \end{marginfigure}
}
% for margin fig with caption: #1: width/size; #2: fig file; #3: caption
\newcommand{\mfigcap}[3]{
  \begin{marginfigure}
    \centering
    \includegraphics[#1]{#2}
    \caption{#3}
  \end{marginfigure}
}

\usepackage{fancyvrb}

% for algorithms
\usepackage[]{algorithm}
\usepackage[]{algpseudocode} % noend
% See [Adjust the indentation whithin the algorithmicx-package when a line is broken](https://tex.stackexchange.com/a/68540/23098)
\newcommand{\algparbox}[1]{\parbox[t]{\dimexpr\linewidth-\algorithmicindent}{#1\strut}}
\newcommand{\hStatex}[0]{\vspace{5pt}}
\makeatletter
\newlength{\trianglerightwidth}
\settowidth{\trianglerightwidth}{$\triangleright$~}
\algnewcommand{\LineComment}[1]{\Statex \hskip\ALG@thistlm \(\triangleright\) #1}
\algnewcommand{\LineCommentCont}[1]{\Statex \hskip\ALG@thistlm%
  \parbox[t]{\dimexpr\linewidth-\ALG@thistlm}{\hangindent=\trianglerightwidth \hangafter=1 \strut$\triangleright$ #1\strut}}
\makeatother

% for footnote/marginnote
% see https://tex.stackexchange.com/a/133265/23098
\usepackage{tikz}
\newcommand{\circled}[1]{%
  \tikz[baseline=(char.base)]
  \node [draw, circle, inner sep = 0.5pt, font = \tiny, minimum size = 8pt] (char) {#1};
}
\renewcommand\thefootnote{\protect\circled{\arabic{footnote}}} % feel free to modify this file
%%%%%%%%%%%%%%%%%%%%
\title{第10讲: 基础数据结构}
\me{林凡琪 }{211240042 }{}{}
\date{\zhtoday} % or like 2019年9月13日
%%%%%%%%%%%%%%%%%%%%
\begin{document}
\maketitle
%%%%%%%%%%%%%%%%%%%%
\noplagiarism % always keep this line
%%%%%%%%%%%%%%%%%%%%
\begin{abstract}
  % \begin{center}{\fcolorbox{blue}{yellow!60}{\parbox{0.65\textwidth}{\large 
  %   \begin{itemize}
  %     \item 
  %   \end{itemize}}}}
  % \end{center}
\end{abstract}
%%%%%%%%%%%%%%%%%%%%
\beginrequired

%%%%%%%%%%%%%%%
\begin{problem}[MA 2.6]
(只要求考察 Push 操作)
\end{problem}

\begin{solution}
  Before:

  Invariance 1:if there are two elems that (a. t) $\in$elems and (b, t)$\in$elems

  $\because$t is the same, so we can conclude that a = b

  $\therefore$a$\ne$b are false

  Invariance 2: before the operation, we know that c is the current time-point, so it is greater than any other time-point in the stack

  Invariance 3:we know that max is the maximum of allowable size, so any elem in the stack, |elem|$\le$max

  After:

  Invariance 1:after the operation, the time-point of the new elem that we  pushed in is different than the former elems in the stack, so we cannot find that (a, t)$\in$elems, (b, t)$\in$elems, that a = b, invariance 1 holds

  Invariance 2:c' stands for the current time-point, which is the time-point of the elem that we pushed in.Also, it is greater than the former c, and the former c is greater than other time-point in the stack, so invariance 2 holds

  Incariance 3: because the elem that we can push in is smaller than max, and any other elem in the stack is smaller than max, so |elems|$\le$max, so invariance 3 holds
\end{solution}
%%%%%%%%%%%%%%%


%%%%%%%%%%%%%%%
\begin{problem}[TC 10.1-4]
重写ENQUEUE和DEQUEUE的代码,使之能处理队列的下溢和上溢。
\end{problem}

\begin{solution}
  \begin{algorithm}[H]
    \caption{ENQUEUE}
    \label{alg:sum}
    \begin{algorithmic}[1]
      \Procedure{ENQUEUE}{$Q,x$}
      \If {$Q.head == Q.tail + 1$}
      \State error "overflow"
      \Else
      \State $Q[Q.tail] = x$
      \If {$Q.tail == n$}
      \State $Q.tail = 1$
      \Else \State $Q.tail = Q.tail + 1$
      \EndIf
      \EndIf
      \EndProcedure
    \end{algorithmic}
  \end{algorithm}
  \begin{algorithm}
    \begin{algorithmic}[2]
      \Procedure{DEQUEUE}{Q}
      \If {$Q.tail == Q.head$}
      \State error "underflow"
      \Else
      \State $x = Q[Q.head]$
      \If {$Q.head == n$}
      \State $Q.head = 1$
      \Else \State $Q.head = Q.head + 1$
      \EndIf
      \State \Return $x$
      \EndIf
      \EndProcedure
    \end{algorithmic}
  \end{algorithm}

\end{solution}
%%%%%%%%%%%%%%%

%%%%%%%%%%%%%%%
\begin{problem}[TC 10.1-6]
说明如何用两个栈实现一个队列,并分析相关队列操作的运行时间。
\end{problem}

\begin{solution}
  设两个栈A和B。\\
  ENQUEUE操作:把元素push进B。DEQUEUE操作:把元素从A中pop出来\\
  如果A是空的,B中的内容就从Bpop出来,push进A。\\
  DEQUEUE操作能在$\Theta (n)$的时间里完成,但这只能在A为空的时候发生。\\
  如果许多ENQUEUE和DEQUEUE的操作被执行,总时间将接近元素数量的倍数,而不是队列的整体长度。
\end{solution}
%%%%%%%%%%%%%%%

%%%%%%%%%%%%%%%
\begin{problem}[TC 10.2-6]
动态集合操作UNION以两个不相交的集合$S_1$和$S_2$作为输入,并返回集合$S = S_1 \cup S_2$,包含$S_1$和$S_2$的所有元素。该操作通常会破坏集合$S_1$和$S_2$。试说明如何选用一种合适的表类数据结构,来支持$O(1)$时间的UNION操作。
\end{problem}

\begin{solution}
  选用链表的数据结构。\\
  不妨假设$S_1$和$S_2$都是双向链表,则将$S_1.tail$与$S_2.head$连接,构成新链表$S$。
\end{solution}
%%%%%%%%%%%%%%%

%%%%%%%%%%%%%%%
\begin{problem}[TC 10.3-4]
我们往往希望双向链表的所有元素在存储器中保持紧凑,例如,在若数组表示中占用前m个下表为止。(在页式虚拟存储的计算环境下,即为这种情况。)假设除指向链表本身的指针外没有其他指针指向该链表的元素,试说明如何实现过程ALLOCATE-OBJECT和FREE-OBJECT,使得该表示保持紧凑(提示:使用栈的数组实现)
\end{problem}

\begin{solution}
  \begin{algorithm}
    \begin{algorithmic}
      \Function{ALLOCATE-OBJECT}{}
      \If {STACK-EMPTY(F)}
      \State error "out of space"
      \Else \State $x = POP(F)$
      \State \Return $x$
      \EndIf
      \EndFunction
    \end{algorithmic}
  \end{algorithm}
  \begin{algorithm}
    \begin{algorithmic}
      \Procedure{FREE-OBJECT}{$x$}
      \State $p = F.top - 1$
      \State $p.prev.next = x$
      \State $p.next.prev = x$
      \State $x.key = p.key$
      \State $x.prev = p.prev$
      \State $x.next = p.next$
      \State $PUSH(F,p)$
      \EndProcedure
    \end{algorithmic}
  \end{algorithm}
\end{solution}
%%%%%%%%%%%%%%%

%%%%%%%%%%%%%%%
\begin{problem}[TC 10.3-5]
设L是一个长度为n的双向链表,存储于长度为m的数组key、prev、和next中。假设这些数组由维护双链自由表F的两个过程ALLOCATE-OBJEXT和FREE-OBJECT进行管理。又假设m个元素中,恰有n个元素在链表L上,m-n个在自由表上。给定链表L和自由表F,试写出一个过程COMPACTIFY-LIST(L,F),用来移动L中的元素使其占用数组中1,2,...,n的位置,调整自由表F以保持其正确性,并且占用数组中n+1,n+2,...,m的位置。要求所写的过程运行时间应为$\Theta(n)$,且只使用固定量的额外存储空间。请证明写的过程是正确的。
\end{problem}

\begin{solution}
  We represent the combination of arrays key, prev, and next by a multible-array A. Each object of A's is either in list L or in the free list F, but not in both. The procedure \text{COMPACTIFY-LIST} transposes the first object in LL with the first object in A, the second objects until the list L is exhausted.

  \begin{algorithm}
    \begin{algorithmic}
      \Procedure{COMPACTIFY-LIST}{$L, F$}
      \If{$n == m$}
      \State \Return
      \EndIf
      \State $e = max\{max_{i\in [m]}\{|key[i]| \}, max_{i\in L}\{|key[i]| \}\}$
      \For {$i \gets 1 to m$}
      \State $k[i] += 2e$
      \EndFor
      \State for every element of L, if its key is greater than e, reduce it by 2e
      \State $f=1$
      \While {key[f]<e}
      \State $f++$
      \EndWhile
      \State $a=L.head$
      \If {$a>m$}
      \State next[prev[f]] = next[f]
      \State prev[next[f]] = prev[f]
      \State next[f] = next[a]
      \State key[f] = key[a]
      \State prev[f] = prev[a]
      \State FREE − OBJECT(a)
      \State f++
      \While {$key[f] < e$}
      \State $f++$
      \EndWhile
      \EndIf
      \While {$a \neq L.head$}
      \If {$a > m$}
      \State next[prev[f]] = next[f]
      \State prev[next[f]] = prev[f]
      \State next[f] = next[a]
      \State key[f] = key[a]
      \State prev[f] = prev[a]
      \State FREE − OBJECT(a)
      \State f + +
      \While {$key[f] < e$}
      \State $f++$
      \EndWhile
      \EndIf
      \EndWhile
      \EndProcedure
      %\State TRANSPOSE(A[L.head], A[1])
      %\If {$F.head == 1$}
      %\State $F.head = L.head$
      %\EndIf
      %\State $L.head = 1$
      %\State $l = A[L.head].next$
      %\State $i = 2$
      %\While {$l \neq NIL$}
      %\State TRANSPOSE(A[l], A[i])
      %\If {$F == i$}
    \end{algorithmic}
  \end{algorithm}
\end{solution}
%%%%%%%%%%%%%%%

%%%%%%%%%%%%%%%
\begin{problem}[TC 10.4-2]
\end{problem}

\begin{solution}
  \begin{algorithm}
    \begin{algorithmic}
      \Procedure{PRINT-TREE}{$T.root$}
      \If {$T,root == NIL$}
      \State \Return
      \Else
      \State $Print (T.root.key)$
      \State $PRINT-TREE(T.root.left)$
      \State $PRINT-TREE(T.root.right)$
      \EndIf
      \EndProcedure
    \end{algorithmic}
  \end{algorithm}
\end{solution}
%%%%%%%%%%%%%%%

%%%%%%%%%%%%%%%
\begin{problem}[TC 10.4-3]
\end{problem}

\begin{solution}
  \begin{algorithm}
    \begin{algorithmic}
      \Procedure{INORDER-PRINT}{$T$}
      \State let S be an empty stack
      \State push(S, T)
      \While {!S.EMPTY}
      \State U = pop(S)
      \If {$U \neq NIL$}
      \State print (U.key)
      \State push(S, U.lef t)
      \State push(S, U.right)
      \EndIf
      \EndWhile
      \EndProcedure
    \end{algorithmic}
  \end{algorithm}

\end{solution}
%%%%%%%%%%%%%%%

%%%%%%%%%%%%%%%
\begin{problem}[TC Problem 10-3]
\end{problem}

\begin{solution}
  a. If the original version of the algorithm takes only t iterations, then, we have that it was only at most t random skips though the list to get to the desired value, since each iteration of the original while loop is a possible random jump followed by a normal step through the linked list.

  b. The for loop on lines 2–7 will get run exactly tt times, each of which is constant runtime. After that, the while loop on lines 8–9 will be run exactly $X_t$ times. So, the total runtime is O(t + E[$X_t$])

  c.Using equation C.25, we have that E[$X_t$] = $\sum$Pr$\{$$X_t$$\ge$i$\}$.So we need to show that Pr$\{$$X_t$$\ge$i$\}$$\le$$(1 - \frac{i}{n})^t$.This can be seen because having $X_t$ being greater than i means that each random choice will result in an element that is either at least i steps before the desired element, or is after the desired element.There are n - i such elements, out of the total n elements that we were pricking from.So, for a single one of the choices to be from such a range, we have a probability of $\frac{n - i}{n}$ = 1 - $\frac{i}{n}$.Since each of the selections was independent, the total probability that all of them were is $(1 - \frac{i}{n})^t$, as desired.Lastly, we can note that since the linked list has length n, the probability that $X_t$ is greater than n is equal to zero

      d.we have that t > 0, and f(x) = $x^t$ is increasing, so

    $\sum$$r^t$ = $\int_0^n$$[r]^t$dr$\le$$\int_0^n$f(r)dr = $\frac{n^t + 1}{t + 1}$

        e.E[$X_t$]$\le$$\sum$$(1 - r/n)^t$ = $\sum$$\frac{(n - r)^t}{n^t}$ = $\frac{1}{n^t}$$\sum$$(n - r)^t$

      and $\sum$$(n - r)^t$ = $(n - 2)^t$ + $(n - 2)^t$ + … + $1^t$ + $0^t$ = $\sum$$r^t$

      so E[$X_t$] = $\sum$$r^t$

        f.O($\frac{t + n}{t + 1}$) = O($\frac{t + n}{t}$)

        g.Since we have that for any number of iterations tt that the first algorithm takes to find its answer, the second algorithm will return it in time O($\frac{t + n}{t}$).In particular, if we just have that t = $\sqrt{n}$.The second algorithm takes time only O($\sqrt{n}$).This means that tihe first list search algorithm is
        O($\sqrt{n}$) as well.

        h.If we don't have distinct key values, then, we may randomly select an element that is further along than we had been before, but not jump to it because it has the same key as what we were currently at. The analysis will break when we try to bound the probability that$X_t$$\ge$i
\end{solution}
%%%%%%%%%%%%%%%

%%%%%%%%%%%%%%%%%%%%
\beginoptional

%%%%%%%%%%%%%%%
\begin{problem}[TC 10.1-7]
\end{problem}

\begin{solution}
\end{solution}
%%%%%%%%%%%%%%%

%%%%%%%%%%%%%%%%%%%%
\beginot

%%%%%%%%%%%%%%%
\begin{ot}[Stack \& Queue]

  Stack 与 Queue 之间的互模拟。

  \noindent 参考资料:
  \begin{itemize}
    \item TC 10.1-6, TC 10.1-7
    \item \href{https://cstheory.stackexchange.com/questions/2562/one-stack-two-queues}{One Stack, Two Queues @ cstheory}
  \end{itemize}
\end{ot}

% \begin{solution}
% \end{solution}
%%%%%%%%%%%%%%%
% \vspace{0.50cm}
%%%%%%%%%%%%%%%
\begin{ot}[Searching a sorted compact list]
  讲解 TC Problem 10-3.
\end{ot}

% \begin{solution}
% \end{solution}
%%%%%%%%%%%%%%%

%%%%%%%%%%%%%%%%%%%%
% 如果没有需要订正的题目,可以把这部分删掉

% \begincorrection
%%%%%%%%%%%%%%%%%%%%

%%%%%%%%%%%%%%%%%%%%
% 如果没有反馈,可以把这部分删掉
\beginfb

% 你可以写
% ~\footnote{优先推荐 \href{problemoverflow.top}{ProblemOverflow}}:
% \begin{itemize}
%   \item 对课程及教师的建议与意见
%   \item 教材中不理解的内容
%   \item 希望深入了解的内容
%   \item $\cdots$
% \end{itemize}
%%%%%%%%%%%%%%%%%%%%
% \bibliography{2-5-solving-recurrence}
% \bibliographystyle{plainnat}
%%%%%%%%%%%%%%%%%%%%
\end{document}