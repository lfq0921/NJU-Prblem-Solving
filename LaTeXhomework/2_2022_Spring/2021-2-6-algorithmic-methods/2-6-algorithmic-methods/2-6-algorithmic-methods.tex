% 2-6-algorithmic-methods.tex

%%%%%%%%%%%%%%%%%%%%
\documentclass[a4paper, justified]{tufte-handout}

% hw-preamble.tex

% geometry for A4 paper
% See https://tex.stackexchange.com/a/119912/23098
\geometry{
  left=20.0mm,
  top=20.0mm,
  bottom=20.0mm,
  textwidth=130mm, % main text block
  marginparsep=5.0mm, % gutter between main text block and margin notes
  marginparwidth=50.0mm % width of margin notes
}

% for colors
\usepackage{xcolor} % usage: \color{red}{text}
% predefined colors
\newcommand{\red}[1]{\textcolor{red}{#1}} % usage: \red{text}
\newcommand{\blue}[1]{\textcolor{blue}{#1}}
\newcommand{\teal}[1]{\textcolor{teal}{#1}}

\usepackage{todonotes}

% heading
\usepackage{sectsty}
\setcounter{secnumdepth}{2}
\allsectionsfont{\centering\huge\rmfamily}

% for Chinese
\usepackage{xeCJK}
\usepackage{zhnumber}
\setCJKmainfont[BoldFont=FandolSong-Bold.otf]{FandolSong-Regular.otf}

% for fonts
\usepackage{fontspec}
\newcommand{\song}{\CJKfamily{song}} 
\newcommand{\kai}{\CJKfamily{kai}} 

% To fix the ``MakeTextLowerCase'' bug:
% See https://github.com/Tufte-LaTeX/tufte-latex/issues/64#issuecomment-78572017
% Set up the spacing using fontspec features
\renewcommand\allcapsspacing[1]{{\addfontfeature{LetterSpace=15}#1}}
\renewcommand\smallcapsspacing[1]{{\addfontfeature{LetterSpace=10}#1}}

% for url
\usepackage{hyperref}
\hypersetup{colorlinks = true, 
  linkcolor = teal,
  urlcolor  = teal,
  citecolor = blue,
  anchorcolor = blue}

\newcommand{\me}[4]{
    \author{
      {\bfseries 姓名:}\underline{#1}\hspace{2em}
      {\bfseries 学号:}\underline{#2}\hspace{2em}\\[10pt]
      {\bfseries 评分:}\underline{#3\hspace{3em}}\hspace{2em}
      {\bfseries 评阅:}\underline{#4\hspace{3em}}
  }
}

% Please ALWAYS Keep This.
\newcommand{\noplagiarism}{
  \begin{center}
    \fbox{\begin{tabular}{@{}c@{}}
      请独立完成作业,不得抄袭。\\
      若得到他人帮助, 请致谢。\\
      若参考了其它资料,请给出引用。\\
      鼓励讨论,但需独立书写解题过程。
    \end{tabular}}
  \end{center}
}

\newcommand{\goal}[1]{
  \begin{center}{\fcolorbox{blue}{yellow!60}{\parbox{0.50\textwidth}{\large 
    \begin{itemize}
      \item 体会``思维的乐趣''
      \item 初步了解递归与数学归纳法 
      \item 初步接触算法概念与问题下界概念
    \end{itemize}}}}
  \end{center}
}

% Each hw consists of four parts:
\newcommand{\beginrequired}{\hspace{5em}\section{作业 (必做部分)}}
\newcommand{\beginoptional}{\section{作业 (选做部分)}}
\newcommand{\beginot}{\section{Open Topics}}
\newcommand{\begincorrection}{\section{订正}}
\newcommand{\beginfb}{\section{反馈}}

% for math
\usepackage{amsmath, mathtools, amsfonts, amssymb}
\newcommand{\set}[1]{\{#1\}}

% define theorem-like environments
\usepackage[amsmath, thmmarks]{ntheorem}

\theoremstyle{break}
\theorempreskip{2.0\topsep}
\theorembodyfont{\song}
\theoremseparator{}
\newtheorem{problem}{题目}[subsection]
\renewcommand{\theproblem}{\arabic{problem}}
\newtheorem{ot}{Open Topics}

\theorempreskip{3.0\topsep}
\theoremheaderfont{\kai\bfseries}
\theoremseparator{:}
\theorempostwork{\bigskip\hrule}
\newtheorem*{solution}{解答}
\theorempostwork{\bigskip\hrule}
\newtheorem*{revision}{订正}

\theoremstyle{plain}
\newtheorem*{cause}{错因分析}
\newtheorem*{remark}{注}

\theoremstyle{break}
\theorempostwork{\bigskip\hrule}
\theoremsymbol{\ensuremath{\Box}}
\newtheorem*{proof}{证明}

% \newcommand{\ot}{\blue{\bf [OT]}}

% for figs
\renewcommand\figurename{图}
\renewcommand\tablename{表}

% for fig without caption: #1: width/size; #2: fig file
\newcommand{\fig}[2]{
  \begin{figure}[htbp]
    \centering
    \includegraphics[#1]{#2}
  \end{figure}
}
% for fig with caption: #1: width/size; #2: fig file; #3: caption
\newcommand{\figcap}[3]{
  \begin{figure}[htbp]
    \centering
    \includegraphics[#1]{#2}
    \caption{#3}
  \end{figure}
}
% for fig with both caption and label: #1: width/size; #2: fig file; #3: caption; #4: label
\newcommand{\figcaplbl}[4]{
  \begin{figure}[htbp]
    \centering
    \includegraphics[#1]{#2}
    \caption{#3}
    \label{#4}
  \end{figure}
}
% for margin fig without caption: #1: width/size; #2: fig file
\newcommand{\mfig}[2]{
  \begin{marginfigure}
    \centering
    \includegraphics[#1]{#2}
  \end{marginfigure}
}
% for margin fig with caption: #1: width/size; #2: fig file; #3: caption
\newcommand{\mfigcap}[3]{
  \begin{marginfigure}
    \centering
    \includegraphics[#1]{#2}
    \caption{#3}
  \end{marginfigure}
}

\usepackage{fancyvrb}

% for algorithms
\usepackage[]{algorithm}
\usepackage[]{algpseudocode} % noend
% See [Adjust the indentation whithin the algorithmicx-package when a line is broken](https://tex.stackexchange.com/a/68540/23098)
\newcommand{\algparbox}[1]{\parbox[t]{\dimexpr\linewidth-\algorithmicindent}{#1\strut}}
\newcommand{\hStatex}[0]{\vspace{5pt}}
\makeatletter
\newlength{\trianglerightwidth}
\settowidth{\trianglerightwidth}{$\triangleright$~}
\algnewcommand{\LineComment}[1]{\Statex \hskip\ALG@thistlm \(\triangleright\) #1}
\algnewcommand{\LineCommentCont}[1]{\Statex \hskip\ALG@thistlm%
  \parbox[t]{\dimexpr\linewidth-\ALG@thistlm}{\hangindent=\trianglerightwidth \hangafter=1 \strut$\triangleright$ #1\strut}}
\makeatother

% for footnote/marginnote
% see https://tex.stackexchange.com/a/133265/23098
\usepackage{tikz}
\newcommand{\circled}[1]{%
  \tikz[baseline=(char.base)]
  \node [draw, circle, inner sep = 0.5pt, font = \tiny, minimum size = 8pt] (char) {#1};
}
\renewcommand\thefootnote{\protect\circled{\arabic{footnote}}} % feel free to modify this file
%%%%%%%%%%%%%%%%%%%%
\title{第6讲: 算法方法}
\me{林凡琪}{211240042}{}{}
\date{\zhtoday} % or like 2019年9月13日
%%%%%%%%%%%%%%%%%%%%
\begin{document}
\maketitle
%%%%%%%%%%%%%%%%%%%%
\noplagiarism % always keep this line
%%%%%%%%%%%%%%%%%%%%
\begin{abstract}
  % \begin{center}{\fcolorbox{blue}{yellow!60}{\parbox{0.65\textwidth}{\large 
  %   \begin{itemize}
  %     \item 
  %   \end{itemize}}}}
  % \end{center}
\end{abstract}
%%%%%%%%%%%%%%%%%%%%
\beginrequired

%%%%%%%%%%%%%%%
\begin{problem}[DH 4-8]
Prove that the maximal distance between any two points on a polygon occurs between two of the vertices.
\end{problem}

\begin{solution}
  Assume that the maximum distance between any two points on a polygon will not occur between two vertices\\
  The maximum distance between a and b of a polygon with y sides is assumed to have x sides.\\
  We know that the maximum distance between any two points on a polygon will not occur between two vertices.\\
  So we can conclude that x$\to$b can form a polygon with (y - x + 1) sides.\\
  Then the problem can be described as the minimum distance of two points is not a straight line, we know that this does not work in all cases.\\
  Therefore, we can show that the maximum distance between any two points on a polygon occurs between two vertices.
\end{solution}
%%%%%%%%%%%%%%%

%%%%%%%%%%%%%%%
\begin{problem}[DH 4-9]
Write a program implementing the maximal polygonal distance algorithm
\end{problem}

\begin{solution}
  \noindent
  \begin{algorithm}
    \begin{algorithmic}[1]
      \Procedure{max}{$P = \{p_1,...,p_n\}$}
      \State p0 = pn;
      \State q = NEXT[p];
      \While {Area(p, NEXT[p], NEXT[q]) > Area(p, NEXT[p], q)}
      \State q = NEXT[q];
      \State q0 = q;
      \While {q != p0}
      \State p = NEXT[p];
      \State print(p, q);
      \While {Area(p, NEXT[p], NEXT[q]) > Area(p, NEXT[p], q)}
      \State q = NEXT[q];
      \If{(p, q) != (q0, p0)}
      \State print(p, q);
      \Else return;
      \EndIf
      \EndWhile
      \If{Area(p, NEXT[p], NEXT[q]) = Area(p, NEXT[p], q)}
      \State \If{(p,q)!=(q0,p0)}
      \State print(p, NEXT[q]);
      \Else print(NEXT[p],q)
      \EndIf
      \EndIf
      \EndWhile
      \EndWhile
      \EndProcedure
    \end{algorithmic}
  \end{algorithm}
  The input is a polygon $P = \{p_1,...,p_n\}$.
  致谢csdn博主(伪代码在下一页)
\end{solution}
%%%%%%%%%%%%%%%

%%%%%%%%%%%%%%%
\begin{problem}[DH 4-12]
Write high-level pseudocode of the greedy algorithm described in the text
for finding a minimal spanning tree.
\end{problem}

\begin{solution}
  \noindent
  \begin{algorithm}
    \begin{algorithmic}[2]
      \Procedure{greedy}{$C, Q[], W[], P[]$}
      \State profit$\gets$0
      \While{C $\neq$0}
      \State max$\gets$0
      \State I$\gets$0
      \For{i from 1 to N}
      \If{P[i] / W[i] > max and Q[i] $\neq$ 0}
      \State max = P[i] / W[i]
      \State I = i
      \EndIf
      \EndFor
      \State C = C - W[I]
      \State profit = profit + P[I]
      \EndWhile
      \State \Return profit
      \EndProcedure
    \end{algorithmic}
  \end{algorithm}
\end{solution}
%%%%%%%%%%%%%%%

%%%%%%%%%%%%%%%
\begin{problem}[DH 4-13]
\end{problem}

\begin{solution}
  (a)\\
  \noindent
  \begin{algorithm}
    \begin{algorithmic}[1]
      \Procedure{DP}{$C, N, Q[], w[], p[]$}
      \State dp[0,...,C] = 0
      \For{$i \gets 1, N$}
      \For{$j \gets C, w[i]$}
      \For{$k \gets 0, min(q[i], j/w[i])$}
      \State dp[j] = max(dp[j], dp[j - k * w[i]] + k*p[i])
      \EndFor
      \EndFor
      \EndFor
      \State print(dp[C])
      \EndProcedure
    \end{algorithmic}
  \end{algorithm}
  (b)The maximal profit is 194.
\end{solution}
%%%%%%%%%%%%%%%

%%%%%%%%%%%%%%%%%%%%
\beginoptional

%%%%%%%%%%%%%%%
\begin{problem}[DH 4-10]
\end{problem}

\begin{solution}
\end{solution}
%%%%%%%%%%%%%%%

%%%%%%%%%%%%%%%%%%%%
\beginot

%%%%%%%%%%%%%%%
本周 OT 关注搜索技术。

\begin{ot}[Alpha–Beta Pruning]
  请介绍 Alpha-Beta 剪枝技术,包括概念、方法、应用 (比如在双人游戏中) 等。

  \noindent 参考资料:
  \begin{itemize}
    \item \href{https://en.wikipedia.org/wiki/Alpha\%E2\%80\%93beta\_pruning}{Alpha–beta pruning @ wiki}
    \item \href{https://github.com/hengxin/problem-solving-class-paperswelove/blob/master/2nd-semester/Knuth\%20(AI\%2C\%201975)\%20An\%20Analysis\%20of\%20Alpha-Beta\%20Pruning.pdf}{``An Analysis of Alpha-Beta Pruning'' @ AI'1975} (可选)
  \end{itemize}
\end{ot}

\begin{solution}
\end{solution}
%%%%%%%%%%%%%%%
% \vspace{0.50cm}
%%%%%%%%%%%%%%%
\begin{ot}[SAT Solver]
  请介绍 \href{https://en.wikipedia.org/wiki/SAT\_solver}{SAT} 的求解算法。

  \noindent 参考资料
  \begin{itemize}
    \item \href{https://en.wikipedia.org/wiki/SAT\_solver#Algorithms\_for\_solving\_SAT}{Solving SAT @ wiki}
    \item \href{https://en.wikipedia.org/wiki/DPLL\_algorithm}{DPLL algorithm @ wiki}
    \item \href{https://yurichev.com/writings/SAT\_SMT\_by\_example.pdf}{Examples} (可选)
  \end{itemize}
\end{ot}

% \begin{solution}
% \end{solution}
%%%%%%%%%%%%%%%

%%%%%%%%%%%%%%%%%%%%
% 如果没有需要订正的题目,可以把这部分删掉

% \begincorrection
%%%%%%%%%%%%%%%%%%%%

%%%%%%%%%%%%%%%%%%%%
% 如果没有反馈,可以把这部分删掉
\beginfb

% 你可以写
% ~\footnote{优先推荐 \href{problemoverflow.top}{ProblemOverflow}}:
% \begin{itemize}
%   \item 对课程及教师的建议与意见
%   \item 教材中不理解的内容
%   \item 希望深入了解的内容
%   \item $\cdots$
% \end{itemize}
%%%%%%%%%%%%%%%%%%%%
\bibliography{2-5-solving-recurrence}
\bibliographystyle{plainnat}
%%%%%%%%%%%%%%%%%%%%
\end{document}