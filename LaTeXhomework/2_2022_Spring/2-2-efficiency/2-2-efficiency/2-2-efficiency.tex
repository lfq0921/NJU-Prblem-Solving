% 2-2-efficiency.tex

%%%%%%%%%%%%%%%%%%%%
\documentclass[a4paper, justified]{tufte-handout}

% hw-preamble.tex

% geometry for A4 paper
% See https://tex.stackexchange.com/a/119912/23098
\geometry{
  left=20.0mm,
  top=20.0mm,
  bottom=20.0mm,
  textwidth=130mm, % main text block
  marginparsep=5.0mm, % gutter between main text block and margin notes
  marginparwidth=50.0mm % width of margin notes
}

% for colors
\usepackage{xcolor} % usage: \color{red}{text}
% predefined colors
\newcommand{\red}[1]{\textcolor{red}{#1}} % usage: \red{text}
\newcommand{\blue}[1]{\textcolor{blue}{#1}}
\newcommand{\teal}[1]{\textcolor{teal}{#1}}

\usepackage{todonotes}

% heading
\usepackage{sectsty}
\setcounter{secnumdepth}{2}
\allsectionsfont{\centering\huge\rmfamily}

% for Chinese
\usepackage{xeCJK}
\usepackage{zhnumber}
\setCJKmainfont[BoldFont=FandolSong-Bold.otf]{FandolSong-Regular.otf}

% for fonts
\usepackage{fontspec}
\newcommand{\song}{\CJKfamily{song}} 
\newcommand{\kai}{\CJKfamily{kai}} 

% To fix the ``MakeTextLowerCase'' bug:
% See https://github.com/Tufte-LaTeX/tufte-latex/issues/64#issuecomment-78572017
% Set up the spacing using fontspec features
\renewcommand\allcapsspacing[1]{{\addfontfeature{LetterSpace=15}#1}}
\renewcommand\smallcapsspacing[1]{{\addfontfeature{LetterSpace=10}#1}}

% for url
\usepackage{hyperref}
\hypersetup{colorlinks = true, 
  linkcolor = teal,
  urlcolor  = teal,
  citecolor = blue,
  anchorcolor = blue}

\newcommand{\me}[4]{
    \author{
      {\bfseries 姓名:}\underline{#1}\hspace{2em}
      {\bfseries 学号:}\underline{#2}\hspace{2em}\\[10pt]
      {\bfseries 评分:}\underline{#3\hspace{3em}}\hspace{2em}
      {\bfseries 评阅:}\underline{#4\hspace{3em}}
  }
}

% Please ALWAYS Keep This.
\newcommand{\noplagiarism}{
  \begin{center}
    \fbox{\begin{tabular}{@{}c@{}}
      请独立完成作业,不得抄袭。\\
      若得到他人帮助, 请致谢。\\
      若参考了其它资料,请给出引用。\\
      鼓励讨论,但需独立书写解题过程。
    \end{tabular}}
  \end{center}
}

\newcommand{\goal}[1]{
  \begin{center}{\fcolorbox{blue}{yellow!60}{\parbox{0.50\textwidth}{\large 
    \begin{itemize}
      \item 体会``思维的乐趣''
      \item 初步了解递归与数学归纳法 
      \item 初步接触算法概念与问题下界概念
    \end{itemize}}}}
  \end{center}
}

% Each hw consists of four parts:
\newcommand{\beginrequired}{\hspace{5em}\section{作业 (必做部分)}}
\newcommand{\beginoptional}{\section{作业 (选做部分)}}
\newcommand{\beginot}{\section{Open Topics}}
\newcommand{\begincorrection}{\section{订正}}
\newcommand{\beginfb}{\section{反馈}}

% for math
\usepackage{amsmath, mathtools, amsfonts, amssymb}
\newcommand{\set}[1]{\{#1\}}

% define theorem-like environments
\usepackage[amsmath, thmmarks]{ntheorem}

\theoremstyle{break}
\theorempreskip{2.0\topsep}
\theorembodyfont{\song}
\theoremseparator{}
\newtheorem{problem}{题目}[subsection]
\renewcommand{\theproblem}{\arabic{problem}}
\newtheorem{ot}{Open Topics}

\theorempreskip{3.0\topsep}
\theoremheaderfont{\kai\bfseries}
\theoremseparator{:}
\theorempostwork{\bigskip\hrule}
\newtheorem*{solution}{解答}
\theorempostwork{\bigskip\hrule}
\newtheorem*{revision}{订正}

\theoremstyle{plain}
\newtheorem*{cause}{错因分析}
\newtheorem*{remark}{注}

\theoremstyle{break}
\theorempostwork{\bigskip\hrule}
\theoremsymbol{\ensuremath{\Box}}
\newtheorem*{proof}{证明}

% \newcommand{\ot}{\blue{\bf [OT]}}

% for figs
\renewcommand\figurename{图}
\renewcommand\tablename{表}

% for fig without caption: #1: width/size; #2: fig file
\newcommand{\fig}[2]{
  \begin{figure}[htbp]
    \centering
    \includegraphics[#1]{#2}
  \end{figure}
}
% for fig with caption: #1: width/size; #2: fig file; #3: caption
\newcommand{\figcap}[3]{
  \begin{figure}[htbp]
    \centering
    \includegraphics[#1]{#2}
    \caption{#3}
  \end{figure}
}
% for fig with both caption and label: #1: width/size; #2: fig file; #3: caption; #4: label
\newcommand{\figcaplbl}[4]{
  \begin{figure}[htbp]
    \centering
    \includegraphics[#1]{#2}
    \caption{#3}
    \label{#4}
  \end{figure}
}
% for margin fig without caption: #1: width/size; #2: fig file
\newcommand{\mfig}[2]{
  \begin{marginfigure}
    \centering
    \includegraphics[#1]{#2}
  \end{marginfigure}
}
% for margin fig with caption: #1: width/size; #2: fig file; #3: caption
\newcommand{\mfigcap}[3]{
  \begin{marginfigure}
    \centering
    \includegraphics[#1]{#2}
    \caption{#3}
  \end{marginfigure}
}

\usepackage{fancyvrb}

% for algorithms
\usepackage[]{algorithm}
\usepackage[]{algpseudocode} % noend
% See [Adjust the indentation whithin the algorithmicx-package when a line is broken](https://tex.stackexchange.com/a/68540/23098)
\newcommand{\algparbox}[1]{\parbox[t]{\dimexpr\linewidth-\algorithmicindent}{#1\strut}}
\newcommand{\hStatex}[0]{\vspace{5pt}}
\makeatletter
\newlength{\trianglerightwidth}
\settowidth{\trianglerightwidth}{$\triangleright$~}
\algnewcommand{\LineComment}[1]{\Statex \hskip\ALG@thistlm \(\triangleright\) #1}
\algnewcommand{\LineCommentCont}[1]{\Statex \hskip\ALG@thistlm%
  \parbox[t]{\dimexpr\linewidth-\ALG@thistlm}{\hangindent=\trianglerightwidth \hangafter=1 \strut$\triangleright$ #1\strut}}
\makeatother

% for footnote/marginnote
% see https://tex.stackexchange.com/a/133265/23098
\usepackage{tikz}
\newcommand{\circled}[1]{%
  \tikz[baseline=(char.base)]
  \node [draw, circle, inner sep = 0.5pt, font = \tiny, minimum size = 8pt] (char) {#1};
}
\renewcommand\thefootnote{\protect\circled{\arabic{footnote}}} % feel free to modify this file
%%%%%%%%%%%%%%%%%%%%
\title{第2讲: 算法的效率}
\me{林凡琪}{211240042}{}{}
\date{\zhtoday} % or like 2019年9月13日
%%%%%%%%%%%%%%%%%%%%
\begin{document}
\maketitle
%%%%%%%%%%%%%%%%%%%%
\noplagiarism % always keep this line
%%%%%%%%%%%%%%%%%%%%
\begin{abstract}
  % \mfigcap{width = 0.85\textwidth}{figs/George-Boole}{George Boole}
  % \begin{center}{\fcolorbox{blue}{yellow!60}{\parbox{0.65\textwidth}{\large 
  %   \begin{itemize}
  %     \item 
  %   \end{itemize}}}}
  % \end{center}
\end{abstract}
%%%%%%%%%%%%%%%%%%%%
\beginrequired

%%%%%%%%%%%%%%%
\begin{problem}[DH Problem 6.18: $\log_{m} n$]
\end{problem}

\begin{solution}
  \begin{algorithm}
    \caption{$\log_{m} n$}
    \label{alg:sum}
    \begin{algorithmic}[1]
      \Function{LG1}{m, n}
      \State $k = 0$
      \State $res = 1$
      \While{$res * m <= n$}
      \State $res = res * m$
      \State $k = k + 1$
      \EndWhile
      \State return $res$
      \EndFunction
    \end{algorithmic}
  \end{algorithm}

  Time complexity: $O (\log_m n)$\\
  Space complexity: $O(1)$\\
\end{solution}
%%%%%%%%%%%%%%%

%%%%%%%%%%%%%%%
\begin{problem}[DH Problem 6.19: $\log_{m} n$]
\end{problem}

\begin{solution}
  \begin{algorithm}
    \caption{$\log_{m} n$}
    \label{alg:sum}
    \begin{algorithmic}[1]
      \Function{LG2}{m, n}
      \State $k = 0$
      \State $a = 1$
      \While{$m * a <= n$}
      \State $k_1 = 1 $
      \State $b = m * m$
      \State $c = m * a$
      \While{$a * b <= n$}
      \State $c = a$
      \State $a = b * c$
      \State $k_1 = 2k_1$
      \State $b = b * b$
      \EndWhile
      \State a = c
      \State $k = k + k_1$
      \EndWhile
      \State return $k$
      \EndFunction
    \end{algorithmic}
  \end{algorithm}
  Time complexity:$O ((\log \log n)^2)$\\
  Space complexity: $O(1)$
\end{solution}


%%%%%%%%%%%%%%%

%%%%%%%%%%%%%%%
\begin{problem}[DH Problem 6.20 (a): $\log_{m} n$]
\end{problem}

\begin{solution}
  \begin{algorithm}
    \caption{LG2}
    \label{alg:sum}
    \begin{algorithmic}[1]
      \Procedure{LG2}{m,n}
      \State $res[0] = m$
      \State $count = 0$
      \While{$res[count] * res[count] <= n$}
      \State $res[count + 1] = res[count] * res[count]$
      \State $count = count + 1$
      \EndWhile
      \State $sum = res[count]$
      \State $ans = pow(2, count)$
      \While{$count >= 1$}
      \If {$sum * res[count - 1] <= n$ AND $sum * res[count - 1] * m * m > n$}
      \State $ans = ans + pow(2, count - 1)$
      \State $sum = sum * res[count - 1]$
      \EndIf
      \State $count = count - 1$
      \EndWhile
      \State return ans
      \EndProcedure
    \end{algorithmic}
  \end{algorithm}
  Time complexity:$O (\log \log n)$\\
  Space complexity: $O(\log n)$
\end{solution}
%%%%%%%%%%%%%%%

%%%%%%%%%%%%%%%
\begin{problem}[TC Exercise 3.1-6]
\end{problem}

\begin{solution}
  充分性:\\
  设该算法的运行时间为$T(n) = \Theta(n) \rightarrow \exists c_1, c_2, n_0, \forall n > n_0, c_1 n <= T(n) <= c_2 n$ \\
  所以$\exists c_1, n_0, \forall n > n_0, c_1 n <= T(n) \rightarrow T(n) = \Omega (n)$即最好情况运行时间是$\Omega (n)$\\
  同样,$\exists c_2, n_0, \forall n > n_0,T(n) <= c_2 n,\rightarrow T(n) = O(n)$即最坏情况运行时间为$O(n)$\\
  必要性:\\
  设该算法运行时间为$T(n) \in O(n)$并且$T(n) \in \Omega(n)$\\
  即$\exists c_1, n_0, \forall n > n_0, c_1 n <= T(n) \rightarrow T(n) = \Omega (n)$且$\exists c_2, n_0, \forall n > n_0,T(n) <= c_2 n,\rightarrow T(n) = O(n)$.\\
  综合两点可得$T(n) = \Theta(n) \rightarrow \exists c_1, c_2, n_0, \forall n > n_0, c_1 n <= T(n) <= c_2 n$\\
  综上,证毕.
\end{solution}
%%%%%%%%%%%%%%%

%%%%%%%%%%%%%%%
\begin{problem}[TC Exercise 3.1-7]
\end{problem}

\begin{solution}
  $o(g(n)) = \{f(n)|\lim_{n \to \infty} \frac{f(n)}{g(n)} = 0\}$\\
  $\omega (g(n)) = \{f(n) | \lim_{n \to \infty} \frac{f(n)}{g(n)} \rightarrow \infty\}$\\
  $\Rightarrow o(g(n)) \cap \omega (g(n)) = \{f(n) | \lim_{n \to \infty} \frac{f(n)}{g(n)} \rightarrow \infty, \lim_{n \to \infty} \frac{f(n)}{g(n)} = 0\}$\\
  显然,$0 \neq \infty$.所以一定不存在符合条件的f(n),所以$o(g(n)) \cap \omega (g(n))$为空集
\end{solution}
%%%%%%%%%%%%%%%

%%%%%%%%%%%%%%%
\begin{problem}[TC Problem 3-3 (a)]
\end{problem}

\begin{solution}

  答案被移到下面去了...
  \begin{table}[]
    \begin{tabular}{|l|l|}
      \hline
      等价类序号 & 函数                                          \\ \hline
      1          & $g_1 = 2^{2^{n + 1}}$                         \\ \hline
      2          & $g_2 = 2^{2^{n}}$                             \\ \hline
      3          & $g_3 = (n + 1)!$                              \\ \hline
      4          & $g_4 = n!$                                    \\ \hline
      5          & $g_5 = e^n$                                   \\ \hline
      6          & $g_6 = n2^n$                                  \\ \hline
      7          & $g_7 = 2 ^n$                                  \\ \hline
      8          & $g_8 = (\frac{3}{2})^n$                       \\ \hline
      9          & $g_9 = n^{\lg \lg n},g_10 = (\lg n) ^{\lg n}$ \\ \hline
      10         & $g_{11} = (\lg n)!$                           \\ \hline
      11         & $g_{12} = n^3$                                \\ \hline
      12         & $g_{13} = n^2,g_{14} = 4^{\lg n}$             \\ \hline
      13         & $g_15 = n\lg n, g_{16} = lg(n!)$              \\ \hline
      14         & $g_{17} = n,g_{18} = 2^{\lg n}$               \\ \hline
      15         & $g_{19} = (\sqrt{2})^{\lg n}$                 \\ \hline
      16         & $g_{20} = 2^{\sqrt{2 \lg n}}$                 \\ \hline
      17         & $g_{21} = \lg^2n$                             \\ \hline
      18         & $g_{22} = \ln n$                              \\ \hline
      19         & $g_{23} = \sqrt{\lg n}$                       \\ \hline
      20         & $g_{24} = \ln \ln n$                          \\ \hline
      21         & $g_{25} = 2^{\lg^*n}$                         \\ \hline
      22         & $g_{26} = \lg^*(\lg n), g_{27} = \lg^*n$      \\ \hline
      23         & $g_{28} = \lg(\lg^*n)$                        \\ \hline
      24         & $g_{29} = n^{1/\lg n}$                        \\ \hline
    \end{tabular}
  \end{table}

\end{solution}
%%%%%%%%%%%%%%%

%%%%%%%%%%%%%%%%%%%%
\beginoptional

%%%%%%%%%%%%%%%
\begin{problem}[DH Problem 6.13]
\end{problem}

\begin{solution}
\end{solution}
%%%%%%%%%%%%%%%

%%%%%%%%%%%%%%%%%%%%
\beginot

%%%%%%%%%%%%%%%
本次 OT 介绍两种证明问题下界的常用技术。

\begin{ot}[Decision Trees]
  介绍 Decison Trees (决策树) 的概念以及在证明问题下界时的应用
  (包括但不限于本次选做题 DH 6.13)。

  参考资料:
  \begin{itemize}
    \item \href{https://en.wikipedia.org/wiki/Decision\_tree\_model}{Decision tree model @ wiki}
    \item \href{http://jeffe.cs.illinois.edu/teaching/algorithms/notes/12-lowerbounds.pdf}{lecture-note by jeffe}
  \end{itemize}
\end{ot}

% \begin{solution}
% \end{solution}
%%%%%%%%%%%%%%%
\vspace{0.50cm}
%%%%%%%%%%%%%%%
\begin{ot}[Adversary Argument]
  介绍 Adversary Argument (对手论证) 的概念
  以及在证明问题下界时的应用。

  \begin{itemize}
    \item \href{http://jeffe.cs.illinois.edu/teaching/algorithms/notes/13-adversary.pdf}{lecture-note by jeffe}
  \end{itemize}
\end{ot}

% \begin{solution}
% \end{solution}
%%%%%%%%%%%%%%%

%%%%%%%%%%%%%%%%%%%%
% 如果没有需要订正的题目,可以把这部分删掉

% \begincorrection
%%%%%%%%%%%%%%%%%%%%

%%%%%%%%%%%%%%%%%%%%
% 如果没有反馈,可以把这部分删掉
\beginfb

% 你可以写
% ~\footnote{优先推荐 \href{problemoverflow.top}{ProblemOverflow}}:
% \begin{itemize}
%   \item 对课程及教师的建议与意见
%   \item 教材中不理解的内容
%   \item 希望深入了解的内容
%   \item $\cdots$
% \end{itemize}
%%%%%%%%%%%%%%%%%%%%
\end{document}