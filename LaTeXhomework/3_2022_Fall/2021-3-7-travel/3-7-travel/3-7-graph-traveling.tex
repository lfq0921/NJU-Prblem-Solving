% 2-15-rb-tree.tex

%%%%%%%%%%%%%%%%%%%%
\documentclass[a4paper, justified]{tufte-handout}

% hw-preamble.tex

% geometry for A4 paper
% See https://tex.stackexchange.com/a/119912/23098
\geometry{
  left=20.0mm,
  top=20.0mm,
  bottom=20.0mm,
  textwidth=130mm, % main text block
  marginparsep=5.0mm, % gutter between main text block and margin notes
  marginparwidth=50.0mm % width of margin notes
}

% for colors
\usepackage{xcolor} % usage: \color{red}{text}
% predefined colors
\newcommand{\red}[1]{\textcolor{red}{#1}} % usage: \red{text}
\newcommand{\blue}[1]{\textcolor{blue}{#1}}
\newcommand{\teal}[1]{\textcolor{teal}{#1}}

\usepackage{todonotes}

% heading
\usepackage{sectsty}
\setcounter{secnumdepth}{2}
\allsectionsfont{\centering\huge\rmfamily}

% for Chinese
\usepackage{xeCJK}
\usepackage{zhnumber}
\setCJKmainfont[BoldFont=FandolSong-Bold.otf]{FandolSong-Regular.otf}

% for fonts
\usepackage{fontspec}
\newcommand{\song}{\CJKfamily{song}} 
\newcommand{\kai}{\CJKfamily{kai}} 

% To fix the ``MakeTextLowerCase'' bug:
% See https://github.com/Tufte-LaTeX/tufte-latex/issues/64#issuecomment-78572017
% Set up the spacing using fontspec features
\renewcommand\allcapsspacing[1]{{\addfontfeature{LetterSpace=15}#1}}
\renewcommand\smallcapsspacing[1]{{\addfontfeature{LetterSpace=10}#1}}

% for url
\usepackage{hyperref}
\hypersetup{colorlinks = true, 
  linkcolor = teal,
  urlcolor  = teal,
  citecolor = blue,
  anchorcolor = blue}

\newcommand{\me}[4]{
    \author{
      {\bfseries 姓名:}\underline{#1}\hspace{2em}
      {\bfseries 学号:}\underline{#2}\hspace{2em}\\[10pt]
      {\bfseries 评分:}\underline{#3\hspace{3em}}\hspace{2em}
      {\bfseries 评阅:}\underline{#4\hspace{3em}}
  }
}

% Please ALWAYS Keep This.
\newcommand{\noplagiarism}{
  \begin{center}
    \fbox{\begin{tabular}{@{}c@{}}
      请独立完成作业,不得抄袭。\\
      若得到他人帮助, 请致谢。\\
      若参考了其它资料,请给出引用。\\
      鼓励讨论,但需独立书写解题过程。
    \end{tabular}}
  \end{center}
}

\newcommand{\goal}[1]{
  \begin{center}{\fcolorbox{blue}{yellow!60}{\parbox{0.50\textwidth}{\large 
    \begin{itemize}
      \item 体会``思维的乐趣''
      \item 初步了解递归与数学归纳法 
      \item 初步接触算法概念与问题下界概念
    \end{itemize}}}}
  \end{center}
}

% Each hw consists of four parts:
\newcommand{\beginrequired}{\hspace{5em}\section{作业 (必做部分)}}
\newcommand{\beginoptional}{\section{作业 (选做部分)}}
\newcommand{\beginot}{\section{Open Topics}}
\newcommand{\begincorrection}{\section{订正}}
\newcommand{\beginfb}{\section{反馈}}

% for math
\usepackage{amsmath, mathtools, amsfonts, amssymb}
\newcommand{\set}[1]{\{#1\}}

% define theorem-like environments
\usepackage[amsmath, thmmarks]{ntheorem}

\theoremstyle{break}
\theorempreskip{2.0\topsep}
\theorembodyfont{\song}
\theoremseparator{}
\newtheorem{problem}{题目}[subsection]
\renewcommand{\theproblem}{\arabic{problem}}
\newtheorem{ot}{Open Topics}

\theorempreskip{3.0\topsep}
\theoremheaderfont{\kai\bfseries}
\theoremseparator{:}
\theorempostwork{\bigskip\hrule}
\newtheorem*{solution}{解答}
\theorempostwork{\bigskip\hrule}
\newtheorem*{revision}{订正}

\theoremstyle{plain}
\newtheorem*{cause}{错因分析}
\newtheorem*{remark}{注}

\theoremstyle{break}
\theorempostwork{\bigskip\hrule}
\theoremsymbol{\ensuremath{\Box}}
\newtheorem*{proof}{证明}

% \newcommand{\ot}{\blue{\bf [OT]}}

% for figs
\renewcommand\figurename{图}
\renewcommand\tablename{表}

% for fig without caption: #1: width/size; #2: fig file
\newcommand{\fig}[2]{
  \begin{figure}[htbp]
    \centering
    \includegraphics[#1]{#2}
  \end{figure}
}
% for fig with caption: #1: width/size; #2: fig file; #3: caption
\newcommand{\figcap}[3]{
  \begin{figure}[htbp]
    \centering
    \includegraphics[#1]{#2}
    \caption{#3}
  \end{figure}
}
% for fig with both caption and label: #1: width/size; #2: fig file; #3: caption; #4: label
\newcommand{\figcaplbl}[4]{
  \begin{figure}[htbp]
    \centering
    \includegraphics[#1]{#2}
    \caption{#3}
    \label{#4}
  \end{figure}
}
% for margin fig without caption: #1: width/size; #2: fig file
\newcommand{\mfig}[2]{
  \begin{marginfigure}
    \centering
    \includegraphics[#1]{#2}
  \end{marginfigure}
}
% for margin fig with caption: #1: width/size; #2: fig file; #3: caption
\newcommand{\mfigcap}[3]{
  \begin{marginfigure}
    \centering
    \includegraphics[#1]{#2}
    \caption{#3}
  \end{marginfigure}
}

\usepackage{fancyvrb}

% for algorithms
\usepackage[]{algorithm}
\usepackage[]{algpseudocode} % noend
% See [Adjust the indentation whithin the algorithmicx-package when a line is broken](https://tex.stackexchange.com/a/68540/23098)
\newcommand{\algparbox}[1]{\parbox[t]{\dimexpr\linewidth-\algorithmicindent}{#1\strut}}
\newcommand{\hStatex}[0]{\vspace{5pt}}
\makeatletter
\newlength{\trianglerightwidth}
\settowidth{\trianglerightwidth}{$\triangleright$~}
\algnewcommand{\LineComment}[1]{\Statex \hskip\ALG@thistlm \(\triangleright\) #1}
\algnewcommand{\LineCommentCont}[1]{\Statex \hskip\ALG@thistlm%
  \parbox[t]{\dimexpr\linewidth-\ALG@thistlm}{\hangindent=\trianglerightwidth \hangafter=1 \strut$\triangleright$ #1\strut}}
\makeatother

% for footnote/marginnote
% see https://tex.stackexchange.com/a/133265/23098
\usepackage{tikz}
\newcommand{\circled}[1]{%
  \tikz[baseline=(char.base)]
  \node [draw, circle, inner sep = 0.5pt, font = \tiny, minimum size = 8pt] (char) {#1};
}
\renewcommand\thefootnote{\protect\circled{\arabic{footnote}}} % feel free to modify this file
%%%%%%%%%%%%%%%%%%%%
\title{第3-7讲: 图的遍历}
\me{林凡琪}{211240042}{}{}
\date{\zhtoday} % or like 2019年9月13日
%%%%%%%%%%%%%%%%%%%%
\begin{document}
\maketitle
%%%%%%%%%%%%%%%%%%%%
\noplagiarism % always keep this line
%%%%%%%%%%%%%%%%%%%%
\begin{abstract}
  % \begin{center}{\fcolorbox{blue}{yellow!60}{\parbox{0.65\textwidth}{\large 
  %   \begin{itemize}
  %     \item 
  %   \end{itemize}}}}
  % \end{center}
\end{abstract}
%%%%%%%%%%%%%%%%%%%%
\beginrequired

%%%%%%%%%%%%%%%
\begin{problem}[TC 22.1-3]
\end{problem}

\begin{solution}
  对于邻接矩阵:$G = G^T$所以不用操作\\
  对于邻接链表:
  \begin{algorithm}[H]
    \begin{algorithmic}[1]
      \Procedure{GT}{$n$}
      \State Bdj[1..V];
      \For {$i = 1, i <= V; i++$}
      \For {$j = 1; j <= V_of_adj[i]; j++$}
      \State $Bdj[Adj[i][j]] \gets i$
      \EndFor
      \EndFor
      \EndProcedure
    \end{algorithmic}
  \end{algorithm}
  时间复杂度:$O(|E|+|V|)$

\end{solution}
%%%%%%%%%%%%%%%

%%%%%%%%%%%%%%%
\begin{problem}[TC 22.1-8]
Suppose that instead of a linked list, each array entry $Adj[u]$ is a hash table containing the vertices $v$ for which $(u, v) \in E$. If all edge lookups are equally likely, what is the expected time to determine whether an edge is in the graph? What disadvantages does this scheme have? Suggest an alternate data structure for each edge list that solves these problems. Does your alternative have disadvantages compared to the hash table?
假设每个数组条目 $Adj[u]$ 不是链接列表,而是一个哈希表,其中包含顶点 $v$,其中 $(u, v) in E$。如果所有边缘查找的可能性相等,则确定边缘是否在图形中的预期时间是多少?这个方案有什么缺点?为每个边缘列表建议一个替代数据结构来解决这些问题。与哈希表相比,您的替代方案是否有缺点?
\end{problem}

\begin{solution}
  期望查找时间是$O(1)$,最差查找时间是$O(|V|)$ \\
  替代方案:\\
  首先把在散列表内的点都排序,然后用二分法查找,这样期望查找时间就是$O(lg|V|)$。\\
  缺点是期望查找时间变长,并且需要时间进行排序。
\end{solution}
%%%%%%%%%%%%%%%

%%%%%%%%%%%%%%%
\begin{problem}[TC 22.2-3("lines 5 and 14" 改为"line 18")]
\end{problem}

\begin{solution}
  将18行删除后,节点不再变成黑色。但是同样不会进入line 13的if判断,不会造成重复判断。\\
  所以用两个颜色标明就可以。
\end{solution}
%%%%%%%%%%%%%%%

%%%%%%%%%%%%%%%
\begin{problem}[TC 22.2-4]
\end{problem}

\begin{solution}
  因为边的数目变成了$V^2$,所以查询边的时间$O(V^2)$\\
  运行时间$O(V+V^2) = O(V^2)$
\end{solution}
%%%%%%%%%%%%%%%

%%%%%%%%%%%%%%%
\begin{problem}[TC 22.2-5]
\end{problem}

\begin{solution}
  (1)证明:根据定理22.5(广度优先算法的正确性)$v.d$是s到v的最短路径,所以不会因为在邻接链表里的顺序不同而改变。\\
  (2)证明:在图22-3可见,$t.d = x.d$,且$u$为他们俩共同拥有的直系后代。因为t的顺序比x前,所以在生成的广度优先树中,u是t的后裔;倘若x的顺序比t早,那么u就会是x的后裔。
\end{solution}
%%%%%%%%%%%%%%%

%%%%%%%%%%%%%%%
\begin{problem}[TC 22.3-6]
证明:再无向图中,根据深度优先搜索算法是先探索(u, v)还是先探索(v,u)来将边(u,v)分类为树边或者后向边,与根据分类列表中的4中类型的次序进行分类是等价的。\\
\end{problem}

\begin{solution}

  根据定理 22.10,无向图的每条边要么是树边,要么是后向边。首先假设 $v$ 是通过探索边缘 $(u, v)$ 首先发现的。然后根据定义,$(u, v)$ 是树边。此外,$(u, v)$ 必须在 $(u, v)$ 之前被发现,因为一旦探索了 $(u, v)$ ,就必然发现了$v$。现在假设$v$ 不是首先被 $(u, v)$ 发现的。然后它必须被$(r, v)$发现,$r \ne u$\\

  如果$u$尚未被发现,那么如果首先探索$(u,v)$,那么它必须是一个后向边,因为$v$是$u$的祖先。如果发现了$u$,那么$u$是$v$的祖先,所以$(v, u)$是后向边。\\
\end{solution}
%%%%%%%%%%%%%%%


%%%%%%%%%%%%%%%
\begin{problem}[TC 22.3-8]
Give a counterexample to the conjecture that if a directed graph $G$ contains a path from $u$ to $v$, and if $u.d < v.d$ in a depth-first search of $G$, then $v$ is a descendant of $u$ in the depth-first forest produced.
\end{problem}

\begin{solution}
  Consider a graph with 3 vertices u, v, and w, and with edges (w, u), (u, w), and (w, v). Suppose that \text{DFS} first explores w, and that w's adjacency list has u before v. We next discover u. The only adjacent vertex is w, but w is already grey, so u finishes. Since v is not yet a descendant of u and u is finished, v can never be a descendant of uu.
\end{solution}
%%%%%%%%%%%%%%%

%%%%%%%%%%%%%%%
\begin{problem}[TC 22.3-9]
Give a counterexample to the conjecture that if a directed graph $G$ contains a path from u to v, then any depth-first search must result in $v.d \le u.f$.
\end{problem}

\begin{solution}
  Consider a graph with 3 vertices u, v, and w, and with edges (w, u), (u, w), and (w, v).\\
  $u.f = 3 < 4 = v.d$
\end{solution}
%%%%%%%%%%%%%%%

%%%%%%%%%%%%%%%
\begin{problem}[TC 22.3-12]
证明:我们可以再无向图G上使用深度优先搜索来获得图G的连通分量,并且深度优先森林所包含的树的棵数与F的连通分量数量相同。更准确地说,请给出如何修改深度优先搜索来让其给每个节点赋予一个介于$1$与$k$之间的整数值$v.cc$,这里$k$是G的连通分量数,使得$u.cc=v.cc$当且仅当节点$u$和节点$v$处于同一个连通分量中。
\end{problem}

\begin{solution}
  串页了(3-12)
  \begin{algorithm}
    \begin{algorithmic}
      \Procedure{DFS-CC}{$G$}
      \For {each vertex $u \in G.V$}
      \State $u.color = WHITE$
      \State $u.\pi =NIL$
      \EndFor
      \State $time = 0$
      \State cc = 1
      \For {each vertex $u \in G.V$}
      \If {$u.color == WHITE$}
      \State u.cc = cc
      \State cc = cc + 1
      \State DFS-VISIT-CC(G,u)
      \EndIf
      \EndFor
      \EndProcedure
      \Procedure{DFS-VISIT-CC}{$G,u$}
      \State time = time + 1
      \State u.d = tiime
      \State u.color = GRAY
      \For {each vertex $v \in G.Adj[u]$}
      \If {v.color ==WHITE}
      \State v.cc = u.CC
      \State $v.\pi = u$
      \State DFS-VISIT-CC(G,v)
      \EndIf
      \EndFor
      \State u.color = BLACK
      \State time = time + 1
      \State u.f = time
      \EndProcedure
    \end{algorithmic}
  \end{algorithm}
\end{solution}
%%%%%%%%%%%%%%%

%%%%%%%%%%%%%%%
\begin{problem}[TC 22.4-2]
\end{problem}

\begin{solution}
  The algorithm works as follows. The attribute u.pathsu.paths of node u tells the number of simple paths from u to v, where we assume that v is fixed throughout the entire process. First of all, a topo sort should be conducted and list the vertex between u, v as $\{v[1], v[2], \dots, v[k - 1]\}$. To count the number of paths, we should construct a solution from vv to uu. Let's call u as v[0] and v as v[k], to avoid overlapping subproblem, the number of paths between $v_k$ and u should be remembered and used as k decrease to 0. Only in this way can we solve the problem in $\Theta(V + E)$
  An bottom-up iterative version is possible only if the graph uses adjacency matrix so whether $v$ is adjacency to u can be determined in O(1) time. But building a adjacency matrix would cost $\Theta(|V|^2)$, so never mind.
  \begin{verbatim}
  SIMPLE-PATHS(G, u, v)
    TOPO-SORT(G)
    let {v[1], v[2]..v[k - 1]} be the vertex between u and v
    v[0] = u
    v[k] = v
    for j = 0 to k - 1
        DP[j] = ∞
    DP[k] = 1
    return SIMPLE-PATHS-AID(G, DP, 0)

    SIMPLE-PATHS-AID(G, DP, i)
    if i > k
        return 0
    else if DP[i] != ∞
        return DP[i]
    else
       DP[i] = 0
       for v[m] in G.adj[v[i]] and 0 < m ≤ k
            DP[i] += SIMPLE-PATHS-AID(G, DP, m)
       return DP[i]
      \end{verbatim}
\end{solution}
%%%%%%%%%%%%%%%

%%%%%%%%%%%%%%%
\begin{problem}[TC 22.4-3]
Give an algorithm that determines whether or not a given undirected graph $G=(V,E)$ contains a cycle. Your algorithm should run in O(V) time, independent of |E|.
\end{problem}

\begin{solution}
  \begin{algorithm}
    \caption{cycle}\label{euclid}
    \begin{algorithmic}[1]
      \Procedure{cycle}{G}
      \If{|G.E|$\geq$ |G.V|}
      \State \Return yes
      \EndIf
      \For {$v\in G.V$}
      \If {!visited[v]}
      \State p $\gets$ dfs(v)\Comment{返回是否有返祖边}
      \If {p == 1}
      \State \Return yes
      \EndIf
      \EndIf
      \EndFor
      \State \Return no
      \EndProcedure
    \end{algorithmic}
  \end{algorithm}
\end{solution}
%%%%%%%%%%%%%%%

%%%%%%%%%%%%%%%
\begin{problem}[TC 22.5-5]
Give an $O(V+E)$-time algorithm to compute the component graph of a directed graph $G=(V,E)$. Make sure that there is at most one edge between two vertices in the component graph your algorithm produces.
\end{problem}

\begin{solution}
  (在下面)

  \noindent
  \begin{algorithm}
    \caption{scc}\label{euclid}
    \begin{algorithmic}[1]
      \Procedure{scc}{G}
      \For {u $\in$ G.V}
      \For {(u,v) $\in$ G.E}
      \If{u.scc $\neq$ v.scc}
      \If {hashmap.find(u,v)}
      \State continue
      \EndIf
      \State addedge(u, v)
      \State hashmap.insert((u, v))
      \EndIf
      \EndFor
      \EndFor
      \EndProcedure
    \end{algorithmic}
  \end{algorithm}
\end{solution}
%%%%%%%%%%%%%%%


%%%%%%%%%%%%%%%%%%%%
\beginoptional

%%%%%%%%%%%%%%%
\begin{problem}[TC 22.5-7]
\end{problem}

\begin{solution}
\end{solution}
%%%%%%%%%%%%%%%


%%%%%%%%%%%%%%%%%%%%
\beginot
%%%%%%%%%%%%%%%
%%%%%%%%%%%%%%%
\begin{ot}[Tarjan's Algorithms]
  [参考资料:\href{https://ieeexplore.ieee.org/document/4569669}{R. Tarjan, "Depth-first search and linear graph algorithms," 12th Annual Symposium on Switching and Automata Theory (swat 1971), East Lansing, MI, USA, 1971, pp. 114-121, doi: 10.1109/SWAT.1971.10.}]

\end{ot}

% \begin{solution}
% \end{solution}
%%%%%%%%%%%%%%%

\begin{ot}[带边标记的DFS算法]
  写出带边标记的DFS算法并证明算法的正确性
\end{ot}

% \begin{solution}
% \end{solution}
%%%%%%%%%%%%%%%




% \vspace{0.50cm}
%%%%%%%%%%%%%%%
% \begin{ot}[]
% 
%   \noindent 参考资料:
%   \begin{itemize}
%     \item 
%   \end{itemize}
% \end{ot}

% \begin{solution}
% \end{solution}
%%%%%%%%%%%%%%%

%%%%%%%%%%%%%%%%%%%%
% 如果没有需要订正的题目,可以把这部分删掉

% \begincorrection
%%%%%%%%%%%%%%%%%%%%

%%%%%%%%%%%%%%%%%%%%
% 如果没有反馈,可以把这部分删掉
\beginfb

% 你可以写
% ~\footnote{优先推荐 \href{problemoverflow.top}{ProblemOverflow}}:
% \begin{itemize}
%   \item 对课程及教师的建议与意见
%   \item 教材中不理解的内容
%   \item 希望深入了解的内容
%   \item $\cdots$
% \end{itemize}
%%%%%%%%%%%%%%%%%%%%
% \bibliography{2-5-solving-recurrence}
% \bibliographystyle{plainnat}
%%%%%%%%%%%%%%%%%%%%
\end{document}