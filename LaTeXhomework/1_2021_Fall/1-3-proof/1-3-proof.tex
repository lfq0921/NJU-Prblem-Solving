% 1-3-proof.tex

%%%%%%%%%%%%%%%%%%%%
\documentclass[a4paper, justified]{tufte-handout}

% hw-preamble.tex

% geometry for A4 paper
% See https://tex.stackexchange.com/a/119912/23098
\geometry{
  left=20.0mm,
  top=20.0mm,
  bottom=20.0mm,
  textwidth=130mm, % main text block
  marginparsep=5.0mm, % gutter between main text block and margin notes
  marginparwidth=50.0mm % width of margin notes
}

% for colors
\usepackage{xcolor} % usage: \color{red}{text}
% predefined colors
\newcommand{\red}[1]{\textcolor{red}{#1}} % usage: \red{text}
\newcommand{\blue}[1]{\textcolor{blue}{#1}}
\newcommand{\teal}[1]{\textcolor{teal}{#1}}

\usepackage{todonotes}

% heading
\usepackage{sectsty}
\setcounter{secnumdepth}{2}
\allsectionsfont{\centering\huge\rmfamily}

% for Chinese
\usepackage{xeCJK}
\usepackage{zhnumber}
\setCJKmainfont[BoldFont=FandolSong-Bold.otf]{FandolSong-Regular.otf}

% for fonts
\usepackage{fontspec}
\newcommand{\song}{\CJKfamily{song}} 
\newcommand{\kai}{\CJKfamily{kai}} 

% To fix the ``MakeTextLowerCase'' bug:
% See https://github.com/Tufte-LaTeX/tufte-latex/issues/64#issuecomment-78572017
% Set up the spacing using fontspec features
\renewcommand\allcapsspacing[1]{{\addfontfeature{LetterSpace=15}#1}}
\renewcommand\smallcapsspacing[1]{{\addfontfeature{LetterSpace=10}#1}}

% for url
\usepackage{hyperref}
\hypersetup{colorlinks = true, 
  linkcolor = teal,
  urlcolor  = teal,
  citecolor = blue,
  anchorcolor = blue}

\newcommand{\me}[4]{
    \author{
      {\bfseries 姓名:}\underline{#1}\hspace{2em}
      {\bfseries 学号:}\underline{#2}\hspace{2em}\\[10pt]
      {\bfseries 评分:}\underline{#3\hspace{3em}}\hspace{2em}
      {\bfseries 评阅:}\underline{#4\hspace{3em}}
  }
}

% Please ALWAYS Keep This.
\newcommand{\noplagiarism}{
  \begin{center}
    \fbox{\begin{tabular}{@{}c@{}}
      请独立完成作业,不得抄袭。\\
      若得到他人帮助, 请致谢。\\
      若参考了其它资料,请给出引用。\\
      鼓励讨论,但需独立书写解题过程。
    \end{tabular}}
  \end{center}
}

\newcommand{\goal}[1]{
  \begin{center}{\fcolorbox{blue}{yellow!60}{\parbox{0.50\textwidth}{\large 
    \begin{itemize}
      \item 体会``思维的乐趣''
      \item 初步了解递归与数学归纳法 
      \item 初步接触算法概念与问题下界概念
    \end{itemize}}}}
  \end{center}
}

% Each hw consists of four parts:
\newcommand{\beginrequired}{\hspace{5em}\section{作业 (必做部分)}}
\newcommand{\beginoptional}{\section{作业 (选做部分)}}
\newcommand{\beginot}{\section{Open Topics}}
\newcommand{\begincorrection}{\section{订正}}
\newcommand{\beginfb}{\section{反馈}}

% for math
\usepackage{amsmath, mathtools, amsfonts, amssymb}
\newcommand{\set}[1]{\{#1\}}

% define theorem-like environments
\usepackage[amsmath, thmmarks]{ntheorem}

\theoremstyle{break}
\theorempreskip{2.0\topsep}
\theorembodyfont{\song}
\theoremseparator{}
\newtheorem{problem}{题目}[subsection]
\renewcommand{\theproblem}{\arabic{problem}}
\newtheorem{ot}{Open Topics}

\theorempreskip{3.0\topsep}
\theoremheaderfont{\kai\bfseries}
\theoremseparator{:}
\theorempostwork{\bigskip\hrule}
\newtheorem*{solution}{解答}
\theorempostwork{\bigskip\hrule}
\newtheorem*{revision}{订正}

\theoremstyle{plain}
\newtheorem*{cause}{错因分析}
\newtheorem*{remark}{注}

\theoremstyle{break}
\theorempostwork{\bigskip\hrule}
\theoremsymbol{\ensuremath{\Box}}
\newtheorem*{proof}{证明}

% \newcommand{\ot}{\blue{\bf [OT]}}

% for figs
\renewcommand\figurename{图}
\renewcommand\tablename{表}

% for fig without caption: #1: width/size; #2: fig file
\newcommand{\fig}[2]{
  \begin{figure}[htbp]
    \centering
    \includegraphics[#1]{#2}
  \end{figure}
}
% for fig with caption: #1: width/size; #2: fig file; #3: caption
\newcommand{\figcap}[3]{
  \begin{figure}[htbp]
    \centering
    \includegraphics[#1]{#2}
    \caption{#3}
  \end{figure}
}
% for fig with both caption and label: #1: width/size; #2: fig file; #3: caption; #4: label
\newcommand{\figcaplbl}[4]{
  \begin{figure}[htbp]
    \centering
    \includegraphics[#1]{#2}
    \caption{#3}
    \label{#4}
  \end{figure}
}
% for margin fig without caption: #1: width/size; #2: fig file
\newcommand{\mfig}[2]{
  \begin{marginfigure}
    \centering
    \includegraphics[#1]{#2}
  \end{marginfigure}
}
% for margin fig with caption: #1: width/size; #2: fig file; #3: caption
\newcommand{\mfigcap}[3]{
  \begin{marginfigure}
    \centering
    \includegraphics[#1]{#2}
    \caption{#3}
  \end{marginfigure}
}

\usepackage{fancyvrb}

% for algorithms
\usepackage[]{algorithm}
\usepackage[]{algpseudocode} % noend
% See [Adjust the indentation whithin the algorithmicx-package when a line is broken](https://tex.stackexchange.com/a/68540/23098)
\newcommand{\algparbox}[1]{\parbox[t]{\dimexpr\linewidth-\algorithmicindent}{#1\strut}}
\newcommand{\hStatex}[0]{\vspace{5pt}}
\makeatletter
\newlength{\trianglerightwidth}
\settowidth{\trianglerightwidth}{$\triangleright$~}
\algnewcommand{\LineComment}[1]{\Statex \hskip\ALG@thistlm \(\triangleright\) #1}
\algnewcommand{\LineCommentCont}[1]{\Statex \hskip\ALG@thistlm%
  \parbox[t]{\dimexpr\linewidth-\ALG@thistlm}{\hangindent=\trianglerightwidth \hangafter=1 \strut$\triangleright$ #1\strut}}
\makeatother

% for footnote/marginnote
% see https://tex.stackexchange.com/a/133265/23098
\usepackage{tikz}
\newcommand{\circled}[1]{%
  \tikz[baseline=(char.base)]
  \node [draw, circle, inner sep = 0.5pt, font = \tiny, minimum size = 8pt] (char) {#1};
}
\renewcommand\thefootnote{\protect\circled{\arabic{footnote}}} % feel free to modify this file
%%%%%%%%%%%%%%%%%%%%
\title{第3讲: 常用的证明方法}
\me{林凡琪}{211240042}{}{}
\date{\zhtoday} % or like 2019年9月13日
%%%%%%%%%%%%%%%%%%%%
\begin{document}
\maketitle
%%%%%%%%%%%%%%%%%%%%
\noplagiarism % always keep this line
%%%%%%%%%%%%%%%%%%%%
\begin{abstract}
  \mfigcap{width = 1.00\textwidth}{figs/dominoeffect}{数学归纳法的``多米诺骨牌效应''}
  \begin{center}{\fcolorbox{blue}{yellow!60}{\parbox{0.60\textwidth}{\large 
    \begin{itemize}
      \item 反证法是你最好的朋友
      \item 数学归纳法是你最最好的朋友
      \item 鸽笼原理, 哦, 有点高冷, 这个朋友不好交
            {(看似具体, 实则抽象; 看似容易, 实则困难)}
    \end{itemize}}}}
  \end{center}
\end{abstract}
%%%%%%%%%%%%%%%%%%%%
\beginrequired

%%%%%%%%%%%%%%%
\begin{problem}[UD Problem $5.12$: $3k + 2$]
\end{problem}

\begin{remark}
本题参考了xx资料(网页链接)。本题与xx同学讨论。
\end{remark}

\begin{solution}
\quad \quad $i)$ $x=0(mod 3)$,$x^2=0(mod 3)$;

$ii) x=1(mod 3),x^2=1(mod 3)$

$iii) x=2(mod)3,x^2=1(mod 3)$

所以不存在整数x,使得$x^2=3(mod 2)$.

\end{solution}
%%%%%%%%%%%%%%%

%%%%%%%%%%%%%%%
\begin{problem}[UD Problem $5.24$: Squaring]
\end{problem}

\begin{solution}
\quad \quad (a)Forall non-negative integers,exists 2 reasonable numbers y and z that are not zero,such that $x^2=y^2+z^2$

(b) 令y=(3/5)*x,z=(4/5)*x,此时必有$x^2=y^2+z^2$

\end{solution}
%%%%%%%%%%%%%%%

%%%%%%%%%%%%%%%
\begin{problem}[Primes 3 (Mod 4) Theorem]
  请证明: There are infinitely many primes 
  that are congruent to 3 modulo 4.
\end{problem}

\begin{solution}
\quad  反证法:

假设共有n个素数形如4k+3(k为整数),按升序排列为p1p2...pn

设q=p1p2...p3+2,显然为奇数,所以q只能为4k+1或4k+3,且p1到pn都不是q的因数.

(1)若q=4k+3,则显然假设错误

(2)若q=4k+1,则q'=q+2=4k+3,假设错误

所以证得有无穷个形如4k+3的素数

本题参考:zhihu(网址打不出来...)

\end{solution}
%%%%%%%%%%%%%%%

%%%%%%%%%%%%%%%
\begin{problem}[改编自 UD Problem $18.20$ 与 UD Problem $18.26$]
  请证明: 
  \begin{enumerate}[(1)]
    \item ``The first principle of mathematical induction'' (Theorem $18.1$)
      与 ``The second principle of mathematical induction'' (Theorem $18.9$) 等价。
    \item ``The second principle of mathematical induction'' 蕴含 
      ``Well-ordering principles of the natural numbers'' (in Chapter 12)。
  \end{enumerate}
\end{problem}

\begin{solution}
\quad 由Theorem $18.1$ $\rightarrow$ Theorem $18.9$:已知P(1)为真,由第一数学归纳法,P(1)可推出P(2)再推出P(3),递推至P(n)时即已知P(1),...,P(n)为真,此时若用第一数学归纳法,则可由P(1)为真和P(n)为真两个条件推出P(n+1)为真,而若用第二数学归纳法,可由P(1),...,P(n)为真推出P(n+1)为真.

由Theorem $18.9$ $\rightarrow$ Theorem $18.1$:由第二数学归纳法,前提为Q(1),...,Q(n)都为真,推出Q(n+1)为真,同样前提下可由Q(1)和Q(n)为真,由第一数学归纳法推出Q(n+1)为真.

\end{solution}
%%%%%%%%%%%%%%%

%%%%%%%%%%%%%%%
\begin{problem}[Lines in the Plane]
  \begin{enumerate}[(1)]
    \item What is the maximum number $L_n$ of regions 
      determined by $n$ straight lines in the plane?
      \mfigcap{width = 1.00\textwidth}{figs/straight-line-ln}{Examples for $L_0$, $L_1$, and $L_2$.}
      (注: 直线两端可以无限延长)
    \item What is the maximum number $Z_n$ of regions 
      determined by $n$ bent lines, each containing one ``zig'', 
      in the plane?
      \mfigcap{width = 1.00\textwidth}{figs/bent-line-zn}{Examples for $Z_1$ and $Z_2$.}
      (注: 两端可以无限延长)
    \item What's the maximum number $ZZ_n$ of regions
      determined by $n$ ``zig-zag'' lines in the plane?
      \mfigcap{width = 1.00\textwidth}{figs/zigzag-zzn}{Example for $ZZ_2$.}
      (注: 两端可以无限延长)
  \end{enumerate}
\end{problem}

\begin{solution}

\quad 1+n(n+1)/2

对于每一组折线,需要区域数最多,即需要与之前的线都相交,一条折线相当于第一题的两条线,每个拐角少两个区域,所以区域数为$2n^2-n+1$

同理,对于z形线,一条折线相当于题一的三条线,每个拐角少两个区域,每有一组平行线,少交一个区域,所以区域数为$(9n^2-7n+2)/2$。

致谢:杨镇源同学
\end{solution}
%%%%%%%%%%%%%%%

%%%%%%%%%%%%%%%
\begin{problem}[ES Problem $24.4$: Distance in Square]
\end{problem}

\begin{solution}

\quad 将边长为一的大正方形分为四个边长为0.5的小正方形,根据鸽笼原理,一定有两个点在同一个小正方形里,此时这两个点的距离最大为$\sqrt{2}/2$


\end{solution}
%%%%%%%%%%%%%%%

%%%%%%%%%%%%%%%
\begin{problem}[ES Problem $24.6$: Lattice Points]
\end{problem}

\begin{solution}

\quad 有9个给定三维空间中的不同格点,他们的连线中必有一条线的中点坐标都为整数。

证明:三维空间中的格点坐标奇偶性共有8种,根据鸽笼原理,9个点中必有2个点的奇偶性完全一致,设分别为(a,b,c)和(d,e,f),则a+d,b+e,c+f必为偶数,根据中点坐标公式,这两个点的之间的中点的点坐标((a+d)/2,(b+e)/2,(c+f)/2)三个坐标值必为整数.
\end{solution}
%%%%%%%%%%%%%%%

%%%%%%%%%%%%%%%
\begin{problem}[ES Problem $24.7$: Monotone Subsequence]
\end{problem}

\begin{solution}

\quad 789456123
\end{solution}
%%%%%%%%%%%%%%%

%%%%%%%%%%%%%%%%%%%%
\beginoptional

%%%%%%%%%%%%%%%
\begin{problem}[Numbers]
  Suppose $A \subseteq \set{1, 2, \cdots, 2n}$ with $|A| = n + 1$.
  Please prove that:
  \mfig{width = 0.70\textwidth}{figs/pigeon-hole-principle}
  \begin{enumerate}[(1)]
    \item There are two numbers in $A$ which are relatively prime (互素).
    \item There are two numbers in $A$ such that one divides (整除) the other.
  \end{enumerate}
\end{problem}

\begin{solution}
\end{solution}
%%%%%%%%%%%%%%%

%%%%%%%%%%%%%%%%%%%%
\beginot

%%%%%%%%%%%%%%%
\begin{ot}[Coq]
  请介绍如何在 Coq 中使用数学归纳法。

  \noindent 参考资料:
  \begin{itemize}
    \item \href{https://github.com/hengxin/problem-solving-class-coq/blob/master/2019-1-coq/Induction.v}{Induction.v}
  \end{itemize}
\end{ot}

\begin{solution}
\end{solution}
%%%%%%%%%%%%%%%

%%%%%%%%%%%%%%%
\begin{ot}[Double Counting]
  ``Double Counting'' 是一种神奇、漂亮的组合证明技巧。
  请了解 Double Counting 并以 ``Counting Trees'' 为例介绍这种证明技巧。

  \noindent 参考资料:
  \mfigcap{width = 0.95\textwidth}{figs/good-will-hunting-counting-trees}{电影《心灵捕手》截图}
  \begin{itemize}
    \item 电影 ``Good Will Hunting'' (心灵捕手)
    \item Chapter 30 ``Cayley's formula for the number of trees''
      of ``Proofs from THE BOOK'' (Fourth Edition)
    \item \href{https://en.wikipedia.org/wiki/Double\_counting\_(proof\_technique)#Counting\_trees}{Counting trees @ wiki}
  \end{itemize}
\end{ot}

\begin{solution}
\end{solution}
%%%%%%%%%%%%%%%

%%%%%%%%%%%%%%%%%%%%
% 如果没有需要订正的题目,可以把这部分删掉
\begincorrection

\begin{problem-non}[1-1-1]
\end{problem-non}

\begin{cause}
\end{cause}

\begin{revision}
\end{revision}
%%%%%%%%%%%%%%%%%%%%
% 如果没有反馈,可以把这部分删掉
\beginfb

%你可以写
%~\footnote{优先推荐 \href{problemoverflow.top}{ProblemOverflow}}:
%\begin{itemize}
%  \item 对课程及教师的建议与意见
%  \item 教材中不理解的内容
%  \item 希望深入了解的内容
%  \item $\cdots$
%\end{itemize}
%%%%%%%%%%%%%%%%%%%%
\end{document}