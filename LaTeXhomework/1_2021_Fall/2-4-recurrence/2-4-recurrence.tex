% 2-4-recurrence.tex

%%%%%%%%%%%%%%%%%%%%
\documentclass[a4paper, justified]{tufte-handout}

% hw-preamble.tex

% geometry for A4 paper
% See https://tex.stackexchange.com/a/119912/23098
\geometry{
  left=20.0mm,
  top=20.0mm,
  bottom=20.0mm,
  textwidth=130mm, % main text block
  marginparsep=5.0mm, % gutter between main text block and margin notes
  marginparwidth=50.0mm % width of margin notes
}

% for colors
\usepackage{xcolor} % usage: \color{red}{text}
% predefined colors
\newcommand{\red}[1]{\textcolor{red}{#1}} % usage: \red{text}
\newcommand{\blue}[1]{\textcolor{blue}{#1}}
\newcommand{\teal}[1]{\textcolor{teal}{#1}}

\usepackage{todonotes}

% heading
\usepackage{sectsty}
\setcounter{secnumdepth}{2}
\allsectionsfont{\centering\huge\rmfamily}

% for Chinese
\usepackage{xeCJK}
\usepackage{zhnumber}
\setCJKmainfont[BoldFont=FandolSong-Bold.otf]{FandolSong-Regular.otf}

% for fonts
\usepackage{fontspec}
\newcommand{\song}{\CJKfamily{song}} 
\newcommand{\kai}{\CJKfamily{kai}} 

% To fix the ``MakeTextLowerCase'' bug:
% See https://github.com/Tufte-LaTeX/tufte-latex/issues/64#issuecomment-78572017
% Set up the spacing using fontspec features
\renewcommand\allcapsspacing[1]{{\addfontfeature{LetterSpace=15}#1}}
\renewcommand\smallcapsspacing[1]{{\addfontfeature{LetterSpace=10}#1}}

% for url
\usepackage{hyperref}
\hypersetup{colorlinks = true, 
  linkcolor = teal,
  urlcolor  = teal,
  citecolor = blue,
  anchorcolor = blue}

\newcommand{\me}[4]{
    \author{
      {\bfseries 姓名:}\underline{#1}\hspace{2em}
      {\bfseries 学号:}\underline{#2}\hspace{2em}\\[10pt]
      {\bfseries 评分:}\underline{#3\hspace{3em}}\hspace{2em}
      {\bfseries 评阅:}\underline{#4\hspace{3em}}
  }
}

% Please ALWAYS Keep This.
\newcommand{\noplagiarism}{
  \begin{center}
    \fbox{\begin{tabular}{@{}c@{}}
      请独立完成作业,不得抄袭。\\
      若得到他人帮助, 请致谢。\\
      若参考了其它资料,请给出引用。\\
      鼓励讨论,但需独立书写解题过程。
    \end{tabular}}
  \end{center}
}

\newcommand{\goal}[1]{
  \begin{center}{\fcolorbox{blue}{yellow!60}{\parbox{0.50\textwidth}{\large 
    \begin{itemize}
      \item 体会``思维的乐趣''
      \item 初步了解递归与数学归纳法 
      \item 初步接触算法概念与问题下界概念
    \end{itemize}}}}
  \end{center}
}

% Each hw consists of four parts:
\newcommand{\beginrequired}{\hspace{5em}\section{作业 (必做部分)}}
\newcommand{\beginoptional}{\section{作业 (选做部分)}}
\newcommand{\beginot}{\section{Open Topics}}
\newcommand{\begincorrection}{\section{订正}}
\newcommand{\beginfb}{\section{反馈}}

% for math
\usepackage{amsmath, mathtools, amsfonts, amssymb}
\newcommand{\set}[1]{\{#1\}}

% define theorem-like environments
\usepackage[amsmath, thmmarks]{ntheorem}

\theoremstyle{break}
\theorempreskip{2.0\topsep}
\theorembodyfont{\song}
\theoremseparator{}
\newtheorem{problem}{题目}[subsection]
\renewcommand{\theproblem}{\arabic{problem}}
\newtheorem{ot}{Open Topics}

\theorempreskip{3.0\topsep}
\theoremheaderfont{\kai\bfseries}
\theoremseparator{:}
\theorempostwork{\bigskip\hrule}
\newtheorem*{solution}{解答}
\theorempostwork{\bigskip\hrule}
\newtheorem*{revision}{订正}

\theoremstyle{plain}
\newtheorem*{cause}{错因分析}
\newtheorem*{remark}{注}

\theoremstyle{break}
\theorempostwork{\bigskip\hrule}
\theoremsymbol{\ensuremath{\Box}}
\newtheorem*{proof}{证明}

% \newcommand{\ot}{\blue{\bf [OT]}}

% for figs
\renewcommand\figurename{图}
\renewcommand\tablename{表}

% for fig without caption: #1: width/size; #2: fig file
\newcommand{\fig}[2]{
  \begin{figure}[htbp]
    \centering
    \includegraphics[#1]{#2}
  \end{figure}
}
% for fig with caption: #1: width/size; #2: fig file; #3: caption
\newcommand{\figcap}[3]{
  \begin{figure}[htbp]
    \centering
    \includegraphics[#1]{#2}
    \caption{#3}
  \end{figure}
}
% for fig with both caption and label: #1: width/size; #2: fig file; #3: caption; #4: label
\newcommand{\figcaplbl}[4]{
  \begin{figure}[htbp]
    \centering
    \includegraphics[#1]{#2}
    \caption{#3}
    \label{#4}
  \end{figure}
}
% for margin fig without caption: #1: width/size; #2: fig file
\newcommand{\mfig}[2]{
  \begin{marginfigure}
    \centering
    \includegraphics[#1]{#2}
  \end{marginfigure}
}
% for margin fig with caption: #1: width/size; #2: fig file; #3: caption
\newcommand{\mfigcap}[3]{
  \begin{marginfigure}
    \centering
    \includegraphics[#1]{#2}
    \caption{#3}
  \end{marginfigure}
}

\usepackage{fancyvrb}

% for algorithms
\usepackage[]{algorithm}
\usepackage[]{algpseudocode} % noend
% See [Adjust the indentation whithin the algorithmicx-package when a line is broken](https://tex.stackexchange.com/a/68540/23098)
\newcommand{\algparbox}[1]{\parbox[t]{\dimexpr\linewidth-\algorithmicindent}{#1\strut}}
\newcommand{\hStatex}[0]{\vspace{5pt}}
\makeatletter
\newlength{\trianglerightwidth}
\settowidth{\trianglerightwidth}{$\triangleright$~}
\algnewcommand{\LineComment}[1]{\Statex \hskip\ALG@thistlm \(\triangleright\) #1}
\algnewcommand{\LineCommentCont}[1]{\Statex \hskip\ALG@thistlm%
  \parbox[t]{\dimexpr\linewidth-\ALG@thistlm}{\hangindent=\trianglerightwidth \hangafter=1 \strut$\triangleright$ #1\strut}}
\makeatother

% for footnote/marginnote
% see https://tex.stackexchange.com/a/133265/23098
\usepackage{tikz}
\newcommand{\circled}[1]{%
  \tikz[baseline=(char.base)]
  \node [draw, circle, inner sep = 0.5pt, font = \tiny, minimum size = 8pt] (char) {#1};
}
\renewcommand\thefootnote{\protect\circled{\arabic{footnote}}} % feel free to modify this file
%%%%%%%%%%%%%%%%%%%%
\title{第4讲: 分治法与递归}
\me{朱宇博}{191220186}{}{}
\date{\zhtoday} % or like 2019年9月13日
%%%%%%%%%%%%%%%%%%%%
\begin{document}
\maketitle
%%%%%%%%%%%%%%%%%%%%
\noplagiarism % always keep this line
%%%%%%%%%%%%%%%%%%%%
\begin{abstract}
  % \begin{center}{\fcolorbox{blue}{yellow!60}{\parbox{0.65\textwidth}{\large 
  %   \begin{itemize}
  %     \item 
  %   \end{itemize}}}}
  % \end{center}
\end{abstract}
%%%%%%%%%%%%%%%%%%%%
\beginrequired
We assume that the root of the recursion-tree is in the $0^{th}$ depth.
%%%%%%%%%%%%%%%
\begin{problem}[TC 4.1-5]
Use the following ideas to develop a nonrecursive, linear-time algorithm for the maximum-subarray problem. Start at the left end of the array, and progress toward the right, keeping track of the maximum subarray seen so far. Knowing a maximum subarray of $A[1 \ldots j],$ extend the answer to find a maximum subarray ending at index $j+1$ by using the following observation: a maximum subarray of $A[1 \ldots j+1]$
is either a maximum subarray of $A[1 \ldots j]$ or a subarray $A[i \ldots j+1],$ for some $1 \leq i \leq j+1 .$ Determine a maximum subarray of the form $A[i \ldots j+1]$ in constant time based on knowing a maximum subarray ending at index $j$.
\end{problem}

\begin{solution}
\noindent
\begin{algorithm}
\caption{ maximum-subarray}\label{euclid}
\begin{algorithmic}[1]
\Procedure{find}{A[],n}\Comment{an array with $n$ elements}
\State sum[$0$]$\gets $ 0 
\For{$i\gets 1$ to $n$}
	\State sum[$i$]$\gets$ sum[$i-1$]+A[$i$]
\EndFor
\State minn=$0$
\State ans=$-\infty$
\For{$i\gets 1$ to $n$}
\State ans$\gets$max(ans,sum[$i$]-minn)
\State minn$\gets$min(minn,sum[$i$])
\EndFor 
\State \Return ans 
\EndProcedure
\end{algorithmic}
\end{algorithm}

\end{solution}
%%%%%%%%%%%%%%%

%%%%%%%%%%%%%%%
\begin{problem}[TC 4.3-3]
We saw that the solution of $T(n)=2 T(\lfloor \frac{n}{2}\rfloor)+n$ is $O(n \lg n) .$ Show that the solution of this recurrence is also $\Omega(n \lg n) .$ Conclude that the solution is $\Theta(n \lg n)$
\end{problem}

\begin{solution}
Prove: $T(n)\geq cn\lg n$ holds for $n\geq 1$,  which $c>0$ is a constant.\\
For $n=1$, $T(1)=1\geq c1\lg 1=0$.\\
 For $n>1$,assume that $\forall m,(m\in\mathbb{N}^{+}\land m<n\to T(m)\geq cn\lg n)$\\
 $T(n)=2 T(\lfloor \frac{n}{2}\rfloor)+n$\\
 \hspace*{2.5em}$\geq 2((c \lfloor \frac{n}{2} \rfloor \lg(\lfloor \frac{n}{2}\rfloor))+n$\\
\hspace*{2.5em}$\geq c(n-1)\lg(n-1)-cnlg2+n$\\
\hspace*{2.5em}$\geq c(n-1)\lg(n-1)-cn+n$\\
\hspace*{2.5em}$= c(n-1)\lg(n\frac{n-1}{n})-cn+n$\\
\hspace*{2.5em}$\geq cn\lg n -c\lg n+cn\lg (\frac{n-1}{n})-cn+n$\\
\hspace*{2.5em}$\geq cn\lg n -cn-cn-cn+n$\\
\hspace*{2.5em}$\geq cn\lg n$\\
The constant $c$ satisfies $0<c\leq\frac{1}{3}$.\\
Therefore, $T(n)=\Omega(n \lg n)$.\\
Since $T(n)=\Omega(n \lg n)$ and $T(n)=O(n\lg n)$, $T(n)=\Theta(n \lg n)$.
\end{solution}
%%%%%%%%%%%%%%%

%%%%%%%%%%%%%%%
\begin{problem}[TC 4.4-5]
Use a recursion tree to determine a good asymptotic upper bound on the recurrence $T(n)=T(n-1)+T(\frac{n}{2})+n$. Use the substitution method to verify your answer.
\end{problem}

\begin{solution}
On the  recursion-tree, the maximum depth is  $n$, the minimal depth is $log_2{n}+1$, and the total cost is  $n(\frac{3}{2})^i-t_i(t_i\geq 0)$  in the $i^{th}$ depth which satisfies $i\leq log_2{n}$.\\ 
We ignore $t_i$, and let the depth be the maximum depth. We have\\
$T(n)=O(\sum\limits_{i=0}^{n-1}n(\frac{3}{2})^i)=O(n(\frac{3}{2})^n)$\\
Therefore, $T(n)=O(n(\frac{3}{2})^n)$.
\end{solution}
%%%%%%%%%%%%%%%

%%%%%%%%%%%%%%%
\begin{problem}[TC 4.5-4]
Can the master method be applied to the recurrence $T(n)=4 T(n / 2)+n^{2} \lg n ?$ Why or why not? Give an asymptotic upper bound for this recurrence.
\end{problem}

\begin{solution}
The master method can not be used to solve that, since $\frac{n^2\lg n}{n^{\log_2{4}}}=\lg n$ which is not not polynomially larger.\\
On the  recursion-tree, The depth is  $log_2{n}+1$, and the total cost is  $n^2log_2n-n^2i$ nodes in the $i^{th}$ depth.\\  
$\sum\limits_{i=0}^{\log_2n}n^2log_2n-n^2i=O(\sum\limits_{i=0}^{\log_2n}n^2log_2n)$\\
Therefore, $T(n)=O(n^2log^2n)$.
\end{solution}
%%%%%%%%%%%%%%%

%%%%%%%%%%%%%%%
\begin{problem}[TC Problem 4-1]
Give asymptotic upper and lower bounds for $T(n)$ in each of the following recurrences. Assume that $T(n)$ is constant for $n \leq 2 .$ Make your bounds as tight as possible, and justify your answers.\\
a. $T(n)=2 T(n / 2)+n^{4}$\\
b. $T(n)=T(7 n / 10)+n$\\
c. $T(n)=16 T(n / 4)+n^{2}$\\
d. $T(n)=7 T(n / 3)+n^{2}$\\
e. $T(n)=7 T(n / 2)+n^{2}$\\
f.  $T(n)=2 T(n / 4)+\sqrt{n}$\\
g. $T(n)=T(n-2)+n^{2}$

\end{problem}

\begin{solution}
(a)\\
$\frac{n^4}{n^{\log_22}}=n^3$, $af(\frac{n}{b})=2(\frac{n}{2})^4=\frac{1}{8}n^4\leq cf(n)$. $c$ is a constant which satisfies $\frac{1}{8}\leq c<1$.\\
The maser method can be used in T(n).\\ 
$T(n)=\Theta(n^4)$\\

\noindent(b)\\
$\frac{n}{n^{\log_{\frac{10}{7}} 1}}=n$, $af(\frac{n}{b})=\frac{7}{10}n\leq cf(n)$. $c$ is a constant which satisfies $\frac{10}{7}\leq c<1$.\\
The maser method can be used in T(n).\\
$T(n)=\Theta(n)$\\

\noindent(c)\\
$\frac{n^2}{n^{\log_{4}16}}=1\to n^2=\Theta(n^{\log_4{16}})$.\\
The maser method can be used in T(n).\\
$T(n)=\Theta(n^2\lg n)$\\

\noindent(d)\\
$\frac{n^3}{n^{\log_{3}7}}=n^{3-log_37}$,$\quad af(\frac{n}{b})=7(\frac{n}{3})^2=\frac{7}{9}n^2\leq cf(n)$. $c$ is a constant which satisfies $\frac{7}{9}\leq c<1$.\\
$-\epsilon$
$T(n)=\Theta(n^2)$\\

\noindent(e)\\
$\frac{n^2}{n^{\log_{2}7}}=n^{2-\log_27}$\\
The maser method can be used in T(n).\\
$T(n)=\Theta(n^{log_27})$\\

\noindent(f)\\
$\frac{n^{\frac{1}{2}}}{n^{\log_42}}=1\to \sqrt{n}=\Theta{n^{log_42}}$\\
The maser method can be used in T(n).\\
$T(n)=\Theta(\sqrt{n}\lg n)$\\

\noindent(g)\\
Assume that $n$ is even, which doesn't affect the result.\\
$T(n)=\sum\limits_{i=1}^{\frac{n}{2}}(2i)^2=\Theta(n^3)$.\\
So $T(n)=\Theta(n^3)$
\end{solution}
%%%%%%%%%%%%%%%

%%%%%%%%%%%%%%%
\begin{problem}[TC Problem 4-3 (Except $f$ and $j$)]
4-3 More recurrence examples Give asymptotic upper and lower bounds for $T(n)$ in each of the following recurrences. Assume that $T(n)$ is constant for sufficiently small $n .$ Make your bounds as tight as possible, and justify your answers.\\
a. $T(n)=4 T(n / 3)+n \lg n$\\
b. $T(n)=3 T(n / 3)+n / \lg n$\\
c. $T(n)=4 T(n / 2)+n^{2} \sqrt{n}$\\
d. $T(n)=3 T(n / 3-2)+n / 2$\\
e. $T(n)=2 T(n / 2)+n / \lg n$\\
g. $T(n)=T(n-1)+1 / n$\\
h. $T(n)=T(n-1)+\lg n$\\
i. $T(n)=T(n-2)+1 / \lg n$
\end{problem}

\begin{solution}
\noindent(a)\\
$n\lg n=O(n^{\log_34-\epsilon})$. $\epsilon$ is a constant which satisfies $0<\epsilon<\log_34-1$.\\
The maser method can be used in T(n).\\
$T(n)=\Theta(n^{\log_34})$\\

\noindent(b)\\
On the  recursion-tree, The depth is  $log_3{n}+1$, and the total cost is  $\frac{n}{\lg n-i\lg 3}$ nodes in the $i^{th}$ depth.\\  
$\sum\limits_{i=0}^{\log_3n} \frac{n}{\lg n-i\lg 3}=\Omega(\sum\limits_{i=0}^{\log_3n} \frac{n}{\lg n})=\Omega(n)$\\
So $T(n)=\Omega(n)$\\
Prove: $T(n)\leq cn\lg n$.\\
$T(n)=3T(\frac{n}{3})+\frac{n}{\lg n}$\\
\hspace*{2.5em}$=cn\lg n-n\lg 3+\frac{n}{\lg n}$\\
\hspace*{2.5em}$\leq cn\lg n$\\
We choose $c=10$ and $\forall n_0\in\{2,3,4,5\},(T(n_0)\leq cn_0\lg n_0)$.\\
So $T(n)=O(n\lg n)$.\\

\noindent(c)\\
$\frac{n^2\sqrt{n}}{n^{log_24}}=n^{\frac{1}{2}}$, $af(\frac{n}{b})=4(\frac{n}{2})^2\sqrt{\frac{n}{2}}=\frac{\sqrt{2}}{2}n^2\sqrt{n}\leq cf(n)$, which $c$ is a constant which satisfies $\frac{\sqrt{2}}{2}\leq c<1$.\\
The maser method can be used in T(n).\\
$T(n)=\Theta(n^2\sqrt{n})$\\

\noindent(d)\\
Prove: $T(n)\geq cn\lg n$\\
$T(n)=3 T(n / 3-2)+\frac{1}{2}n$\\
\hspace*{2.5em}$=3c(n/3-2)\lg(n\frac{n-6}{3n})+\frac{1}{2}n$\\
\hspace*{2.5em}$=cn\lg n+ cn \lg\frac{n-6}{3n}-6c\lg n-6c\lg\frac{n-6}{3n}+\frac{1}{2}n$\\
\hspace*{2.5em}$\geq cn\lg n-2cn-6c\lg n+12c+\frac{1}{2}n$\\
\hspace*{2.5em}$\geq cn\lg n$\\
It will fit when $n>24$ and let $c=\frac{1}{26}$.\\
So $T(n)=\Omega(n\lg n)$.\\
For $T_1(n)=3f(n/3)+\frac{1}{2}n$, we have that $T(n)=O(T_1(n))$.\\
With the maser method, $T_1(n)=\Theta(n\lg n)$, so $T(n)=O(n\lg n)$.\\
Since $T(n)=\Omega(n\lg n)$ and $T(n)=O(n\lg n)$, $T(n)=\Theta(n\lg n)$.\\

\noindent(e)\\
On the  recursion-tree, The depth is  $log_2{n}+1$, and the total cost is  $\frac{n}{\lg n-i}$ nodes in the $i^{th}$ depth.\\  
$\sum\limits_{i=0}^{\log_2n} \frac{n}{\lg n-i}=\Omega(\sum\limits_{i=0}^{\log_2n} \frac{n}{\lg n})=\Omega(n)$\\
So $T(n)=\Omega(n)$\\
Prove: $T(n)\leq cn\lg n$.\\
$T(n)=2T(\frac{n}{2})+\frac{n}{\lg n}$\\
\hspace*{2.5em}$=cn\lg n-cn+\frac{n}{\lg n}$\\
\hspace*{2.5em}$\leq cn\lg n$\\
We choose $c=10$ and $\forall n_0\in\{2,3\},(T(n_0)\leq cn_0\lg n_0)$.\\
So $T(n)=O(n\lg n)$.\\

\noindent(g)\\
$T(n)=\Theta(\sum\limits_{i=1}^{n}\frac{1}{n})=\Theta(H_n)$.\\
So $T(n)=\Theta(\lg n)$.\\

\noindent(h)\\
 $T(n)=\Theta(\sum\limits_{i=1}^{n}\lg i)$\\
 $\Theta(\sum\limits_{i=1}^{n}\lg i)=\Theta(\lg n!)=\Theta(n\lg n)$\\
 $T(n)=\Theta(n\lg n)$\\
 
 \noindent(i)\\
  $T(n)=\Theta(\sum\limits_{i=1}^{\frac{n}{2}}\frac{1}{\lg (2i)})$\\
$\sum\limits_{i=1}^{\frac{n}{2}}(\frac{1}{\lg (2i)})=\Omega(\sum\limits_{i=1}^{\frac{n}{2}}\frac{1}{\lg n})=\Omega(\frac{n}{\lg n})$\\
 So $T(n)=\Omega(\frac{n}{\lg n})$\\
$\sum\limits_{i=1}^{\frac{n}{2}}(\frac{1}{\lg (2i)})=O(\sum\limits_{i=1}^{\frac{n}{2}}1)=O(n)$\\	
So $T(n)=O(n)$
\end{solution}
%%%%%%%%%%%%%%%

%%%%%%%%%%%%%%%%%%%%
\beginoptional

%%%%%%%%%%%%%%%
\begin{problem}[TC Problem 4-3 ($f$ and $j$)]
4-3 More recurrence examples Give asymptotic upper and lower bounds for $T(n)$ in each of the following recurrences. Assume that $T(n)$ is constant for sufficiently small $n .$ Make your bounds as tight as possible, and justify your answers.\\
f. $T(n)=T(n / 2)+T(n / 4)+T(n / 8)+n$\\
j. $T(n)=\sqrt{n} T(\sqrt{n})+n$
\end{problem}

\begin{solution}
(f)\\
On the  recursion-tree, the maxinum depth is  $log_2{n}+1$, the minimal depth is $log_8{n}+1$, and the total cost is  $n(\frac{7}{8})^i$  in the $i^{th}$ depth which satisfies $i\leq log_8{n}$.\\  
Obviously, $T(n)=\Omega(\sum\limits_{i=0}^{log_8n}n(\frac{7}{8})^i)=\Omega(n)$, $T(n)=O(\sum\limits_{i=0}^{log_2n}n(\frac{7}{8})^i)=O(n)$.\\
Therefore, $T(n)=\Theta(n)$.\\
(j)\\
On the  recursion-tree, the total cost is  $n$ in the $i^{th}$ depth.\\  
Since $\lim\limits_{n\to +\infty}n^{\frac{1}{n}}=1$, $n^{\frac{1}{n}}=n^{(\frac{1}{2})^{\log_2n}}$, the depth is $\log_2n+1$.\\
$T(n)=\Theta(\sum\limits_{i=0}^{\log_2n}n)=\Theta(n\lg n)$.\\

\noindent \textbf{Question}:\\
We note the depth as $k$, and $k$ satisfies $n^{2^{-k}}<2$, we have $k>\log_2(\log_2n)$.\\
Then $T(n)=\Theta(n\lg\lg n)$.\\
\textbf{For the First method}, we go through the recursion-tree until $n\to 1$, and the answer is $\Theta(n\lg n)$. \\
\textbf{For the second method}, we go through the recursion-tree until $n<2$,  assume that T(p) which satisfies $p\to 2$ is a constant, and the answer is $\Theta(n\lg \lg n)$.\\
\textbf{Which method is correct?}
\end{solution}
%%%%%%%%%%%%%%%

%%%%%%%%%%%%%%%%%%%%
\beginot

%%%%%%%%%%%%%%%
\begin{ot}[Akra-Bazzi Method]
  介绍求解递归式的 Akra-Bazzi Method, 比如定理介绍、应用与简要证明思路。

  \noindent 参考资料:
  \begin{itemize}
    \item 论文 ``On the Solution of Linear Recurrence Equations''~\cite{ABMethod}。
    \item \href{https://en.wikipedia.org/wiki/Akra\%E2\%80\%93Bazzi\_method}{Akra–Bazzi method @ wiki}
    \item 更多精彩, 由你掌握。
  \end{itemize}
\end{ot}

% \begin{solution}
% \end{solution}
%%%%%%%%%%%%%%%
\vspace{0.50cm}
%%%%%%%%%%%%%%%
\begin{ot}[\textsc{Merge-Sort}]
  请你深入分析 \textsc{Merge-Sort}, 例如:
  \begin{itemize}
    \item 严格求解 \textsc{Merge-Sort} 的递推式
      \[
        T(n) = T(\lfloor n/2 \rfloor) + T(\lceil n/2 \rceil) + N, \text{for } n > 1 \text{ with } T(1) = 0.
      \]
      参考资料: Section 2.6 of ``An Introduction to the Analysis of Algorithm'' (2nd Edition)~\cite{AoA}。
    \item \textsc{Merge} 阶段的下界。
      重点介绍两个有序数组大小相同的情况; 可概述其它情况。
      参考资料: Section 5.3.2 ``Minimum Comparison Merging'' of TAOCP Vol 3~\cite{TAOCP-Vol3}。
    \item 更多精彩, 由你掌握。
  \end{itemize}
\end{ot}

% \begin{solution}
% \end{solution}
%%%%%%%%%%%%%%%

%%%%%%%%%%%%%%%%%%%%
% 如果没有需要订正的题目,可以把这部分删掉

% \begincorrection
%%%%%%%%%%%%%%%%%%%%

%%%%%%%%%%%%%%%%%%%%
% 如果没有反馈,可以把这部分删掉
\beginfb

% 你可以写
% ~\footnote{优先推荐 \href{problemoverflow.top}{ProblemOverflow}}:
% \begin{itemize}
%   \item 对课程及教师的建议与意见
%   \item 教材中不理解的内容
%   \item 希望深入了解的内容
%   \item $\cdots$
% \end{itemize}
%%%%%%%%%%%%%%%%%%%%
\bibliography{2-4-recurrence}
\bibliographystyle{plainnat}
%%%%%%%%%%%%%%%%%%%%
\end{document}