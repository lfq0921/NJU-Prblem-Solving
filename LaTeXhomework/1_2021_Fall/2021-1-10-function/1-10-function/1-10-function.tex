% 1-10-function.tex

%%%%%%%%%%%%%%%%%%%%
\documentclass[a4paper, justified]{tufte-handout}

% hw-preamble.tex

% geometry for A4 paper
% See https://tex.stackexchange.com/a/119912/23098
\geometry{
  left=20.0mm,
  top=20.0mm,
  bottom=20.0mm,
  textwidth=130mm, % main text block
  marginparsep=5.0mm, % gutter between main text block and margin notes
  marginparwidth=50.0mm % width of margin notes
}

% for colors
\usepackage{xcolor} % usage: \color{red}{text}
% predefined colors
\newcommand{\red}[1]{\textcolor{red}{#1}} % usage: \red{text}
\newcommand{\blue}[1]{\textcolor{blue}{#1}}
\newcommand{\teal}[1]{\textcolor{teal}{#1}}

\usepackage{todonotes}

% heading
\usepackage{sectsty}
\setcounter{secnumdepth}{2}
\allsectionsfont{\centering\huge\rmfamily}

% for Chinese
\usepackage{xeCJK}
\usepackage{zhnumber}
\setCJKmainfont[BoldFont=FandolSong-Bold.otf]{FandolSong-Regular.otf}

% for fonts
\usepackage{fontspec}
\newcommand{\song}{\CJKfamily{song}} 
\newcommand{\kai}{\CJKfamily{kai}} 

% To fix the ``MakeTextLowerCase'' bug:
% See https://github.com/Tufte-LaTeX/tufte-latex/issues/64#issuecomment-78572017
% Set up the spacing using fontspec features
\renewcommand\allcapsspacing[1]{{\addfontfeature{LetterSpace=15}#1}}
\renewcommand\smallcapsspacing[1]{{\addfontfeature{LetterSpace=10}#1}}

% for url
\usepackage{hyperref}
\hypersetup{colorlinks = true, 
  linkcolor = teal,
  urlcolor  = teal,
  citecolor = blue,
  anchorcolor = blue}

\newcommand{\me}[4]{
    \author{
      {\bfseries 姓名:}\underline{#1}\hspace{2em}
      {\bfseries 学号:}\underline{#2}\hspace{2em}\\[10pt]
      {\bfseries 评分:}\underline{#3\hspace{3em}}\hspace{2em}
      {\bfseries 评阅:}\underline{#4\hspace{3em}}
  }
}

% Please ALWAYS Keep This.
\newcommand{\noplagiarism}{
  \begin{center}
    \fbox{\begin{tabular}{@{}c@{}}
      请独立完成作业,不得抄袭。\\
      若得到他人帮助, 请致谢。\\
      若参考了其它资料,请给出引用。\\
      鼓励讨论,但需独立书写解题过程。
    \end{tabular}}
  \end{center}
}

\newcommand{\goal}[1]{
  \begin{center}{\fcolorbox{blue}{yellow!60}{\parbox{0.50\textwidth}{\large 
    \begin{itemize}
      \item 体会``思维的乐趣''
      \item 初步了解递归与数学归纳法 
      \item 初步接触算法概念与问题下界概念
    \end{itemize}}}}
  \end{center}
}

% Each hw consists of four parts:
\newcommand{\beginrequired}{\hspace{5em}\section{作业 (必做部分)}}
\newcommand{\beginoptional}{\section{作业 (选做部分)}}
\newcommand{\beginot}{\section{Open Topics}}
\newcommand{\begincorrection}{\section{订正}}
\newcommand{\beginfb}{\section{反馈}}

% for math
\usepackage{amsmath, mathtools, amsfonts, amssymb}
\newcommand{\set}[1]{\{#1\}}

% define theorem-like environments
\usepackage[amsmath, thmmarks]{ntheorem}

\theoremstyle{break}
\theorempreskip{2.0\topsep}
\theorembodyfont{\song}
\theoremseparator{}
\newtheorem{problem}{题目}[subsection]
\renewcommand{\theproblem}{\arabic{problem}}
\newtheorem{ot}{Open Topics}

\theorempreskip{3.0\topsep}
\theoremheaderfont{\kai\bfseries}
\theoremseparator{:}
\theorempostwork{\bigskip\hrule}
\newtheorem*{solution}{解答}
\theorempostwork{\bigskip\hrule}
\newtheorem*{revision}{订正}

\theoremstyle{plain}
\newtheorem*{cause}{错因分析}
\newtheorem*{remark}{注}

\theoremstyle{break}
\theorempostwork{\bigskip\hrule}
\theoremsymbol{\ensuremath{\Box}}
\newtheorem*{proof}{证明}

% \newcommand{\ot}{\blue{\bf [OT]}}

% for figs
\renewcommand\figurename{图}
\renewcommand\tablename{表}

% for fig without caption: #1: width/size; #2: fig file
\newcommand{\fig}[2]{
  \begin{figure}[htbp]
    \centering
    \includegraphics[#1]{#2}
  \end{figure}
}
% for fig with caption: #1: width/size; #2: fig file; #3: caption
\newcommand{\figcap}[3]{
  \begin{figure}[htbp]
    \centering
    \includegraphics[#1]{#2}
    \caption{#3}
  \end{figure}
}
% for fig with both caption and label: #1: width/size; #2: fig file; #3: caption; #4: label
\newcommand{\figcaplbl}[4]{
  \begin{figure}[htbp]
    \centering
    \includegraphics[#1]{#2}
    \caption{#3}
    \label{#4}
  \end{figure}
}
% for margin fig without caption: #1: width/size; #2: fig file
\newcommand{\mfig}[2]{
  \begin{marginfigure}
    \centering
    \includegraphics[#1]{#2}
  \end{marginfigure}
}
% for margin fig with caption: #1: width/size; #2: fig file; #3: caption
\newcommand{\mfigcap}[3]{
  \begin{marginfigure}
    \centering
    \includegraphics[#1]{#2}
    \caption{#3}
  \end{marginfigure}
}

\usepackage{fancyvrb}

% for algorithms
\usepackage[]{algorithm}
\usepackage[]{algpseudocode} % noend
% See [Adjust the indentation whithin the algorithmicx-package when a line is broken](https://tex.stackexchange.com/a/68540/23098)
\newcommand{\algparbox}[1]{\parbox[t]{\dimexpr\linewidth-\algorithmicindent}{#1\strut}}
\newcommand{\hStatex}[0]{\vspace{5pt}}
\makeatletter
\newlength{\trianglerightwidth}
\settowidth{\trianglerightwidth}{$\triangleright$~}
\algnewcommand{\LineComment}[1]{\Statex \hskip\ALG@thistlm \(\triangleright\) #1}
\algnewcommand{\LineCommentCont}[1]{\Statex \hskip\ALG@thistlm%
  \parbox[t]{\dimexpr\linewidth-\ALG@thistlm}{\hangindent=\trianglerightwidth \hangafter=1 \strut$\triangleright$ #1\strut}}
\makeatother

% for footnote/marginnote
% see https://tex.stackexchange.com/a/133265/23098
\usepackage{tikz}
\newcommand{\circled}[1]{%
  \tikz[baseline=(char.base)]
  \node [draw, circle, inner sep = 0.5pt, font = \tiny, minimum size = 8pt] (char) {#1};
}
\renewcommand\thefootnote{\protect\circled{\arabic{footnote}}} % feel free to modify this file
%%%%%%%%%%%%%%%%%%%%
\title{第10讲: 函数}
\me{林凡琪}{211240042}{}{}
\date{\zhtoday} % or like 2019年9月13日
%%%%%%%%%%%%%%%%%%%%
\begin{document}
\maketitle
%%%%%%%%%%%%%%%%%%%%
\noplagiarism % always keep this line
%%%%%%%%%%%%%%%%%%%%
\begin{abstract}
	\begin{center}{\fcolorbox{blue}{yellow!60}{\parbox{0.65\textwidth}{\large
					\begin{itemize}
						\item 有了 functions,(大部分) 数学就 functions 了。
					\end{itemize}}}}
	\end{center}
\end{abstract}
%%%%%%%%%%%%%%%%%%%%
\beginrequired

%%%%%%%%%%%%%%%
\begin{problem}[UD Problem 14.3 (b, d, g)]
\end{problem}

\begin{solution}
	(b) The relation in (b) is not function, because when $x = 1$ and $x \in \mathbb{R} $, The denominator in f (x) is 0, there is no element to match with it.So it violates condition(i).\\
	(d)The relation in (d) is function.$\forall x \in [a, b], f(x) = a, a \in \mathbb{R}$;and $ \forall x \in [a, b]$, there is only $a\in \mathbb{R}$ can satisfy $(x, a) \in f$\\
	(g)The relation in (g) is not function.$\exists x = 6k(k \in \mathbb{Z} ) s.t. x \in 2\mathbb{Z} and x \in 3\mathbb{Z} $such that $f(x) = x + 1 = x - 1$, that means -1 = 1, which is obviously wrong.
\end{solution}
%%%%%%%%%%%%%%%

%%%%%%%%%%%%%%%
\begin{problem}[UD Problem 14.5]
\end{problem}

\begin{proof}
	for condition(i): let there be a set $A \subseteq \mathbb{R} $ there doesn't exist $b \in \mathbb{Z} $ such that $(A, b) \in f$. That means $A \cap \mathbb{N} = \emptyset$ is false and $A \cap \mathbb{N} \neq  \emptyset$ is also false, in other words $A \cap \mathbb{N} = \emptyset$ and $A \cap \mathbb{N} \neq  \emptyset$, which is obviously wrong.So there is no such A. This assumption is wrong.\\
	for condition(ii): if and only if $A \cap \mathbb{N} = \emptyset$ and $A \cap \mathbb{N} \neq  \emptyset$, there will be two different values corresponding to the same $A$. There is no such $A$.
\end{proof}
%%%%%%%%%%%%%%%

%%%%%%%%%%%%%%%
\begin{problem}[UD Problem 14.23]
\end{problem}

\begin{solution}
	x = x
\end{solution}
%%%%%%%%%%%%%%%

%%%%%%%%%%%%%%%
\begin{problem}[UD Problem 15.10 (f, g, h)]
\end{problem}

\begin{solution}
	(f) The function is one-to-one. \\
	$\forall a_1, a_2,f(a_1) = f(a_2) \rightarrow \exists b\in B$, and  $(a_1, b) = (a_2, b) \rightarrow a_1 = a_2$;\\
	The function is onto.\\

	(g)The function is one-to-one.\\
	$\forall A, B \subseteq P(X)$, $f(A) = f(B) \rightarrow X \backslash A = X \backslash B \rightarrow A = B$\\
	The functionn is onto.\\
	(h)The function is not one-to-one.\\
	Let $B \subseteq X, A \subseteq X \backslash B$ such that $A \cap B = \emptyset$. Obviously, A can be $\emptyset$ and for $B != X$ there must exist another C != $emptyset$ and $C \subseteq X \backslash B$ such that $C \cap B = \emptyset$.\\
	example:$X = \{1, 2, 3, 4, 5\}; B = {1, 2}; A = \emptyset; C = {3}$, $C \cap B = A \cap B = \emptyset$ and $A != C$\\
	The function is not onto. The range is $B$.

\end{solution}
%%%%%%%%%%%%%%%

%%%%%%%%%%%%%%%
\begin{problem}[UD Problem 15.14]
\end{problem}

\begin{solution}
	Let $x \in [a, b], y \in [c, d]$\\
	Define $f:[a, b]\rightarrow [c, d]$ by $f(x) = ((x - a) / (b - a)) * (d - c)$;\\
	$proof.$\\
	one-to-one:f(q) = f(p) $\rightarrow$ $((q - a) / (b - a)) * (d - c) = ((p - a) / (b - a)) * (d - c)$, it's easy to get that $p = q$.\\
	onto:Let $y \in [c, d]$ and let $x = (y / (d - c)) * (b - a) + a$ Then $x \in [a, b] = dom(f)$ and $f(x) = (((y / (d - c)) * (b - a) + a) - a) / (b - a)) * (d - c) = y$.\\
\end{solution}
%%%%%%%%%%%%%%%

%%%%%%%%%%%%%%%
\begin{problem}[UD Problem 15.15]
\end{problem}

\begin{solution}
	$\phi$ is a function from $F([0, 1])$ to $\mathbb{R}$.\\
	$\forall f \in F([0, 1]), \exists y = f(0) = \phi(f)$\\
	$\forall f \in F([0, 1]), \forall y_1, y_2 \in \mathbb{R}$ , and $\phi (f) = f(0) = y_1 = y_2$\\
	$\phi$ is not one to one. $f_1(x) = x, f_2(x) = x^2$, $\phi(f_1) = f_1(0) = 0, \phi(f_2) = f_2(0) = 0$, we can get that $\phi(f_1) = \phi(f_2), f_1 \neq f_2$.\\
	$\phi$ is onto.$\forall y \in \mathbb{R}, \exists f \in dom(\phi), \phi(f) = f(0) = y$\\
	example:Let $f_q = x + q, q \in \mathbb{R}$, $f(0) = q \in \mathbb{R} $.
\end{solution}
%%%%%%%%%%%%%%%

%%%%%%%%%%%%%%%
\begin{problem}[UD Problem 16.6]
\end{problem}

\begin{solution}
	(a) $f \circ g = \frac{\frac{3 + 2x}{1-x} - 3}{\frac{3 + 2x}{1-x} + 2} = x (x \neq 1)$\\
	$g\circ f = \frac{3 + 2 * \frac{x - 3}{x + 2}}{1 - \frac{x - 3}{x + 2}} = x, (x \neq -2)$\\
	(b) $g = f^{-1}$\\
	Theorem 16.4. Let $f:A \Rightarrow B$ be a bijective function. Then (iv) $If g: B \rightarrow A$ is a functioni satisfying $f \circ g = i_B or g \circ f = i_A, then g = f^{-1}.$
\end{solution}
%%%%%%%%%%%%%%%

%%%%%%%%%%%%%%%
\begin{problem}[UD Problem 16.14]
\end{problem}

\begin{solution}
	$\forall y \in B, \exists x, f(g_1(x)) = f(g_2(x)) = y$ so $g_1(x) = g_2(x).$\\
	$\forall y \in A, \exists x, g_1(f(x)) = g_2(f(x)) = y$, that is $g_1 = g_2.$\\
\end{solution}
%%%%%%%%%%%%%%%

%%%%%%%%%%%%%%%
\begin{problem}[UD Problem 16.17]
\end{problem}

\begin{solution}
	(a)\\
	one-to-one:\\
	let $H(a_1, c_1) = H(a_2, c_2) = (f(a_1), f(c_1)) = (f(a_2), f(c_2))$. That is $f(a_1) = f(a_2)$ and $f(c_1) = f(c_2)$. So $a_1 = a_2, c_1 = c_2$\\
	\\
	function:\\
	$\forall x_1 \in A,x_2 \in C$ there exists $f(x_1) \in B$ and $f(x_2) \in D$, that is $\forall (x_1, x_2) \in A\times C$ there exists $(f(x_1), f(x_2)) \in B \times D$.\\
	(b)\\
	$\forall y_1 \in B$, $\exists x_1 \in A, f(x_1) = y_1$,and $\forall y_2 \in B$, $\exists x_2 \in A, g(x_2) = y_2$\\
	So $\forall (y_1, y_2) \in B\times D$, $\exists (x_1, x_2) \in A \times C$ $(f(x_1), g(x_2)) = (y_1, y_2)$\\
\end{solution}
%%%%%%%%%%%%%%%

%%%%%%%%%%%%%%%
\begin{problem}[UD Problem 16.22]
\end{problem}

\begin{proof}
	$f(f(x)) = x$
	$f(x_1) = f(x_2)\rightarrow f(f(x_1)) = f(f(x_2)) \rightarrow x_1 = x_2$\\
	So $f$ is bijective.
\end{proof}
%%%%%%%%%%%%%%%

%%%%%%%%%%%%%%%
\begin{problem}[UD Problem 17.22]
\end{problem}

\begin{solution}
	(a)Needn't.\\
	Let $f(x) = x^2$, $A_1 = \{1, 2\}, A_2 = \{-1, -2\}$, $f(A_1) = f(A_2) = \{1, 4\}$\\
	(b) $\forall x_1 \in A_1, \forall x_2 \in A_2$, for $f$ is a bijective function, $f(x_1) = f(x_2) \rightarrow x_1 = x_2$.\\
	So if $f(A_1) = f(A_2)$, then $\forall x_1 \in A_1 \Leftrightarrow  \forall x_2 \in A_2, \Rightarrow A_1 = A_2$.
\end{solution}
%%%%%%%%%%%%%%%

%%%%%%%%%%%%%%%
\begin{problem}[UD Problem 17.23]
\end{problem}

\begin{solution}
	(a)Needn't.\\
	Let $f(x) = x^2$, $B_1 = \{-1, 0\}, B_2 = \{0\}$, obviously $f^{-1}(B_1) = f^{-1}(B_2)$ and $B_1 \neq B_2$.\\
	(b)Because $f$ is onto, $\forall y_1\in B_1, \forall y_2 \in B_2, \exists x_1 \in f^{-1}(B_1), \exists x_2 \in f^{-1}(B_2), f(x_1) = y_1, f(x_2) = y_2$(onto)\\
	If $f^{-1}(y_1) = f^{-1}(y_2) \rightarrow y_1 = y_2$.(one-to-one)\\
	So if $f^{-1}(B_1) = f^{-1}(B_2) \rightarrow B_1 = B_2$
\end{solution}
%%%%%%%%%%%%%%%

%%%%%%%%%%%%%%%%%%%%
\beginoptional

%%%%%%%%%%%%%%%
\begin{problem}[Monotonicity]
Assume that $F: \mathcal{P}(A) \to \mathcal{P}(A)$ and that $F$ has the monotonicity property:
\[
	X \subseteq Y \subseteq A \implies F(X) \subseteq F(Y).
\]

\noindent Define
\[
	B = \bigcap \set{X \subseteq A \mid F(X) \subseteq X}
\]

\[
	C = \bigcup \set{X \subseteq A \mid X \subseteq F(X)}.
\]

\begin{enumerate}[(a)]
	\item Show that $F(B) = B$ and $F(C) = C$.
	\item Show that if $F(X) = X$, then $B \subseteq X \subseteq C$.
\end{enumerate}
\end{problem}

\begin{solution}
\end{solution}
%%%%%%%%%%%%%%%

%%%%%%%%%%%%%%%%%%%%
\beginot

%%%%%%%%%%%%%%%
%%%%%%%%%%%%%%%
\begin{ot}[自然数]
	介绍如何使用集合定义 (不限于):
	\begin{itemize}
		\item 自然数
		\item 自然数上的大小关系
		\item 自然数上的运算
	\end{itemize}

	\noindent 参考资料:
	\begin{itemize}
		\item \href{https://en.wikipedia.org/wiki/Natural\_number}{Natural number @ wiki}
	\end{itemize}
\end{ot}

% \begin{solution}
% \end{solution}
%%%%%%%%%%%%%%%
\vspace{0.50cm}
%%%%%%%%%%%%%%%
\begin{ot}[选择公理]
	介绍选择公理 (Axiom of Choice), 如 (不限于):
	\begin{itemize}
		\item 不同定义形式
		\item 怎么理解 (怎么也不理解)
		\item 有什么用
	\end{itemize}

	\noindent 参考资料:
	\begin{itemize}
		\item \href{https://en.wikipedia.org/wiki/Axiom\_of\_choice}{Axiom of choice @ wiki}
		\item \href{https://plato.stanford.edu/entries/axiom-choice/}{The Axiom of Choice @ Stanford Encyclopedia of Philosophy}
	\end{itemize}
\end{ot}

% \begin{solution}
% \end{solution}
%%%%%%%%%%%%%%%


%%%%%%%%%%%%%%%%%%%%
% 如果没有需要订正的题目,可以把这部分删掉

\begincorrection

%%%%%%%%%%%%%%%%%%%%

%%%%%%%%%%%%%%%%%%%%
% 如果没有反馈,可以把这部分删掉
\beginfb

% 你可以写
% ~\footnote{优先推荐 \href{problemoverflow.top}{ProblemOverflow}}:
% \begin{itemize}
%   \item 对课程及教师的建议与意见
%   \item 教材中不理解的内容
%   \item 希望深入了解的内容
%   \item $\cdots$
% \end{itemize}
%%%%%%%%%%%%%%%%%%%%
\end{document}