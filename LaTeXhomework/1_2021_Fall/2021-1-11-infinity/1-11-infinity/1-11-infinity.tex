% 1-11-infinity.tex

%%%%%%%%%%%%%%%%%%%%
\documentclass[a4paper, justified]{tufte-handout}

% hw-preamble.tex

% geometry for A4 paper
% See https://tex.stackexchange.com/a/119912/23098
\geometry{
  left=20.0mm,
  top=20.0mm,
  bottom=20.0mm,
  textwidth=130mm, % main text block
  marginparsep=5.0mm, % gutter between main text block and margin notes
  marginparwidth=50.0mm % width of margin notes
}

% for colors
\usepackage{xcolor} % usage: \color{red}{text}
% predefined colors
\newcommand{\red}[1]{\textcolor{red}{#1}} % usage: \red{text}
\newcommand{\blue}[1]{\textcolor{blue}{#1}}
\newcommand{\teal}[1]{\textcolor{teal}{#1}}

\usepackage{todonotes}

% heading
\usepackage{sectsty}
\setcounter{secnumdepth}{2}
\allsectionsfont{\centering\huge\rmfamily}

% for Chinese
\usepackage{xeCJK}
\usepackage{zhnumber}
\setCJKmainfont[BoldFont=FandolSong-Bold.otf]{FandolSong-Regular.otf}

% for fonts
\usepackage{fontspec}
\newcommand{\song}{\CJKfamily{song}} 
\newcommand{\kai}{\CJKfamily{kai}} 

% To fix the ``MakeTextLowerCase'' bug:
% See https://github.com/Tufte-LaTeX/tufte-latex/issues/64#issuecomment-78572017
% Set up the spacing using fontspec features
\renewcommand\allcapsspacing[1]{{\addfontfeature{LetterSpace=15}#1}}
\renewcommand\smallcapsspacing[1]{{\addfontfeature{LetterSpace=10}#1}}

% for url
\usepackage{hyperref}
\hypersetup{colorlinks = true, 
  linkcolor = teal,
  urlcolor  = teal,
  citecolor = blue,
  anchorcolor = blue}

\newcommand{\me}[4]{
    \author{
      {\bfseries 姓名:}\underline{#1}\hspace{2em}
      {\bfseries 学号:}\underline{#2}\hspace{2em}\\[10pt]
      {\bfseries 评分:}\underline{#3\hspace{3em}}\hspace{2em}
      {\bfseries 评阅:}\underline{#4\hspace{3em}}
  }
}

% Please ALWAYS Keep This.
\newcommand{\noplagiarism}{
  \begin{center}
    \fbox{\begin{tabular}{@{}c@{}}
      请独立完成作业,不得抄袭。\\
      若得到他人帮助, 请致谢。\\
      若参考了其它资料,请给出引用。\\
      鼓励讨论,但需独立书写解题过程。
    \end{tabular}}
  \end{center}
}

\newcommand{\goal}[1]{
  \begin{center}{\fcolorbox{blue}{yellow!60}{\parbox{0.50\textwidth}{\large 
    \begin{itemize}
      \item 体会``思维的乐趣''
      \item 初步了解递归与数学归纳法 
      \item 初步接触算法概念与问题下界概念
    \end{itemize}}}}
  \end{center}
}

% Each hw consists of four parts:
\newcommand{\beginrequired}{\hspace{5em}\section{作业 (必做部分)}}
\newcommand{\beginoptional}{\section{作业 (选做部分)}}
\newcommand{\beginot}{\section{Open Topics}}
\newcommand{\begincorrection}{\section{订正}}
\newcommand{\beginfb}{\section{反馈}}

% for math
\usepackage{amsmath, mathtools, amsfonts, amssymb}
\newcommand{\set}[1]{\{#1\}}

% define theorem-like environments
\usepackage[amsmath, thmmarks]{ntheorem}

\theoremstyle{break}
\theorempreskip{2.0\topsep}
\theorembodyfont{\song}
\theoremseparator{}
\newtheorem{problem}{题目}[subsection]
\renewcommand{\theproblem}{\arabic{problem}}
\newtheorem{ot}{Open Topics}

\theorempreskip{3.0\topsep}
\theoremheaderfont{\kai\bfseries}
\theoremseparator{:}
\theorempostwork{\bigskip\hrule}
\newtheorem*{solution}{解答}
\theorempostwork{\bigskip\hrule}
\newtheorem*{revision}{订正}

\theoremstyle{plain}
\newtheorem*{cause}{错因分析}
\newtheorem*{remark}{注}

\theoremstyle{break}
\theorempostwork{\bigskip\hrule}
\theoremsymbol{\ensuremath{\Box}}
\newtheorem*{proof}{证明}

% \newcommand{\ot}{\blue{\bf [OT]}}

% for figs
\renewcommand\figurename{图}
\renewcommand\tablename{表}

% for fig without caption: #1: width/size; #2: fig file
\newcommand{\fig}[2]{
  \begin{figure}[htbp]
    \centering
    \includegraphics[#1]{#2}
  \end{figure}
}
% for fig with caption: #1: width/size; #2: fig file; #3: caption
\newcommand{\figcap}[3]{
  \begin{figure}[htbp]
    \centering
    \includegraphics[#1]{#2}
    \caption{#3}
  \end{figure}
}
% for fig with both caption and label: #1: width/size; #2: fig file; #3: caption; #4: label
\newcommand{\figcaplbl}[4]{
  \begin{figure}[htbp]
    \centering
    \includegraphics[#1]{#2}
    \caption{#3}
    \label{#4}
  \end{figure}
}
% for margin fig without caption: #1: width/size; #2: fig file
\newcommand{\mfig}[2]{
  \begin{marginfigure}
    \centering
    \includegraphics[#1]{#2}
  \end{marginfigure}
}
% for margin fig with caption: #1: width/size; #2: fig file; #3: caption
\newcommand{\mfigcap}[3]{
  \begin{marginfigure}
    \centering
    \includegraphics[#1]{#2}
    \caption{#3}
  \end{marginfigure}
}

\usepackage{fancyvrb}

% for algorithms
\usepackage[]{algorithm}
\usepackage[]{algpseudocode} % noend
% See [Adjust the indentation whithin the algorithmicx-package when a line is broken](https://tex.stackexchange.com/a/68540/23098)
\newcommand{\algparbox}[1]{\parbox[t]{\dimexpr\linewidth-\algorithmicindent}{#1\strut}}
\newcommand{\hStatex}[0]{\vspace{5pt}}
\makeatletter
\newlength{\trianglerightwidth}
\settowidth{\trianglerightwidth}{$\triangleright$~}
\algnewcommand{\LineComment}[1]{\Statex \hskip\ALG@thistlm \(\triangleright\) #1}
\algnewcommand{\LineCommentCont}[1]{\Statex \hskip\ALG@thistlm%
  \parbox[t]{\dimexpr\linewidth-\ALG@thistlm}{\hangindent=\trianglerightwidth \hangafter=1 \strut$\triangleright$ #1\strut}}
\makeatother

% for footnote/marginnote
% see https://tex.stackexchange.com/a/133265/23098
\usepackage{tikz}
\newcommand{\circled}[1]{%
  \tikz[baseline=(char.base)]
  \node [draw, circle, inner sep = 0.5pt, font = \tiny, minimum size = 8pt] (char) {#1};
}
\renewcommand\thefootnote{\protect\circled{\arabic{footnote}}} % feel free to modify this file
%%%%%%%%%%%%%%%%%%%%
\title{第11讲: 有穷与无穷}
\me{林凡琪}{211240042}{}{}
\date{\zhtoday} % or like 2019年9月13日
%%%%%%%%%%%%%%%%%%%%
\begin{document}
\maketitle
%%%%%%%%%%%%%%%%%%%%
\noplagiarism % always keep this line
%%%%%%%%%%%%%%%%%%%%
\begin{abstract}
  \mfigcap{width = 0.80\textwidth}{figs/cantor}{Georg Cantor (1845 $\sim$ 1918)}
  \begin{center}{\fcolorbox{blue}{yellow!60}{\parbox{0.65\textwidth}{\large
          \begin{itemize}
            \item ``Veniet tempus, quo ista que nunc latent, in lucem dies extrahat et longioris avi diligentia.''
          \end{itemize}}}}
  \end{center}
\end{abstract}
%%%%%%%%%%%%%%%%%%%%
\beginrequired

%%%%%%%%%%%%%%%
\begin{problem}[UD Problem 21.6]
\end{problem}

\begin{solution}
  (a)Let $f(x) = tan(\pi x - \frac{\pi}{2}), x\in(0, 1), ran(f) = \mathbb{R}.$\\
  (b)Let $f(x) = |x|, x \in (\mathbb{R} / {0}),\\
    f(0) = 1$\\
  then $f$ is the bijection from $\mathbb{R}$ to $\mathbb{R}^+.$
\end{solution}
%%%%%%%%%%%%%%%

%%%%%%%%%%%%%%%
\begin{problem}[UD Problem 21.10]
\end{problem}

\begin{proof}
  According to Problem 16.17, let $f: A \rightarrow B$ and $g:C \rightarrow D$ be functions. Then $H:A\times C \rightarrow B \times D$ is a one-to-one function.\\
  Similarly, $A\approx C$ and $B\approx D$(it means there exists bijections, $f:A \rightarrow C$, $g:B \rightarrow D$),so it is easy to know that $H:A\times B \rightarrow C \times D$, that is $A\times B \approx C \times D$.\\
\end{proof}
%%%%%%%%%%%%%%%

%%%%%%%%%%%%%%%
\begin{problem}[UD Problem 22.10]
\end{problem}

\begin{solution}
  According to Corollary 21.10, let S be a finite set. Then every subset of S is finite.\\
  B is an infinite subset of A, so A can't be a finite set, that is A is a infinite set.
\end{solution}
%%%%%%%%%%%%%%%

%%%%%%%%%%%%%%%
\begin{problem}[UD Problem 22.11]
\end{problem}

\begin{solution}
  If there doesn't exist $b \in B$ such that $A_b = f^{-1}(\{b\})$ is infinite, then $ran(f^{-1}) = \cup A_b$ is finite, while actually $ran(f^{-1}) = A$ is infinite. \\
  Thus there exists $b \in B$ such that $f^{-1}(\{b\})$ is infinite.
\end{solution}
%%%%%%%%%%%%%%%

%%%%%%%%%%%%%%%
\begin{problem}[UD Problem 22.18]
\end{problem}

\begin{solution}
  (a)\\
  Let the cardinality of A to be n, the cardinality of B be m.\\
  Suppose, to the contrary, |B| > |A|, which means m > n.\\
  According to the pigeonhole principle, we can get that $f:B \rightarrow A$ is not a one-to-one function.\\
  But actually $B \subseteq A$, we can easily get that $g:B \rightarrow A$ by $g(x) = x$.It is contradict with our assumption.\\
  So $|B| <= |A|$.\\
  \\
  (b)\\
  According  to (a), A is a finite set and $B \subseteq A$, then $|B| <= |A|$.We suppose that |B| = n, |A| = m. \\
  If $|B| = |A|$. And $B \subseteq A$, then $\forall b\in B, \exists a\in A$, that $a = b$. \\
  Because there is m elements in B, so there is at least n elements which satisfy $a = b$ in A. And m = n, so all elements in A satisfy $a = b$, that is A = B, which is contradict with $B \neq A$. \\
  So $|B| = |A|$ is wrong.\\
  Since
  $|B| <= |A|$ and $|B| \neq |A|$, we can know $|B| < |A|$.\\
  \\
  (c)\\
  According to (a) if $B \subseteq A$ then $|B| <= |A|$. \\
  So if in the same time $|B| >= |A|$ then $|B| = |A|$.\\
  According to (b), if $|B| = |A|$ then A = B.\\
\end{solution}
%%%%%%%%%%%%%%%

%%%%%%%%%%%%%%%
\begin{problem}[UD Problem 22.21]
\end{problem}

\begin{solution}
  1)Suppose to the contrary that if $f$ is one-to-one then $f$ is not onto.\\
  Let $f(x) = x, x\in A$. We can easily know $f$ is a one-to-one function.\\
  Meanwhile $f$ is onto.\\
  So the assumption is false.\\
  \\
  2)Suppose to the contrary that if $f$ is onto then $f$ is not one-to-one.\\
  Put forward the same example, $f(x) = x, x\in A$, Since $f$ is a bijection, which means $f$ is onto and $f$ is one-to-one.\\
  So the assumption is false.\\
  Judging from (1) and (2),  we can know $f:A\rightarrow A$ is one-to-one if and only if it is onto.\\
  \\
  Using the same example $f(x) = x, x\in A$, we can also reach the same conclusion when A is infinite.\\
\end{solution}
%%%%%%%%%%%%%%%

%%%%%%%%%%%%%%%
\begin{problem}[UD Problem 23.1]
\end{problem}

\begin{solution}
  (a)$\{k\mathbb{N} \}$, k is a prime number.\\
  (b)Not possible.\\
  (c)Not possible.\\

\end{solution}
%%%%%%%%%%%%%%%

%%%%%%%%%%%%%%%
\begin{problem}[UD Problem 23.3 (a, d)]
\end{problem}

\begin{solution}
  (a)It's countable.\\
  Define the set of all lines with rational slopes to be denoted by A.\\
  Let $f:A \rightarrow \mathbb{Q} $, for the line $l:y = kx + b,f(l) = k \in \mathbb{Q}$. So $A \approx \mathbb{Q}$.\\
  Since $\mathbb{Q}$ is countable, $A$ is also countable.

  (d)It's uncountable. Let $f:\mathbb{R} \rightarrow \mathbb{R}, y = f(x) = 1 - x$. Since $f{x}$ is a rigidly monotonically increasing function, $f$ is bijection.\\
  So$\{(x,y)\in \mathbb{R} \times \mathbb{R} :x + y = 1\} \approx \mathbb{R}$.Since $\mathbb{R}$ is infinity, $\{(x,y)\in \mathbb{R} \times \mathbb{R} :x + y = 1\}$ is also infinity.
\end{solution}
%%%%%%%%%%%%%%%

%%%%%%%%%%%%%%%
\begin{problem}[UD Problem 23.4]
\end{problem}

\begin{solution}
  It's uncountable.\\
  We will suppose, to the contrary, that the sequences are countable.Let the sequences be denoted by A. We can know that \{0, 1\} is finite, there exists a bijection function $f:\{0, 1\} \rightarrow A$. We will list the values of $f$. So\\
  $f(1) = a_{11}a_{12}a_{13}...$\\
  $f(2) = a_{21}a_{22}a_{23}...$\\
  $f(3) = a_{31}a_{32}a_{33}...$\\
  .\\
  .\\
  .\\
  where each $a_{i, j}$ represents 0 or 1. Since $f$ is onto, each sequence in A appears in this list.\\
  The odd thing is this: we can construct a number $b = b_1b_2...$ not in this list  (hence showing that our function cannot possibly be onto) by describing it as follows.\\
  We constructed $b$ so that $b_n \neq a_{nn}$ and therefore $b \neq f(n)$ for every n. Then $b$ can not be in our list. So this contradiction must mean that we have assunmed falsely that the sequence are countable.\\
\end{solution}
%%%%%%%%%%%%%%%

%%%%%%%%%%%%%%%
\begin{problem}[UD Problem 23.9]
\end{problem}

\begin{proof}
  First, if the set A is finite, which means it's countable, then its subset is also finite, so its subset is also countable.\\
  If the set A is countably infinite. There can be a bijection $f:A \rightarrow \mathbb{N} $. \\
  Define a restriction$f|_{B}:B \rightarrow \mathbb{N}$. Then $f|_{B}$ is a one-to-one function. \\
  And according to Exercise 23.5, B is countable.\\
\end{proof}
%%%%%%%%%%%%%%%

%%%%%%%%%%%%%%%
\begin{problem}[UD Problem 23.12]
\end{problem}

\begin{solution}
  $A_i = \{\frac{p}{i} | i \in \mathbb{N} ^+\}$
  Define a function $f_i:A_i \rightarrow \mathbb{N} ^+$ ny $f_i(x) = x \times i$.\\
  $\forall x_1, x_2 \in A_i,$if $f_i(x_1) = f_i(x_2)$ ,then $x_1 = x_2 = \frac{n}{i}$.\\
  So $f_i$ is one-to-one.\\
  $\Rightarrow A_i$ is countable.
  We will prove $\cup_{i \in \mathbb{N^+}}^n A_i$ is countable by introduction on $n$.\\
  $(i)$ (The base step) (n = 1), $\cup_{i \in \mathbb{N^+}}^1 A_i = A_1$ is obviously countable, and\\
  $(ii)$ (The induction step)for the positive integer $n$, $\cup_{i \in \mathbb{N^+}}^n A_i$ is countable.\\
  Then $\cup_{i \in \mathbb{N^+}}^{n + 1} A_i$ = $(\cup_{i \in \mathbb{N^+}}^n A_i) \cup A_{m + 1}$.\\
  According to Theorem 23.6, $\cup_{i \in \mathbb{N^+}}^{n + 1} A_i$ is countable.\\
  So, $\cup_{i \in \mathbb{N}_+} A_i$ is countable.
  So $\mathbb{Q}_+$ is countable.\\
  So $\mathbb{Q} = \mathbb{Q}_+ \cup \mathbb{Q}_- \cup \{0\}$ is countble.
\end{solution}
%%%%%%%%%%%%%%%

%%%%%%%%%%%%%%%
\begin{problem}[UD Problem 24.16]
\end{problem}

\begin{solution}
  In the proof of Theorem 23.12. We got that $(0, 1) \approx \mathbb{R}$.\\
  According to Theorem 21.13, $(0, 1) \times (0, 1) \approx \mathbb{R} \times \mathbb{R}$\\
  Define $f:(0, 1) \rightarrow (0, 1) \times (0, 1)$, $f$ is bijection. So $(0, 1) \approx (0, 1) \times (0, 1)$
  Since $(0, 1) \times (0, 1) \approx \mathbb{R} \times \mathbb{R}$, $(0, 1) \approx \mathbb{R} \times \mathbb{R}$\\
  So $\mathbb{R} \approx \mathbb{R} \times \mathbb{R}$\\
  Define $g:\mathbb{R} \times \mathbb{R} \rightarrow \mathbb{C}$ by $g(a, b) = a + bi, a,b \in \mathbb{R}$.\\
  Since $g$ is a bijection, $\mathbb{R} \times \mathbb{R} \approx \mathbb{C}$\\
  Since $\mathbb{R} \times \mathbb{R} \approx \mathbb{R}$, $\mathbb{R} \approx \mathbb{C}$.\\
  So $|\mathbb{R}| = |\mathbb{C}|$
\end{solution}
%%%%%%%%%%%%%%%

%%%%%%%%%%%%%%%%%%%%
\beginoptional

%%%%%%%%%%%%%%%
\begin{problem}[UD Problem 24.15]
\end{problem}

\begin{solution}
\end{solution}
%%%%%%%%%%%%%%%

%%%%%%%%%%%%%%%%%%%%
\beginot

注: 基数与序数比较难以理解。你可以选择介绍一些相对容易的部分。
%%%%%%%%%%%%%%%
\begin{ot}[基数]
  \mfig{width = 0.60\textwidth}{figs/aleph-null}
  请介绍基数 (Cardinal number) ,如 (不限于):
  \begin{itemize}
    \item 定义
    \item 运算
    \item 你认为有意思的相关内容
  \end{itemize}

  \noindent 参考资料:
  \begin{itemize}
    \item \href{https://en.wikipedia.org/wiki/Cardinal\_number}{Cardinal number @ wiki}
  \end{itemize}
\end{ot}

% \begin{solution}
% \end{solution}
%%%%%%%%%%%%%%%
\vspace{0.50cm}
%%%%%%%%%%%%%%%
\begin{ot}[序数]
  请介绍序数 (Ordinal number) ,如 (不限于):
  \mfig{width = 0.85\textwidth}{figs/ordinal-number}
  \begin{itemize}
    \item 定义
    \item 运算
    \item 你认为有意思的相关内容
  \end{itemize}

  \noindent 参考资料:
  \begin{itemize}
    \item \href{https://en.wikipedia.org/wiki/Ordinal\_number}{Ordinal number@ wiki}
  \end{itemize}
\end{ot}

% \begin{solution}
% \end{solution}
%%%%%%%%%%%%%%%

%%%%%%%%%%%%%%%%%%%%
% 如果没有需要订正的题目,可以把这部分删掉

\begincorrection
8-1\\
(f)$(B \cap C) \backslash A$\\
(g)$(A \cup B \cup C) \backslash (A \cap B \cap C)$ \\
\\
\\
8-2\\
(d)\\
$(i)$ \\
If $A \subseteq B$ then $\forall x\in A,\rightarrow x \in B$, so $\forall x \notin B \rightarrow x\notin A$. That is $\forall x \in (X \backslash B) \rightarrow x \in (X \backslash A)$.\\
So $(X \backslash B) \subseteq (X \backslash A)$\\
$(ii)$\\
If $(X \backslash B) \subseteq (X \backslash A)$, then $\forall x \in (X \backslash B) \rightarrow x \in (X \backslash A)$,that is $\forall x \notin B \rightarrow x\notin A$. So $\forall x\in A,\rightarrow x \in B$.\\
So $A \subseteq B$.\\
\\
(f)\\
$(i)$\\
If $A \cap B = B$\\
$\Rightarrow \forall x \in B, \rightarrow x \in A \cap B$, that is $x \in A$.\\
So $B \subseteq A$\\
$(ii)$\\
If $B \subseteq A$, then $\forall x \in B \rightarrow x \in A$. So $x \in A \cap B$.\\
In other words, $\forall x \in B \rightarrow x \in A \cap B$.\\
So $B = A\cap B$.\\
\\
\\
8-7\\
(a)\\
$\bigcup^\infty _{n=1}A_n$ = [0, 1)\\
$\bigcup^\infty _{n=1}B_n$ = [0, 1]\\
$\bigcup^\infty _{n=1}C_n$ = (0, 1)\\
(b)\\
$\bigcap^\infty _{n=1}A_n$ = \{0\}\\
$\bigcap^\infty _{n=1}B_n$ = \{0\}\\
$\bigcap^\infty _{n=1}C_n$ = $\emptyset$\\
\\
\\
8-11\\
Let $E \in \mathcal{P} (A_\alpha)$
\\then $E \in \mathcal{P} (A_{\alpha_1}) \cup \mathcal{P} (A_{\alpha_2}) \cup \mathcal{P} (A_{\alpha_3})...\cup\mathcal{P} (A_{\alpha_n}),\alpha_1, \alpha_2 ... \in I$\\
then $E \in \mathcal{P} (A_{\alpha_1})\cup \mathcal{P}(A_{\alpha_2})...\cup \mathcal{P}(A_{\alpha_n})$\\
So, $E \in \mathcal{P} (A_{\alpha_1}\cup A_{\alpha_2}...\cup A_{\alpha_n})$\\
\\
\\
8-12\\
(i)prove $\bigcap_{\alpha\in I}\mathcal{P} (A_\alpha) \subseteq \mathcal{P}(\bigcap_{\alpha\in I}A_\alpha):$\\
$E \in \bigcap_{\alpha\in I}\mathcal{P}(A_\alpha)$\\
then $E \in \mathcal{P} (A_{\alpha_1})\cap \mathcal{P}(A_{\alpha_2})...\cap \mathcal{P}(A_{\alpha_n})$\\
then $E \in \mathcal{P} (A_{\alpha_1}\cap A_{\alpha_2}...\cap A_{\alpha_n})$\\
so $E \in \mathcal{P}(\bigcap_{\alpha\in I}A_\alpha)$\\
(ii)prove $\mathcal{P}(\bigcap_{\alpha\in I}A_\alpha \subseteq \bigcap_{\alpha\in I}\mathcal{P} (A_\alpha) $\\
$E \in \mathcal{P} (\bigcap_{\alpha\in I}A_\alpha)$\\
then $E \in \mathcal{P} (A_{\alpha_1}\cap A_{\alpha_2}...\cap A_{\alpha_n})$\\
then $E \in \mathcal{P} (A_{\alpha_1})\cap \mathcal{P}(A_{\alpha_2})...\cap \mathcal{P}(A_{\alpha_n})$\\
then $E \in \mathcal{P} (A_\alpha) $\\
\\
SO $\bigcap_{\alpha\in I}\mathcal{P} (A_\alpha)   = \mathcal{P}(\bigcap_{\alpha\in I}A_\alpha)$\\
\\
\\
3-4
第一类数学归纳法证明第二类数学归纳法\\
要证第二类数学归纳法,也即任给一个命题$F$,若满足$F(1)$及$(F(1)\wedge F(2) \wedge ...\wedge F(n))\Rightarrow F(n + 1)$ ,则有$\forall k \in \mathbb{N} , F(k)$成立。\\
构造命题$G(n) = F(1)\wedge F(2) \wedge ...\wedge F(n)$\\
显然,$G(n)\Rightarrow F(n + 1)$,又$G(n)\Rightarrow G(n)$所以$G(n)\Rightarrow G(n)\wedge F(n + 1) = G(n+1)$\\
所以G满足第一类数学归纳法的条件,所以$\forall k \in \mathbb{N} , G(k)$成立。而$G(n)\Rightarrow F(n)$,故$\forall k \in \mathbb{N} , F(k)$成立 ,也即第二类数学归纳法成立。\\
\\
第二类数学归纳法证明第一类数学归纳法
要证第一类数学归纳法,也即任给一个命题$F$,若满足$F(1)$及$F(n)\rightarrow F(n + 1)$,则有$\forall k \in \mathbb{N} , F(k)$成立。\\
因为1的条件比2强,所以F一定满足第二类数学归纳法.故根据第二类数学归纳法$F(k)$对所有正整数$k$都成立,也即第一类数学归纳法成立.\\

显然, [公式] 是满足第二类数学归纳法的条件的(因为1的条件比2强),故根据第二类数学归纳法, [公式] 对所有正整数 [公式] 成立,也即第一类数学归纳法成立.\\
\\
\\
致谢:鄢振宇学长(zhihu)
%%%%%%%%%%%%%%%%%%%%

%%%%%%%%%%%%%%%%%%%%
% 如果没有反馈,可以把这部分删掉
\beginfb

% 你可以写
% ~\footnote{优先推荐 \href{problemoverflow.top}{ProblemOverflow}}:
% \begin{itemize}
%   \item 对课程及教师的建议与意见
%   \item 教材中不理解的内容
%   \item 希望深入了解的内容
%   \item $\cdots$
% \end{itemize}
%%%%%%%%%%%%%%%%%%%%
\end{document}