% 1-5-data-structure.tex

%%%%%%%%%%%%%%%%%%%%
\documentclass[a4paper, justified]{tufte-handout}

% hw-preamble.tex

% geometry for A4 paper
% See https://tex.stackexchange.com/a/119912/23098
\geometry{
  left=20.0mm,
  top=20.0mm,
  bottom=20.0mm,
  textwidth=130mm, % main text block
  marginparsep=5.0mm, % gutter between main text block and margin notes
  marginparwidth=50.0mm % width of margin notes
}

% for colors
\usepackage{xcolor} % usage: \color{red}{text}
% predefined colors
\newcommand{\red}[1]{\textcolor{red}{#1}} % usage: \red{text}
\newcommand{\blue}[1]{\textcolor{blue}{#1}}
\newcommand{\teal}[1]{\textcolor{teal}{#1}}

\usepackage{todonotes}

% heading
\usepackage{sectsty}
\setcounter{secnumdepth}{2}
\allsectionsfont{\centering\huge\rmfamily}

% for Chinese
\usepackage{xeCJK}
\usepackage{zhnumber}
\setCJKmainfont[BoldFont=FandolSong-Bold.otf]{FandolSong-Regular.otf}

% for fonts
\usepackage{fontspec}
\newcommand{\song}{\CJKfamily{song}} 
\newcommand{\kai}{\CJKfamily{kai}} 

% To fix the ``MakeTextLowerCase'' bug:
% See https://github.com/Tufte-LaTeX/tufte-latex/issues/64#issuecomment-78572017
% Set up the spacing using fontspec features
\renewcommand\allcapsspacing[1]{{\addfontfeature{LetterSpace=15}#1}}
\renewcommand\smallcapsspacing[1]{{\addfontfeature{LetterSpace=10}#1}}

% for url
\usepackage{hyperref}
\hypersetup{colorlinks = true, 
  linkcolor = teal,
  urlcolor  = teal,
  citecolor = blue,
  anchorcolor = blue}

\newcommand{\me}[4]{
    \author{
      {\bfseries 姓名:}\underline{#1}\hspace{2em}
      {\bfseries 学号:}\underline{#2}\hspace{2em}\\[10pt]
      {\bfseries 评分:}\underline{#3\hspace{3em}}\hspace{2em}
      {\bfseries 评阅:}\underline{#4\hspace{3em}}
  }
}

% Please ALWAYS Keep This.
\newcommand{\noplagiarism}{
  \begin{center}
    \fbox{\begin{tabular}{@{}c@{}}
      请独立完成作业,不得抄袭。\\
      若得到他人帮助, 请致谢。\\
      若参考了其它资料,请给出引用。\\
      鼓励讨论,但需独立书写解题过程。
    \end{tabular}}
  \end{center}
}

\newcommand{\goal}[1]{
  \begin{center}{\fcolorbox{blue}{yellow!60}{\parbox{0.50\textwidth}{\large 
    \begin{itemize}
      \item 体会``思维的乐趣''
      \item 初步了解递归与数学归纳法 
      \item 初步接触算法概念与问题下界概念
    \end{itemize}}}}
  \end{center}
}

% Each hw consists of four parts:
\newcommand{\beginrequired}{\hspace{5em}\section{作业 (必做部分)}}
\newcommand{\beginoptional}{\section{作业 (选做部分)}}
\newcommand{\beginot}{\section{Open Topics}}
\newcommand{\begincorrection}{\section{订正}}
\newcommand{\beginfb}{\section{反馈}}

% for math
\usepackage{amsmath, mathtools, amsfonts, amssymb}
\newcommand{\set}[1]{\{#1\}}

% define theorem-like environments
\usepackage[amsmath, thmmarks]{ntheorem}

\theoremstyle{break}
\theorempreskip{2.0\topsep}
\theorembodyfont{\song}
\theoremseparator{}
\newtheorem{problem}{题目}[subsection]
\renewcommand{\theproblem}{\arabic{problem}}
\newtheorem{ot}{Open Topics}

\theorempreskip{3.0\topsep}
\theoremheaderfont{\kai\bfseries}
\theoremseparator{:}
\theorempostwork{\bigskip\hrule}
\newtheorem*{solution}{解答}
\theorempostwork{\bigskip\hrule}
\newtheorem*{revision}{订正}

\theoremstyle{plain}
\newtheorem*{cause}{错因分析}
\newtheorem*{remark}{注}

\theoremstyle{break}
\theorempostwork{\bigskip\hrule}
\theoremsymbol{\ensuremath{\Box}}
\newtheorem*{proof}{证明}

% \newcommand{\ot}{\blue{\bf [OT]}}

% for figs
\renewcommand\figurename{图}
\renewcommand\tablename{表}

% for fig without caption: #1: width/size; #2: fig file
\newcommand{\fig}[2]{
  \begin{figure}[htbp]
    \centering
    \includegraphics[#1]{#2}
  \end{figure}
}
% for fig with caption: #1: width/size; #2: fig file; #3: caption
\newcommand{\figcap}[3]{
  \begin{figure}[htbp]
    \centering
    \includegraphics[#1]{#2}
    \caption{#3}
  \end{figure}
}
% for fig with both caption and label: #1: width/size; #2: fig file; #3: caption; #4: label
\newcommand{\figcaplbl}[4]{
  \begin{figure}[htbp]
    \centering
    \includegraphics[#1]{#2}
    \caption{#3}
    \label{#4}
  \end{figure}
}
% for margin fig without caption: #1: width/size; #2: fig file
\newcommand{\mfig}[2]{
  \begin{marginfigure}
    \centering
    \includegraphics[#1]{#2}
  \end{marginfigure}
}
% for margin fig with caption: #1: width/size; #2: fig file; #3: caption
\newcommand{\mfigcap}[3]{
  \begin{marginfigure}
    \centering
    \includegraphics[#1]{#2}
    \caption{#3}
  \end{marginfigure}
}

\usepackage{fancyvrb}

% for algorithms
\usepackage[]{algorithm}
\usepackage[]{algpseudocode} % noend
% See [Adjust the indentation whithin the algorithmicx-package when a line is broken](https://tex.stackexchange.com/a/68540/23098)
\newcommand{\algparbox}[1]{\parbox[t]{\dimexpr\linewidth-\algorithmicindent}{#1\strut}}
\newcommand{\hStatex}[0]{\vspace{5pt}}
\makeatletter
\newlength{\trianglerightwidth}
\settowidth{\trianglerightwidth}{$\triangleright$~}
\algnewcommand{\LineComment}[1]{\Statex \hskip\ALG@thistlm \(\triangleright\) #1}
\algnewcommand{\LineCommentCont}[1]{\Statex \hskip\ALG@thistlm%
  \parbox[t]{\dimexpr\linewidth-\ALG@thistlm}{\hangindent=\trianglerightwidth \hangafter=1 \strut$\triangleright$ #1\strut}}
\makeatother

% for footnote/marginnote
% see https://tex.stackexchange.com/a/133265/23098
\usepackage{tikz}
\newcommand{\circled}[1]{%
  \tikz[baseline=(char.base)]
  \node [draw, circle, inner sep = 0.5pt, font = \tiny, minimum size = 8pt] (char) {#1};
}
\renewcommand\thefootnote{\protect\circled{\arabic{footnote}}} % feel free to modify this file
%%%%%%%%%%%%%%%%%%%%
\title{第5讲: 数据结构}
\me{林凡琪}{211240042}{}{}
\date{\zhtoday} % or like 2019年9月13日
%%%%%%%%%%%%%%%%%%%%
\begin{document}
\maketitle
%%%%%%%%%%%%%%%%%%%%
\noplagiarism % always keep this line
%%%%%%%%%%%%%%%%%%%%
\begin{abstract}
  \mfig{width = 1.30\textwidth}{figs/data-structure-linus}
  \begin{center}{\fcolorbox{blue}{yellow!60}{\parbox{0.32\textwidth}{\large
          \begin{itemize}
            \item 数据之美,在于结构
            \item 掌握基础的数据结构
          \end{itemize}}}}
  \end{center}
\end{abstract}
%%%%%%%%%%%%%%%%%%%%
\beginrequired

%%%%%%%%%%%%%%%
\begin{problem}[DH 2.11: Generating Permutations]
\end{problem}

\begin{solution}
  \noindent
  \begin{algorithm}
    \caption{ permutations}\label{euclid}
    \begin{algorithmic}[1]
      \State read{$N$}
      \State solve{$1$}
      \Procedure{solve}{$k$}
      \If {$k = N + 1$}
      \For{$i \gets 1$ to $N$}
      \State print(val[$i$])
      \State print(endl)
      \EndFor
      \State return
      \EndIf
      \For{$i\gets 1$ to $N$}
      \If{vis[$i$] = $1$}
      \State continue
      \EndIf
      \State val[$k$] = $i$
      \State vis[$i$] = $1$
      \State solve($k + 1$)
      \State vis[$i$] = 0
      \EndFor
      \EndProcedure
    \end{algorithmic}
  \end{algorithm}
\end{solution}
%%%%%%%%%%%%%%%

%%%%%%%%%%%%%%%
\begin{problem}[DH 2.12 (a: III; b; c): Examples for Generating Permutations via Stack]
\end{problem}

\begin{solution}
  (a)$iii.$\\
  read($X$) 1\\
  push($X$,$S$) 1\\
  read($X$) 2\\
  push($X$,$S$) 2\\
  read($X$) 3\\
  print($X$) 3\\
  read($X$) 4\\
  push($X$,$S$) 4\\
  read($X$) 5\\
  print($X$) 5\\
  read($X$) 6\\
  push($X$,$S$) 6\\
  read($X$) 7\\
  print($X$) 7\\
  pop($X$,$S$) 6\\
  print($X$) 6\\
  read($X$) 8\\
  print($X$) 8\\
  pop($X$,$S$) 4\\
  print($X$) 4\\
  read($X$) 9\\
  print($X$) 9\\
  pop($X$,$S$) 2\\
  print($X$) 2\\
  read($X$) 10\\
  print($X$) 10\\
  pop($X$,$S$) 1\\
  print($X$) 1\\
  (b)\\
  $i.$\\
  证明:输出3后,栈顶元素应为2,1一定在2后面输出,所以不可能是出栈序列\\
  $ii.$\\
  证明:输出7时,6一定是栈顶元素,2和1一定在6的后面输出,所以不可能是出栈序列\\
  (c)\\
  $A_4 = 24$,四个元素共有14种,所以有10种不是出栈序列

\end{solution}
%%%%%%%%%%%%%%%

%%%%%%%%%%%%%%%
\begin{problem}[DH 2.13: Algorithms for Generating Permutations via Stack]
\end{problem}

\begin{solution}
  \noindent
  \begin{algorithm}
    \caption{match}\label{euclid}
    \begin{algorithmic}[2]
      \Procedure {judge}{}
      \State read {$order[]$}
      \State $n$ = size of $order[]$
      \For{$i \gets 1$ to $n$}
      \State $arr[i] = i$
      \EndFor
      \State i = 1
      \State $k$ = $1$
      \For{$j \gets 1$ to $n$}
      \State operation[k] = read
      \State k++
      \If{$arr[j]$ = order[i]}
      \State operation[k] = print
      \State k++
      \Else
      \State push(S)
      \State operation[k] = push
      \State k++
      \EndIf
      \If{top = order[i]}
      \State pop($S$)
      \State operation[k] = pop
      \State k++
      \State operation[k] = print
      \State k++
      \State i++
      \EndIf
      \EndFor
      \If{is-empty(S)}
      \State return true
      \Else return false
      \EndIf
      \EndProcedure

      \If{judge = true}
      \State print "yes"
      \For{i $\gets 1$ to $k - 1$}
      \State print operation[i]
      \EndFor
      \Else
      \State print "no"
      \EndIf
    \end{algorithmic}
  \end{algorithm}
\end{solution}
%%%%%%%%%%%%%%%

%%%%%%%%%%%%%%%
\begin{problem}[DH 2.14 (b, c): Generating Permutations via Queue]
\end{problem}

\begin{solution}
  $(b)$\\
  证明:当需要队列中的第n个数时,只需要将其前面的n-1个数一一弹出再一一入队,此时n就到了队列顶端,可以弹出并输出.\\
  $(c)$\\
  证明:当需要栈S1中的第n个数时,只需要将其后面的N-n个数一一弹出,再一一进入栈S2,此时n就到了栈顶,可以弹出并输出.\\
\end{solution}
%%%%%%%%%%%%%%%

%%%%%%%%%%%%%%%
\begin{problem}[DH 2.16: Treesort]
\end{problem}

\begin{solution}
  \noindent
  (a)\\
  以第一个数为rootT,如果一个数小于所在的T且left(T)非空,就走向left(T),将其当做新的T继续判断,如果这个数大于所在的T且right(T)非空,就走向right(T),将其当成新的T继续判断,直到碰见空T,就把该数字放进去。\\
  (b)\\
  中序遍历会输出从小到大的排列顺序\\
  按照题目的遍历方式会输出从大到小的排列顺序

\end{solution}
%%%%%%%%%%%%%%%

%%%%%%%%%%%%%%%%%%%%
\beginoptional

%%%%%%%%%%%%%%%
\begin{problem}[Generating Permutations via Stack]
How many permutations of $A_n$ can be obtained by a stack?
\end{problem}

\begin{solution}
\end{solution}
%%%%%%%%%%%%%%%

%%%%%%%%%%%%%%%%%%%%
\beginot

\mfig{width = 0.85\textwidth}{figs/boring-teachers}
{\bf 注意:} 本周的两个OT都是介绍程序设计语言的特性。
平铺直叙的介绍方式很容易让听众精神涣散。
你需要思考如何才能让大家对你以及你所讲解的知识保持兴趣。

%%%%%%%%%%%%%%%
\begin{ot}[Pointers and Arrays]
  介绍 C/C++ 语言中的指针与数组, 如 (不限于):
  \begin{itemize}
    \item 指针的基本概念
    \item 数组的声明与使用
    \item 指针与数组的关系
    \item 多维数组
  \end{itemize}

  \noindent 参考资料
  \begin{itemize}
    \item \href{http://cslabcms.nju.edu.cn/problem\_solving/images/c/cc/The\_C\_Programming\_Language\_\%282nd\_Edition\_Ritchie\_Kernighan\%29.pdf}{Chapter 5 of ``K\&R: The C Programming Language (2nd Edition)''}
    \item \href{http://cslabcms.nju.edu.cn/problem\_solving/images/5/53/Understanding\_and\_Using\_C\_Pointers\_\%28Richard\_Reese\%29.pdf}{Chapter 4 of ``Understanding and Using C Pointers''}
    \item \href{http://ptgmedia.pearsoncmg.com/images/9780321714114/samplepages/0321714113.pdf}{Sections 3.5 and 3.6 of ``C++ Primer (5th Edition)''}
  \end{itemize}
\end{ot}

% \begin{solution}
% \end{solution}
%%%%%%%%%%%%%%%
\vspace{0.50cm}
%%%%%%%%%%%%%%%
\begin{ot}[Sequential Containers in C++ STL]
  请介绍 C++ STL 中的顺序存储容器
  (\texttt{vector}, \texttt{deque}, \texttt{list}, \texttt{forward\_list}, \texttt{string}) 的用法。
  \mfig{width = 0.70\textwidth}{figs/data-structure-first}

  \noindent 要求:
  \begin{itemize}
    \item 不要引入过多的概念,报告的目的是让大家掌握这些基础容器的用法 (目前不必了解过多技术细节)
  \end{itemize}

  \noindent 参考资料
  \begin{itemize}
    \item \href{http://ptgmedia.pearsoncmg.com/images/9780321714114/samplepages/0321714113.pdf}{Chapters 3 and 9 of ``C++ Primer (5th Edition)''}
    \item \href{https://en.wikipedia.org/wiki/Sequence\_container\_(C\%2B\%2B)}{Sequence container (C++) @ wiki}
  \end{itemize}
\end{ot}

% \begin{solution}
% \end{solution}
%%%%%%%%%%%%%%%

%%%%%%%%%%%%%%%%%%%%
% 如果没有需要订正的题目,可以把这部分删掉

\begincorrection
1-4\\
三分法最优。因为三分法称量后可以使每个单份中的硬币数量最少,且一定能判断假币在哪一堆。\\
因为每操作一次后,样本容量变为原来的1/3,当样本容量为3时就只需一次称量。所以下界为  $\lceil \log_3 n \rceil$\\
3-3\\
假设形如4k+3的素数是有限个,设其为$p_1, p_2,...,p_n$\\
令$n = 4 * p_1, p_2,...,p_n - 1$\\
首先可知n形如4k+1且n>$p_n$\\
继续证明n为素数\\
因为n为奇数,所以它的因子只能为奇数,又因为任何形如4k+3的素数都不是它的因子,所以只需证明形如4k+1的素数也不是它的因子。\\
易知,如果n有形如4k+1的因子,那n也能被表达为x倍的4k+1的形式,但n实际上是4k+3的形式,所以n不可能有形如4k+1的因子。\\

%%%%%%%%%%%%%%%%%%%%

%%%%%%%%%%%%%%%%%%%%
% 如果没有反馈,可以把这部分删掉
\beginfb

% 你可以写
% ~\footnote{优先推荐 \href{problemoverflow.top}{ProblemOverflow}}:
% \begin{itemize}
%   \item 对课程及教师的建议与意见
%   \item 教材中不理解的内容
%   \item 希望深入了解的内容
%   \item $\cdots$
% \end{itemize}
%%%%%%%%%%%%%%%%%%%%
\end{document}