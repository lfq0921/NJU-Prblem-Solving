% 1-8-set.tex

%%%%%%%%%%%%%%%%%%%%
\documentclass[a4paper, justified]{tufte-handout}

% hw-preamble.tex

% geometry for A4 paper
% See https://tex.stackexchange.com/a/119912/23098
\geometry{
  left=20.0mm,
  top=20.0mm,
  bottom=20.0mm,
  textwidth=130mm, % main text block
  marginparsep=5.0mm, % gutter between main text block and margin notes
  marginparwidth=50.0mm % width of margin notes
}

% for colors
\usepackage{xcolor} % usage: \color{red}{text}
% predefined colors
\newcommand{\red}[1]{\textcolor{red}{#1}} % usage: \red{text}
\newcommand{\blue}[1]{\textcolor{blue}{#1}}
\newcommand{\teal}[1]{\textcolor{teal}{#1}}

\usepackage{todonotes}

% heading
\usepackage{sectsty}
\setcounter{secnumdepth}{2}
\allsectionsfont{\centering\huge\rmfamily}

% for Chinese
\usepackage{xeCJK}
\usepackage{zhnumber}
\setCJKmainfont[BoldFont=FandolSong-Bold.otf]{FandolSong-Regular.otf}

% for fonts
\usepackage{fontspec}
\newcommand{\song}{\CJKfamily{song}} 
\newcommand{\kai}{\CJKfamily{kai}} 

% To fix the ``MakeTextLowerCase'' bug:
% See https://github.com/Tufte-LaTeX/tufte-latex/issues/64#issuecomment-78572017
% Set up the spacing using fontspec features
\renewcommand\allcapsspacing[1]{{\addfontfeature{LetterSpace=15}#1}}
\renewcommand\smallcapsspacing[1]{{\addfontfeature{LetterSpace=10}#1}}

% for url
\usepackage{hyperref}
\hypersetup{colorlinks = true, 
  linkcolor = teal,
  urlcolor  = teal,
  citecolor = blue,
  anchorcolor = blue}

\newcommand{\me}[4]{
    \author{
      {\bfseries 姓名:}\underline{#1}\hspace{2em}
      {\bfseries 学号:}\underline{#2}\hspace{2em}\\[10pt]
      {\bfseries 评分:}\underline{#3\hspace{3em}}\hspace{2em}
      {\bfseries 评阅:}\underline{#4\hspace{3em}}
  }
}

% Please ALWAYS Keep This.
\newcommand{\noplagiarism}{
  \begin{center}
    \fbox{\begin{tabular}{@{}c@{}}
      请独立完成作业,不得抄袭。\\
      若得到他人帮助, 请致谢。\\
      若参考了其它资料,请给出引用。\\
      鼓励讨论,但需独立书写解题过程。
    \end{tabular}}
  \end{center}
}

\newcommand{\goal}[1]{
  \begin{center}{\fcolorbox{blue}{yellow!60}{\parbox{0.50\textwidth}{\large 
    \begin{itemize}
      \item 体会``思维的乐趣''
      \item 初步了解递归与数学归纳法 
      \item 初步接触算法概念与问题下界概念
    \end{itemize}}}}
  \end{center}
}

% Each hw consists of four parts:
\newcommand{\beginrequired}{\hspace{5em}\section{作业 (必做部分)}}
\newcommand{\beginoptional}{\section{作业 (选做部分)}}
\newcommand{\beginot}{\section{Open Topics}}
\newcommand{\begincorrection}{\section{订正}}
\newcommand{\beginfb}{\section{反馈}}

% for math
\usepackage{amsmath, mathtools, amsfonts, amssymb}
\newcommand{\set}[1]{\{#1\}}

% define theorem-like environments
\usepackage[amsmath, thmmarks]{ntheorem}

\theoremstyle{break}
\theorempreskip{2.0\topsep}
\theorembodyfont{\song}
\theoremseparator{}
\newtheorem{problem}{题目}[subsection]
\renewcommand{\theproblem}{\arabic{problem}}
\newtheorem{ot}{Open Topics}

\theorempreskip{3.0\topsep}
\theoremheaderfont{\kai\bfseries}
\theoremseparator{:}
\theorempostwork{\bigskip\hrule}
\newtheorem*{solution}{解答}
\theorempostwork{\bigskip\hrule}
\newtheorem*{revision}{订正}

\theoremstyle{plain}
\newtheorem*{cause}{错因分析}
\newtheorem*{remark}{注}

\theoremstyle{break}
\theorempostwork{\bigskip\hrule}
\theoremsymbol{\ensuremath{\Box}}
\newtheorem*{proof}{证明}

% \newcommand{\ot}{\blue{\bf [OT]}}

% for figs
\renewcommand\figurename{图}
\renewcommand\tablename{表}

% for fig without caption: #1: width/size; #2: fig file
\newcommand{\fig}[2]{
  \begin{figure}[htbp]
    \centering
    \includegraphics[#1]{#2}
  \end{figure}
}
% for fig with caption: #1: width/size; #2: fig file; #3: caption
\newcommand{\figcap}[3]{
  \begin{figure}[htbp]
    \centering
    \includegraphics[#1]{#2}
    \caption{#3}
  \end{figure}
}
% for fig with both caption and label: #1: width/size; #2: fig file; #3: caption; #4: label
\newcommand{\figcaplbl}[4]{
  \begin{figure}[htbp]
    \centering
    \includegraphics[#1]{#2}
    \caption{#3}
    \label{#4}
  \end{figure}
}
% for margin fig without caption: #1: width/size; #2: fig file
\newcommand{\mfig}[2]{
  \begin{marginfigure}
    \centering
    \includegraphics[#1]{#2}
  \end{marginfigure}
}
% for margin fig with caption: #1: width/size; #2: fig file; #3: caption
\newcommand{\mfigcap}[3]{
  \begin{marginfigure}
    \centering
    \includegraphics[#1]{#2}
    \caption{#3}
  \end{marginfigure}
}

\usepackage{fancyvrb}

% for algorithms
\usepackage[]{algorithm}
\usepackage[]{algpseudocode} % noend
% See [Adjust the indentation whithin the algorithmicx-package when a line is broken](https://tex.stackexchange.com/a/68540/23098)
\newcommand{\algparbox}[1]{\parbox[t]{\dimexpr\linewidth-\algorithmicindent}{#1\strut}}
\newcommand{\hStatex}[0]{\vspace{5pt}}
\makeatletter
\newlength{\trianglerightwidth}
\settowidth{\trianglerightwidth}{$\triangleright$~}
\algnewcommand{\LineComment}[1]{\Statex \hskip\ALG@thistlm \(\triangleright\) #1}
\algnewcommand{\LineCommentCont}[1]{\Statex \hskip\ALG@thistlm%
  \parbox[t]{\dimexpr\linewidth-\ALG@thistlm}{\hangindent=\trianglerightwidth \hangafter=1 \strut$\triangleright$ #1\strut}}
\makeatother

% for footnote/marginnote
% see https://tex.stackexchange.com/a/133265/23098
\usepackage{tikz}
\newcommand{\circled}[1]{%
  \tikz[baseline=(char.base)]
  \node [draw, circle, inner sep = 0.5pt, font = \tiny, minimum size = 8pt] (char) {#1};
}
\renewcommand\thefootnote{\protect\circled{\arabic{footnote}}} % feel free to modify this file
%%%%%%%%%%%%%%%%%%%%
\title{第8讲: 集合及其运算}
\me{林凡琪}{21240042}{}{}
\date{\zhtoday} % or like 2019年9月13日
%%%%%%%%%%%%%%%%%%%%
\begin{document}
\maketitle
%%%%%%%%%%%%%%%%%%%%
\noplagiarism % always keep this line
%%%%%%%%%%%%%%%%%%%%
\begin{abstract}
  \mfigcap{width = 1.00\textwidth}{figs/frege}{``左边说得在理, 我深有体会''}
  \begin{center}{\fcolorbox{blue}{yellow!60}{\parbox{0.32\textwidth}{\large
          \begin{itemize}
            \item 集合作为数学的基础
            \item 基础不牢, 地动山摇
          \end{itemize}}}}
  \end{center}
\end{abstract}
%%%%%%%%%%%%%%%%%%%%
\beginrequired

%%%%%%%%%%%%%%%
\begin{problem}[UD Problem 6.6 (f, g)]
\end{problem}

\begin{solution}
  第一张图:\\
  $\complement_X(A\cup B)$\\
  第二张图:\\
  $\complement_X(A\cup B) \cup (A\cap B) $
\end{solution}
%%%%%%%%%%%%%%%

%%%%%%%%%%%%%%%
\begin{problem}[UD Problem 7.1 (d, f)]
\end{problem}


\begin{solution}
  \noindent(d)必要性:$x\in A \Rightarrow x\in B$
  其逆否命题$x\notin B \Rightarrow x\notin A$也成立\\
  充分性:$x\notin B \Rightarrow x\notin A$其逆否命题$x\in A \Rightarrow x\in B$也成立\\
  \\
  (f)if $A \cup B = B$
  即$x\in B \Rightarrow x\in A$ \\
  即$B \subseteq A$
\end{solution}
%%%%%%%%%%%%%%%

%%%%%%%%%%%%%%%
\begin{problem}[UD Problem 7.2]
\end{problem}

\begin{proof}
  $A \cup B = \emptyset \Rightarrow x\in B$ $x \notin A$即$B \subseteq (X / A)$
\end{proof}
%%%%%%%%%%%%%%%

%%%%%%%%%%%%%%%
\begin{problem}[UD Problem 7.14]
\end{problem}
\begin{solution}
  (a)若$x \in A \backslash B$, 则$x \in A$且$x \notin B$\\
  所以不存在$x \in B$使得$x \in A \backslash B$\\
  \\
  (b)若$x \in A\cup B$ 那么 $x \in A$ 或者 $x \in B$\\
  若$x\in (A \backslash B) \cup B$ 那么 $x \in A and x\notin B$ or $x \in B$ 即$x \in A$ 或者$x \in B$\\
  由此可证$A\cup B = (A \backslash B) \cup B$

\end{solution}
%%%%%%%%%%%%%%%

%%%%%%%%%%%%%%%
\begin{problem}[UD Problem 7.19]
\end{problem}

\begin{proof}
  $A\cap (B^c \cap C^c) = \emptyset$ 即$\nexists x$符合$x \in A$并且$x \notin B \cup C$\\
  即如果$x \in A$,则$x \in B \cup C$\\
  即$A \subseteq (B \cup C)$\\
  反推:若$A \subseteq (B \cup C)$,则如果$x \in A$,则$x \in B \cup C$\\
  所以不存在x符合$x \in A$并且$x \notin B \cup C$ \\
  即$A\cap (B^c \cap C^c) = \emptyset$
\end{proof}
%%%%%%%%%%%%%%%

%%%%%%%%%%%%%%%
\begin{problem}[UD Problem 7.20]
\end{problem}

\begin{proof}
  如果$x \in (A \cup B)\backslash (C \cup D) $则$x \in A$或$x \in B$并且$x \notin C, x \notin D$\\
  如果$x\in(X \in (A \backslash (C \cup D))\cup (B \backslash (C \cup D)))$则$x \in A$ 并且$x \notin C, x \notin D$;或者$x \in B$ 并且$x \notin C, x \notin D$
  与$x \in A$或$x \in B$并且$x \notin C, x \notin D$等价
  则证得$(A \cup B)\backslash (C \cup D) = (A \backslash (C \cup D))\cup (B \backslash (C \cup D))$
\end{proof}
%%%%%%%%%%%%%%%

%%%%%%%%%%%%%%%
\begin{problem}[UD Problem 8.1 (a, b)]
\end{problem}

\begin{solution}
\end{solution}
%%%%%%%%%%%%%%%

%%%%%%%%%%%%%%%
\begin{problem}[UD Problem 8.14]
\end{problem}

\begin{solution}
  (a)[0, 1); [0, 1]; (0, 1)\\
  (b)0, 0, $\emptyset$
\end{solution}
%%%%%%%%%%%%%%%

%%%%%%%%%%%%%%%
\begin{problem}[UD Problem 8.15]
\end{problem}

\begin{solution}
  Guess:A = {2n|n $\in$ Z}\\
  proof:$R\backslash {2n}$即为所有非偶数的实数的集合\\
  $\mathbb{Q} \backslash \cap(R\backslash {2n})$即代表,集合里的元素是有理数,但不是非偶数的实数,所以集合里的元素就是偶数.
\end{solution}
%%%%%%%%%%%%%%%

%%%%%%%%%%%%%%%
\begin{problem}[UD Problem 9.8]
\end{problem}

\begin{proof}
  如果$A \subseteq B$即$x \in A $都满足$x\in B$\\
  而如果$x \in A $,则$x \in P(A)$;如果$x \in B $,则$x \in P(B)$,所以$A \subseteq B$则$P(A) \subseteq P(B)$.
\end{proof}
%%%%%%%%%%%%%%%

%%%%%%%%%%%%%%%
\begin{problem}[UD Problem 9.9]
\end{problem}

\begin{proof}
  $P(A_\alpha ) \subseteq P(\cup A_\alpha )$\\
  所以$\cup P(A_\alpha ) \subseteq P(\cup A_\alpha )$
\end{proof}
%%%%%%%%%%%%%%%

%%%%%%%%%%%%%%%
\begin{problem}[UD Problem 9.10]
\end{problem}

\begin{proof}
  若$x \subseteq P(\cap A_\alpha )$则$x \subseteq \cap A_\alpha$\\
  即$x \subseteq P(\cap A_\alpha)$\\
  \\
  若 $x \subseteq P(\cap A_\alpha)$则$x \subseteq \cap A_\alpha$所以$x \subseteq P(\cap A_\alpha )$
\end{proof}
%%%%%%%%%%%%%%%

%%%%%%%%%%%%%%%
\begin{problem}[改编自 UD Problem 9.19]
请证明:
$A \times (B \setminus C) = (A \times B) \setminus (A \times C)$

\end{problem}

\begin{proof}
  设(x, y),\\
  $A \times (B \setminus C)$ 即 $x \in A, y\in B \backslash C$,即$x \in A,y\in B,y \notin C$\\
  $(A \times B) \setminus (A \times C)$即$x \in A, y\in B$且不存在$x \in A, y\in C$,即$x \in A,y\in B,y \notin C$\\
  可知两种情况等价

\end{proof}
%%%%%%%%%%%%%%%

%%%%%%%%%%%%%%%%%%%%
\beginoptional

%%%%%%%%%%%%%%%
\begin{problem}[UD Problem 9.23]
\end{problem}

\begin{solution}
\end{solution}
%%%%%%%%%%%%%%%

%%%%%%%%%%%%%%%%%%%%
\beginot

%%%%%%%%%%%%%%%
%\begin{ot}[自然数]
%  介绍如何使用集合定义 (不限于):
%  \begin{itemize}
%    \item 自然数
%    \item 自然数上的大小关系
%    \item 自然数上的运算
%  \end{itemize}
%
%  \noindent 参考资料:
%  \begin{itemize}
%    \item \href{https://en.wikipedia.org/wiki/Natural\_number}{Natural number @ wiki}
%  \end{itemize}
%\end{ot}
%
%% \begin{solution}
%% \end{solution}
%%%%%%%%%%%%%%%%
%\vspace{0.50cm}
%%%%%%%%%%%%%%%%
%\begin{ot}[选择公理]
%  介绍选择公理 (Axiom of Choice), 如 (不限于):
%  \begin{itemize}
%    \item 不同定义形式
%    \item 怎么理解 (怎么也不理解)
%    \item 有什么用
%  \end{itemize}
%
%  \noindent 参考资料:
%  \begin{itemize}
%    \item \href{https://en.wikipedia.org/wiki/Axiom\_of\_choice}{Axiom of choice @ wiki}
%    \item \href{https://plato.stanford.edu/entries/axiom-choice/}{The Axiom of Choice @ Stanford Encyclopedia of Philosophy}
%  \end{itemize}
%\end{ot}

% \begin{solution}
% \end{solution}
%%%%%%%%%%%%%%%

%%%%%%%%%%%%%%%%%%%%
% 如果没有需要订正的题目,可以把这部分删掉

\begincorrection

%%%%%%%%%%%%%%%%%%%%

%%%%%%%%%%%%%%%%%%%%
% 如果没有反馈,可以把这部分删掉
\beginfb

% 你可以写
% ~\footnote{优先推荐 \href{problemoverflow.top}{ProblemOverflow}}:
% \begin{itemize}
%   \item 对课程及教师的建议与意见
%   \item 教材中不理解的内容
%   \item 希望深入了解的内容
%   \item $\cdots$
% \end{itemize}
%%%%%%%%%%%%%%%%%%%%
\end{document}