% 1-13-boolean-algebra.tex

%%%%%%%%%%%%%%%%%%%%
\documentclass[a4paper, justified]{tufte-handout}

% hw-preamble.tex

% geometry for A4 paper
% See https://tex.stackexchange.com/a/119912/23098
\geometry{
  left=20.0mm,
  top=20.0mm,
  bottom=20.0mm,
  textwidth=130mm, % main text block
  marginparsep=5.0mm, % gutter between main text block and margin notes
  marginparwidth=50.0mm % width of margin notes
}

% for colors
\usepackage{xcolor} % usage: \color{red}{text}
% predefined colors
\newcommand{\red}[1]{\textcolor{red}{#1}} % usage: \red{text}
\newcommand{\blue}[1]{\textcolor{blue}{#1}}
\newcommand{\teal}[1]{\textcolor{teal}{#1}}

\usepackage{todonotes}

% heading
\usepackage{sectsty}
\setcounter{secnumdepth}{2}
\allsectionsfont{\centering\huge\rmfamily}

% for Chinese
\usepackage{xeCJK}
\usepackage{zhnumber}
\setCJKmainfont[BoldFont=FandolSong-Bold.otf]{FandolSong-Regular.otf}

% for fonts
\usepackage{fontspec}
\newcommand{\song}{\CJKfamily{song}} 
\newcommand{\kai}{\CJKfamily{kai}} 

% To fix the ``MakeTextLowerCase'' bug:
% See https://github.com/Tufte-LaTeX/tufte-latex/issues/64#issuecomment-78572017
% Set up the spacing using fontspec features
\renewcommand\allcapsspacing[1]{{\addfontfeature{LetterSpace=15}#1}}
\renewcommand\smallcapsspacing[1]{{\addfontfeature{LetterSpace=10}#1}}

% for url
\usepackage{hyperref}
\hypersetup{colorlinks = true, 
  linkcolor = teal,
  urlcolor  = teal,
  citecolor = blue,
  anchorcolor = blue}

\newcommand{\me}[4]{
    \author{
      {\bfseries 姓名:}\underline{#1}\hspace{2em}
      {\bfseries 学号:}\underline{#2}\hspace{2em}\\[10pt]
      {\bfseries 评分:}\underline{#3\hspace{3em}}\hspace{2em}
      {\bfseries 评阅:}\underline{#4\hspace{3em}}
  }
}

% Please ALWAYS Keep This.
\newcommand{\noplagiarism}{
  \begin{center}
    \fbox{\begin{tabular}{@{}c@{}}
      请独立完成作业,不得抄袭。\\
      若得到他人帮助, 请致谢。\\
      若参考了其它资料,请给出引用。\\
      鼓励讨论,但需独立书写解题过程。
    \end{tabular}}
  \end{center}
}

\newcommand{\goal}[1]{
  \begin{center}{\fcolorbox{blue}{yellow!60}{\parbox{0.50\textwidth}{\large 
    \begin{itemize}
      \item 体会``思维的乐趣''
      \item 初步了解递归与数学归纳法 
      \item 初步接触算法概念与问题下界概念
    \end{itemize}}}}
  \end{center}
}

% Each hw consists of four parts:
\newcommand{\beginrequired}{\hspace{5em}\section{作业 (必做部分)}}
\newcommand{\beginoptional}{\section{作业 (选做部分)}}
\newcommand{\beginot}{\section{Open Topics}}
\newcommand{\begincorrection}{\section{订正}}
\newcommand{\beginfb}{\section{反馈}}

% for math
\usepackage{amsmath, mathtools, amsfonts, amssymb}
\newcommand{\set}[1]{\{#1\}}

% define theorem-like environments
\usepackage[amsmath, thmmarks]{ntheorem}

\theoremstyle{break}
\theorempreskip{2.0\topsep}
\theorembodyfont{\song}
\theoremseparator{}
\newtheorem{problem}{题目}[subsection]
\renewcommand{\theproblem}{\arabic{problem}}
\newtheorem{ot}{Open Topics}

\theorempreskip{3.0\topsep}
\theoremheaderfont{\kai\bfseries}
\theoremseparator{:}
\theorempostwork{\bigskip\hrule}
\newtheorem*{solution}{解答}
\theorempostwork{\bigskip\hrule}
\newtheorem*{revision}{订正}

\theoremstyle{plain}
\newtheorem*{cause}{错因分析}
\newtheorem*{remark}{注}

\theoremstyle{break}
\theorempostwork{\bigskip\hrule}
\theoremsymbol{\ensuremath{\Box}}
\newtheorem*{proof}{证明}

% \newcommand{\ot}{\blue{\bf [OT]}}

% for figs
\renewcommand\figurename{图}
\renewcommand\tablename{表}

% for fig without caption: #1: width/size; #2: fig file
\newcommand{\fig}[2]{
  \begin{figure}[htbp]
    \centering
    \includegraphics[#1]{#2}
  \end{figure}
}
% for fig with caption: #1: width/size; #2: fig file; #3: caption
\newcommand{\figcap}[3]{
  \begin{figure}[htbp]
    \centering
    \includegraphics[#1]{#2}
    \caption{#3}
  \end{figure}
}
% for fig with both caption and label: #1: width/size; #2: fig file; #3: caption; #4: label
\newcommand{\figcaplbl}[4]{
  \begin{figure}[htbp]
    \centering
    \includegraphics[#1]{#2}
    \caption{#3}
    \label{#4}
  \end{figure}
}
% for margin fig without caption: #1: width/size; #2: fig file
\newcommand{\mfig}[2]{
  \begin{marginfigure}
    \centering
    \includegraphics[#1]{#2}
  \end{marginfigure}
}
% for margin fig with caption: #1: width/size; #2: fig file; #3: caption
\newcommand{\mfigcap}[3]{
  \begin{marginfigure}
    \centering
    \includegraphics[#1]{#2}
    \caption{#3}
  \end{marginfigure}
}

\usepackage{fancyvrb}

% for algorithms
\usepackage[]{algorithm}
\usepackage[]{algpseudocode} % noend
% See [Adjust the indentation whithin the algorithmicx-package when a line is broken](https://tex.stackexchange.com/a/68540/23098)
\newcommand{\algparbox}[1]{\parbox[t]{\dimexpr\linewidth-\algorithmicindent}{#1\strut}}
\newcommand{\hStatex}[0]{\vspace{5pt}}
\makeatletter
\newlength{\trianglerightwidth}
\settowidth{\trianglerightwidth}{$\triangleright$~}
\algnewcommand{\LineComment}[1]{\Statex \hskip\ALG@thistlm \(\triangleright\) #1}
\algnewcommand{\LineCommentCont}[1]{\Statex \hskip\ALG@thistlm%
  \parbox[t]{\dimexpr\linewidth-\ALG@thistlm}{\hangindent=\trianglerightwidth \hangafter=1 \strut$\triangleright$ #1\strut}}
\makeatother

% for footnote/marginnote
% see https://tex.stackexchange.com/a/133265/23098
\usepackage{tikz}
\newcommand{\circled}[1]{%
  \tikz[baseline=(char.base)]
  \node [draw, circle, inner sep = 0.5pt, font = \tiny, minimum size = 8pt] (char) {#1};
}
\renewcommand\thefootnote{\protect\circled{\arabic{footnote}}} % feel free to modify this file
%%%%%%%%%%%%%%%%%%%%
\title{第13讲: 布尔代数}
\me{林凡琪}{{ 211240042     }}{}{}
\date{\zhtoday} % or like 2019年9月13日
%%%%%%%%%%%%%%%%%%%%
\begin{document}
\maketitle
%%%%%%%%%%%%%%%%%%%%
\noplagiarism % always keep this line
%%%%%%%%%%%%%%%%%%%%
\begin{abstract}
  \mfigcap{width = 0.85\textwidth}{figs/George-Boole}{George Boole}
  % \begin{center}{\fcolorbox{blue}{yellow!60}{\parbox{0.65\textwidth}{\large 
  %   \begin{itemize}
  %     \item 
  %   \end{itemize}}}}
  % \end{center}
\end{abstract}
%%%%%%%%%%%%%%%%%%%%
\beginrequired

%%%%%%%%%%%%%%%
\begin{problem}[Definition]
请证明: A bounded, distributive, and complemented lattice is a Boolean algebra.
\end{problem}

\begin{solution}
  代数系统<B, $\lor$ , $\land$>($\lor, \land$ 为B上二元运算), 称为布尔代数, 如果B满足以下条件:\\
  (1)运算$\lor$, $\land$满足交换律.\\
  (2)$\lor$运算对$\land$运算满足分配律,$\land$运算对$\lor$运算也满足分配律.\\
  (3)B有$\lor$运算幺元1和$\land$ 运算零元0, $\land$运算幺元和$\lor$运算零元1.\\
  (4)B中的任意元素a,都有其补元$a'$\\
  \\
  有补分配格首先是格,所以满足(1);\\
  分配格满足分配律,所以满足条件(2);\\
  因为B是有补分配格,所以每一个元素都有唯一一个补元,其中元素0, 1的补元就是对方, 所以一定满足(3);\\
  B是有补分配格, 所以不妨设a为B中任意一个元素, b和c 都是a的补元, 那么$a \land b = 0 = a \land c$, $a \lor b = 1 = a \lor c$\\
  但因为B是分配格, 所以当且仅当$b = c$时,$a \land b = a \land c$,$a \lor b = a \lor c$, 所以$a$只有唯一补元.

\end{solution}
%%%%%%%%%%%%%%%

%%%%%%%%%%%%%%%
\begin{problem}[$D_{n}$]
请证明: $D_{n}$ (定义见阅读材料 Example 15.1 (c))
是 Boolean algebra 当且仅当 $n = p_1 p_2 \cdots p_k$ (for some $k$),
这里 $p_i$ 皆为素数且互异。
\end{problem}

\begin{solution}
  不妨假设$p_i$不为素数, $p_i = p_{i - 1} \times p_{i + 1}$, 则lcm($p_{i - 1}, p_{i + 1}$) = $p_i$ $\neq$ n, 不符合分配格的条件,所以不是布尔代数.\\
  再不妨假设$p_i = p_{i + 1}$, 则一定有一个$p_{i + 2} = p_i \times p_{i + 1}$,此时$p_{i + 2}$不是素数,证明过程同上一假设.
\end{solution}
%%%%%%%%%%%%%%%

%%%%%%%%%%%%%%%
\begin{problem}[Atom]
设 $B$ 为 Boolean algebra, 对于任意元素 $a \in B$,
定义 $\textsf{Atom}(a) = \set{x \le a \mid x \text{ is an atom}}$。

\noindent 现假设 $B$ 为有穷 Boolean algebra。
请证明:
\[
  \forall a \in B: a \neq 0 \implies \textsf{Atom}(a) \neq \emptyset.
\]
\end{problem}

\begin{solution}
  因为B为有穷布尔代数,所以对B中每一元素a,均存在元素$a^{'}$,使得$a \land a^{'} = 0$ 所以$0 \leqslant a$所以$\forall a \in B: a \neq 0 \implies \textsf{Atom}(a) \neq \emptyset.$因为0一定小于a.\\
  证明:设存在原子b,使得$b \leqslant a$\\
  1)如果a是原子,则令a = b, 则$b \leqslant a$\\
  2)如果a不是原子,则必存在$a_1 \in B$使得$0 < b_1 < a$,如果$b_1$不是原子,则必存在$b_2 \in B$使得$0 < b_2 < b_1 < a$,如此下去,因为B是有穷布尔代数,上述过程经过有限步骤而最后会结束,最后得到原子$b_k$,$0 < b_k < ... <b_2 < b_1 < a$令$b_k = b$ 则$b \leqslant a$
\end{solution}
%%%%%%%%%%%%%%%

%%%%%%%%%%%%%%%
\begin{problem}[Isomorphic]
请证明: 有穷且等势的 Boolean algebras 均同构。
\end{problem}

\begin{solution}
  由Stone布尔代数的表示定理可推出有穷且等势的 Boolean algebras 均同构, 所以在此证明Stone定理即可.\\
  即证:任意有限布尔代数<B,$\lor$,$land$,->,M是所有原子构成的集合,则<B,$lor$,$land$,->与<P(M),$\cup$,∩,~>同构.\\
  证明:构造映射f:B$\rightarrow\Pi (M)$, 对于$\xi \in B$有
  $$ f(x)=\left\{
    \begin{aligned}
      \Phi                           &  & {x = 0}    \\
      \{a | a \in M, a \leqslant x\} &  & {x \neq 0} \\
    \end{aligned}
    \right.
  $$
  (1)先证明$\phi$是双射\\
  (a)先证明$\phi$是入射:只有$\xi = 0$时,才有$\phi(\xi) = \Phi$.\\
  任取$x_1,x_2\in B,x_1 \neq 0, x_2 \neq 0, 且x_1\neq x_2$\\
  $x_1 = a_1 \lor a_2 \lor ... \lor a_k$其中$a_i \leqslant x_1$  ($1 \leqslant i \leqslant k)$\\
  $x_2 = b_1 \lor b_2 \lor ... \lor b_m$其中$b_j \leqslant x_2$  ($1 \leqslant j \leqslant m)$\\
  因为每一个非0元素写成上述表达式的形式是唯一的,又因为$x_1\neq x_2$, 所以$\{a_1, a_2,...a_k\} \neq \{b_1, b_2,...,b_m\}$故$\phi(x_1) \neq f(x_2)$,f入射.\\
  (b)证明f满射:任取$M_1 \in \Pi(M)$如果$M_1$为$Phi$,则$\phi(0) = M_1$,如果$M_1 \neq Phi$,令$M_1 = \{a_1,a_2,...,a_k\}$,由$\lor$的封闭性得,必存在$\xi \in B$,使得$a_1 \lor a_2 \lor ... \lor a_k= x$显然每个$a_i \leqslant \xi$,故$\phi(\xi) = M_1$,所以$\phi$是满射的.\\
  由(1)得$\phi$是双射的.\\
  (2)证明f满足三个同构关系式.\\
  任取$x_1, x_2,\in B$因为$phi$是双射,必有$M_1, M_2 \in \Pi(M)$,使得$f(x_1) = M_1, f(x_2) = M_2$,\\
  (a)证明$f(x_1 \land x_2) = f(x_1) \cap f(x_2) = M_1 \cap M_2$\\
  先证$M_3 \subseteq M_1 \cap M_2$\\
  如果$M_3 = \Phi$显然有$M_3 \subseteq M_1 \cap M_2$\\
  如果$M_3 \neq \Phi$,任取$a \in M_3$,由$f$定义得$a \leqslant x_1 \land x_2$,又因为$x_1 \land x_2 \leqslant x_1$,$x_1 \land x_2 \leqslant x_2$,所以$a \leqslant x_1, a \leqslant x_2$由f定义得$a\in f(x_1), a\in f(x_2)$即$a \in M_1, a\in M_2$,故$a \in M_1 \cap M_2$,所以$M_3 \subseteq M_1 \cap M_2$\\
  再证$M_1 \cap M_2 \subseteq M_3$\\
  如果 $M_1 \cap M_2 = \Phi$ 显然有$M_1 \cap M_2 \subseteq M_3$\\
  如果$M_1 \cap M_2 \neq \Phi$, 任取$a \in M_1 \cap M_2$ 是满足$a \leqslant x_1, a \leqslant x_2$的原子,$a \leqslant x_1 \land x_2$由f定义得\\
  $a\in f(x_1 \land x_2)$即$a\in M_3$,所以$M_1 \cap M_2 \subseteq M_3$\\
  所以$M_1 \cap M_2 = M_3$ 即$f(x_1 \land x_2) = f(x_1) \cap f(x_2)$\\
  (b)证明$f(x_1 \lor x_2) = f(x_1) \cup f(x_2) = M_1 \cup M_2$\\
  令$f(x_1 \lor x_2) = M_4$,即证明$M_4 = M_1 \cup M_2$\\
  先证$M_4 \subseteq M_1 \cup M_2$\\
  若$M_4 = \Phi$,显然$M_4 \subseteq M_1 \cup M_2$\\
  如果$M_4 \neq Phi$,任取$a \in M_4$,由f定义得$a \leqslant x_1 \lor x_2$,则必有$a \leqslant x_1$ 或者 $a \leqslant x_2$\\
  由f定义得$a\in f(x_1)$即 $a \in M_1$ 或$a \in f(x_2)$ 即 $a \in M_2$\\
  所以$a \in M_1 \cup M_2$,则 $M_4 \subseteq M_1 \cup M_2$.\\
  再证$M_1 \cup M_2 \subseteq M_4$\\
  如果 $M_1 \cup M_2 = \Phi$,显然有$M_1 \cup M_2 \subseteq M_4$\\
  如果$M_1 \cup M_2 \neq \Phi$,任取$a \in M_1 \cup M_2$\\
  如果$a \in M_1$,则$a \leqslant x_1 \leqslant x_1 \lor x_2$, 所以$a \in f(x_1 \lor x_2)$, $a \in M_4$\\
  如果$a \in M_2$,则$a \leqslant x_2 \leqslant x_1 \lor x_2$, 所以$a \in f(x_1 \lor x_2)$, $a \in M_4$\\
  所以$M_1 \cup M_2 \subseteq M_4$\\
  综上所述 $M_4 = M_1 \cup M_2$\\
  即$f(x_1 \lor x_2) = f(x_1) \cup f(x_2)$
  (3)证明$f(\overline{x_1}) = ~f(x_1), x \in B$\\
  令$x_2 = \overline{x_1}$且$f(x_1) = M_1,f(x_2) = M_2$\\
  于是$x_1 \lor x_2 = 1, x_1 \land x_2 = 0$\\
  $\phi(x_1 \lor x_2) = M, \phi(x_1 \land x_2) = \Phi$\\
  $\phi(x_1 \lor x_2) =\phi(x_1) \cup \phi(x_2) = M_1 \cup M_2 = M$\\
  $\phi(x_1 \land x_2) =\phi(x_1) \cap \phi(x_2) = M_1 \cap M_2 = \Phi$\\
  所以 $M_2 = ~M_1$ 即\\
  由(1)(2)(3)得$f(x_1 \land x_2) = f(x_1) \cap f(x_2),\phi(x_1 \lor x_2) =\phi(x_1) \cup \phi(x_2)$
  所以<B,$\lor$,$land$,->与<P(M),$\cup$,∩,~>同构
  所以可推得有限等势布尔代数同构。
\end{solution}
%%%%%%%%%%%%%%%


%%%%%%%%%%%%%%%%%%%%
\beginoptional

%%%%%%%%%%%%%%%
\begin{problem}[Isomorphic]
是否任何 Boolean Algebra 都与某个幂集 Boolean Algebra 同构?
请证明或给出反例。
\end{problem}

\begin{solution}
\end{solution}
%%%%%%%%%%%%%%%

%%%%%%%%%%%%%%%%%%%%
\beginot

%%%%%%%%%%%%%%%
\begin{ot}[Karnaugh map]
  以三变量为例,介绍卡诺图的应用与基本原理。

  参考资料:
  \begin{itemize}
    \item \href{https://en.wikipedia.org/wiki/Karnaugh\_map}{Karnaugh map @ wiki}
    \item 课程阅读材料 Section 15.12
  \end{itemize}
\end{ot}

% \begin{solution}
% \end{solution}
%%%%%%%%%%%%%%%
\vspace{0.50cm}
%%%%%%%%%%%%%%%
\begin{ot}[Circuit Design]
  为了在液晶显示器上显示数字 $0 \sim 9$,
  我们通常设置 7 个液晶段 $a \sim g$。
  请设计数字电路,实现该显示器的功能。

  \mfig{width = 1.00\textwidth}{figs/digital}
  提示: 该电路有 4 个输入信号,7个输出信号。如右图所示。
\end{ot}

% \begin{solution}
% \end{solution}
%%%%%%%%%%%%%%%

%%%%%%%%%%%%%%%%%%%%
% 如果没有需要订正的题目,可以把这部分删掉

% \begincorrection
%%%%%%%%%%%%%%%%%%%%

%%%%%%%%%%%%%%%%%%%%
% 如果没有反馈,可以把这部分删掉
\beginfb

% 你可以写
% ~\footnote{优先推荐 \href{problemoverflow.top}{ProblemOverflow}}:
% \begin{itemize}
%   \item 对课程及教师的建议与意见
%   \item 教材中不理解的内容
%   \item 希望深入了解的内容
%   \item $\cdots$
% \end{itemize}
%%%%%%%%%%%%%%%%%%%%
\end{document}