
% 2-15-rb-tree.tex

%%%%%%%%%%%%%%%%%%%%
\documentclass[a4paper, justified]{tufte-handout}

% hw-preamble.tex

% geometry for A4 paper
% See https://tex.stackexchange.com/a/119912/23098
\geometry{
  left=20.0mm,
  top=20.0mm,
  bottom=20.0mm,
  textwidth=130mm, % main text block
  marginparsep=5.0mm, % gutter between main text block and margin notes
  marginparwidth=50.0mm % width of margin notes
}

% for colors
\usepackage{xcolor} % usage: \color{red}{text}
% predefined colors
\newcommand{\red}[1]{\textcolor{red}{#1}} % usage: \red{text}
\newcommand{\blue}[1]{\textcolor{blue}{#1}}
\newcommand{\teal}[1]{\textcolor{teal}{#1}}

\usepackage{todonotes}

% heading
\usepackage{sectsty}
\setcounter{secnumdepth}{2}
\allsectionsfont{\centering\huge\rmfamily}

% for Chinese
\usepackage{xeCJK}
\usepackage{zhnumber}
\setCJKmainfont[BoldFont=FandolSong-Bold.otf]{FandolSong-Regular.otf}

% for fonts
\usepackage{fontspec}
\newcommand{\song}{\CJKfamily{song}} 
\newcommand{\kai}{\CJKfamily{kai}} 

% To fix the ``MakeTextLowerCase'' bug:
% See https://github.com/Tufte-LaTeX/tufte-latex/issues/64#issuecomment-78572017
% Set up the spacing using fontspec features
\renewcommand\allcapsspacing[1]{{\addfontfeature{LetterSpace=15}#1}}
\renewcommand\smallcapsspacing[1]{{\addfontfeature{LetterSpace=10}#1}}

% for url
\usepackage{hyperref}
\hypersetup{colorlinks = true, 
  linkcolor = teal,
  urlcolor  = teal,
  citecolor = blue,
  anchorcolor = blue}

\newcommand{\me}[4]{
    \author{
      {\bfseries 姓名:}\underline{#1}\hspace{2em}
      {\bfseries 学号:}\underline{#2}\hspace{2em}\\[10pt]
      {\bfseries 评分:}\underline{#3\hspace{3em}}\hspace{2em}
      {\bfseries 评阅:}\underline{#4\hspace{3em}}
  }
}

% Please ALWAYS Keep This.
\newcommand{\noplagiarism}{
  \begin{center}
    \fbox{\begin{tabular}{@{}c@{}}
      请独立完成作业,不得抄袭。\\
      若得到他人帮助, 请致谢。\\
      若参考了其它资料,请给出引用。\\
      鼓励讨论,但需独立书写解题过程。
    \end{tabular}}
  \end{center}
}

\newcommand{\goal}[1]{
  \begin{center}{\fcolorbox{blue}{yellow!60}{\parbox{0.50\textwidth}{\large 
    \begin{itemize}
      \item 体会``思维的乐趣''
      \item 初步了解递归与数学归纳法 
      \item 初步接触算法概念与问题下界概念
    \end{itemize}}}}
  \end{center}
}

% Each hw consists of four parts:
\newcommand{\beginrequired}{\hspace{5em}\section{作业 (必做部分)}}
\newcommand{\beginoptional}{\section{作业 (选做部分)}}
\newcommand{\beginot}{\section{Open Topics}}
\newcommand{\begincorrection}{\section{订正}}
\newcommand{\beginfb}{\section{反馈}}

% for math
\usepackage{amsmath, mathtools, amsfonts, amssymb}
\newcommand{\set}[1]{\{#1\}}

% define theorem-like environments
\usepackage[amsmath, thmmarks]{ntheorem}

\theoremstyle{break}
\theorempreskip{2.0\topsep}
\theorembodyfont{\song}
\theoremseparator{}
\newtheorem{problem}{题目}[subsection]
\renewcommand{\theproblem}{\arabic{problem}}
\newtheorem{ot}{Open Topics}

\theorempreskip{3.0\topsep}
\theoremheaderfont{\kai\bfseries}
\theoremseparator{:}
\theorempostwork{\bigskip\hrule}
\newtheorem*{solution}{解答}
\theorempostwork{\bigskip\hrule}
\newtheorem*{revision}{订正}

\theoremstyle{plain}
\newtheorem*{cause}{错因分析}
\newtheorem*{remark}{注}

\theoremstyle{break}
\theorempostwork{\bigskip\hrule}
\theoremsymbol{\ensuremath{\Box}}
\newtheorem*{proof}{证明}

% \newcommand{\ot}{\blue{\bf [OT]}}

% for figs
\renewcommand\figurename{图}
\renewcommand\tablename{表}

% for fig without caption: #1: width/size; #2: fig file
\newcommand{\fig}[2]{
  \begin{figure}[htbp]
    \centering
    \includegraphics[#1]{#2}
  \end{figure}
}
% for fig with caption: #1: width/size; #2: fig file; #3: caption
\newcommand{\figcap}[3]{
  \begin{figure}[htbp]
    \centering
    \includegraphics[#1]{#2}
    \caption{#3}
  \end{figure}
}
% for fig with both caption and label: #1: width/size; #2: fig file; #3: caption; #4: label
\newcommand{\figcaplbl}[4]{
  \begin{figure}[htbp]
    \centering
    \includegraphics[#1]{#2}
    \caption{#3}
    \label{#4}
  \end{figure}
}
% for margin fig without caption: #1: width/size; #2: fig file
\newcommand{\mfig}[2]{
  \begin{marginfigure}
    \centering
    \includegraphics[#1]{#2}
  \end{marginfigure}
}
% for margin fig with caption: #1: width/size; #2: fig file; #3: caption
\newcommand{\mfigcap}[3]{
  \begin{marginfigure}
    \centering
    \includegraphics[#1]{#2}
    \caption{#3}
  \end{marginfigure}
}

\usepackage{fancyvrb}

% for algorithms
\usepackage[]{algorithm}
\usepackage[]{algpseudocode} % noend
% See [Adjust the indentation whithin the algorithmicx-package when a line is broken](https://tex.stackexchange.com/a/68540/23098)
\newcommand{\algparbox}[1]{\parbox[t]{\dimexpr\linewidth-\algorithmicindent}{#1\strut}}
\newcommand{\hStatex}[0]{\vspace{5pt}}
\makeatletter
\newlength{\trianglerightwidth}
\settowidth{\trianglerightwidth}{$\triangleright$~}
\algnewcommand{\LineComment}[1]{\Statex \hskip\ALG@thistlm \(\triangleright\) #1}
\algnewcommand{\LineCommentCont}[1]{\Statex \hskip\ALG@thistlm%
  \parbox[t]{\dimexpr\linewidth-\ALG@thistlm}{\hangindent=\trianglerightwidth \hangafter=1 \strut$\triangleright$ #1\strut}}
\makeatother

% for footnote/marginnote
% see https://tex.stackexchange.com/a/133265/23098
\usepackage{tikz}
\newcommand{\circled}[1]{%
  \tikz[baseline=(char.base)]
  \node [draw, circle, inner sep = 0.5pt, font = \tiny, minimum size = 8pt] (char) {#1};
}
\renewcommand\thefootnote{\protect\circled{\arabic{footnote}}} % feel free to modify this file
%%%%%%%%%%%%%%%%%%%%
\title{第4-4讲: 数论初步}
\me{林凡琪 }{211240042 }{}{}
\date{\zhtoday} % or like 2019年9月13日
%%%%%%%%%%%%%%%%%%%%
\begin{document}
\maketitle
%%%%%%%%%%%%%%%%%%%%
\noplagiarism % always keep this line
%%%%%%%%%%%%%%%%%%%%
\begin{abstract}
  % \begin{center}{\fcolorbox{blue}{yellow!60}{\parbox{0.65\textwidth}{\large 
  %   \begin{itemize}
  %     \item 
  %   \end{itemize}}}}
  % \end{center}
\end{abstract}
%%%%%%%%%%%%%%%%%%%%
\beginrequired

%%%%%%%%%%%%%%%
\begin{problem}[TJ 2-15(b,f)]
\end{problem}

\begin{solution}
  (b) 234 and 165

  $$gcd(234,165)=3$$

  $$r=12,s=-17$$

  (f)-4357 and 3754

  $$gcd(-4357,3754) = 1$$

  $$r=1463,s=1698$$
\end{solution}
%%%%%%%%%%%%%%%

%%%%%%%%%%%%%%%
\begin{problem}[TJ 2-16]
\end{problem}

\begin{proof}
  令$gcd(a,b)=t$,那么$a=k_1t,b=k_2t,k_1,k_2 \neq 0$,可知

  $$ar+bs=t(k_1r+k_2s)=1$$

  因为$k_1r+k_2s \neq 0$, 所以$t|1$

  可知$t=1$
\end{proof}
%%%%%%%%%%%%%%%

%%%%%%%%%%%%%%%
\begin{problem}[TJ 2-19]
\end{problem}

\begin{proof}
  令
  $$xy=p_1^{2k_1}p_2^{2k_2}...p_t^{2k_t},k_i \geq 0$$

  $$x=p_1^{a_1}p_2^{a_2}...p_t^{a_t},a_i \geq 0$$

  $$y=p_1^{b_1}p_2^{b_2}...p_t^{b_t},b_i \geq 0$$

  所以
  $$gcd(x,y)=p_1^{min(a_1,b_1)}p_2^{min(a_2,b_2)}...p_t^{min(a_t,b_t)}=1$$

  所以
  $$min(a_i,b_i)=0 \Rightarrow a_i=0,b_i=2k_i \text{ or } a_i =2k_i
    ,b_i =0$$

  所以x,y都是perfect squares.

\end{proof}
%%%%%%%%%%%%%%%

%%%%%%%%%%%%%%%
\begin{problem}[TJ 2-29]
\end{problem}

\begin{proof}
  反证法:

  假设有有限的质数 $p_0=5,p_1,p_2,...,p_k$可以用$6n+5$的形式表示.

  令$S=\{p_1,p_2,...,p_k\}$.

  令$P=6p_1p_2...p_k+5$

  当$P$是质数, 与假设矛盾.

  当$P=q_1q_2...q_s$(其中$q_i$是质数),显然$q_i \neq 0,2,3,4(\mod 6)$

  如果$\forall q_i,q_i=1(\mod 6).$那么,$P=q_1q_2...q_s=1(\mod 6)$,这和$P=5(\mod 6)$矛盾

  如果$\exists q_i=p_t=5(\mod 6) \in S$,那么$q_i|P\Rightarrow p_t|P \Rightarrow p_t|6p_1p_2...p_k+5 \Rightarrow p_t|5$.但是与$\forall p_t \in S, p_t >5$矛盾

  如果$\exists q_i=5.$那么$q_i|P\Rightarrow 5|6p_1p_2...p_k+5 \Rightarrow 5|6p_1p_2...p_k \Rightarrow \exists p_t \in S,5|p_t$

  但这和$p_t$是质数矛盾。

  综上得证。

\end{proof}
%%%%%%%%%%%%%%%

%%%%%%%%%%%%%%%
\begin{problem}[TJ 2-30]
\end{problem}

\begin{proof}
  反证法:

  假设有有限的质数$p_0=3,p_1,p_2,...,p_k$可以用$4n-1$的形式表示.

  令$S=\{p_1,p_2,...,p_k\}$.

  令$P=4p_1p_2...p_k+3$

  当$P$是质数, 与假设矛盾.

  当$P=q_1q_2...q_s$(其中$q_i$是质数),显然$q_i \neq 0,2(\mod 4)$

  如果$\forall q_i,q_i=1(\mod 4).$那么,$P=q_1q_2...q_s=1(\mod 4)$,这和$P=3(\mod 4)$矛盾

  如果$\exists q_i=p_t\in S$,那么$q_i|P\Rightarrow p_t|P \Rightarrow p_t|4p_1p_2...p_k-1 \Rightarrow p_t|3$.但是与$\forall p_t \in S, p_t >3$矛盾

  如果$\exists q_i=3.$那么$q_i|P\Rightarrow 3|4p_1p_2...p_k+3 \Rightarrow 3|4p_1p_2...p_k \Rightarrow \exists p_t \in S,3|p_t$

  但这和$p_t$是质数矛盾。

  综上得证。
\end{proof}
%%%%%%%%%%%%%%%

%%%%%%%%%%%%%%%
\begin{problem}[CS 2.2-2]
\end{problem}

\begin{solution}
  能保证$a$有模$m$的逆
  %No, it does not guarantee that $a$ has an inverse mod $m$.

  根据Lemma 2.8,$a$有模$m$的逆的充要条件是$a$和$m$互质。

  而在题目中$a · 133 − 2m · 277 = 1$.

  前提条件有$n\ geq 2$

  可知$$n=m \geq2,y=-544,a^{-1}=133$$

  说明 $gcd(a,m) = 1$,所以$a$有模$m$的逆。

  %A necessary and sufficient condition for $a$ to have an inverse mod $m$ is that $a$ and $m$ are coprime, i.e., their greatest common divisor is 1. In this case, $a · 133 − 2m · 277 = 1$ implies that gcd(a,m) = 1, so a has an inverse mod m. However, if we change the equation to $a · 133 − 2m · 276 = 1$, then $gcd(a,m) = 2$ and a does not have an inverse mod m.

  %If $a$ has an inverse mod $m$, it can be found using the extended Euclidean algorithm, which gives $x$ and $y$ such that $ax + my = gcd(a,m)$. Then $x$ is the inverse of $a \mod m$. For example, if $a = 7$ and $m = 26$, then using the extended Euclidean algorithm we get $x = -11$ and $y = -3$ such that $7x + 26y = -77 + -78 = -1$. Then -11 is the inverse of $7 \mod 26$.
\end{solution}
%%%%%%%%%%%%%%%

%%%%%%%%%%%%%%%
\begin{problem}[CS 2.2-4]
\end{problem}

\begin{solution}
  根据Corallary 2.16可知,

  $gcd(31,32)=1,22$ 在$Z_{31}$里有一个逆

  $gcd(10,2)=2,2$在$Z_{10}$里没有逆
\end{solution}
%%%%%%%%%%%%%%%

%%%%%%%%%%%%%%%
\begin{problem}[CS 2.2-6]
\end{problem}

\begin{solution}
  根据TH 2.15可知,two positive integers $j$ and $k$ have greatest common divisor 1 (and thus are relatively prime) if and only if there are integers $x$ and $y$ such that $jx+ky=1$

  所以$$gcd(a,m)=1$$
\end{solution}
%%%%%%%%%%%%%%%

%%%%%%%%%%%%%%%
\begin{problem}[CS 2.2-8]
\end{problem}

\begin{solution}
  According to TH 2.1, which is exactly Euclid's Division Theorem. Let $j$ be a positive integer. Then for every integer $k$, there exists unique integers $q$ and $r$ and $0\leq r < n$

  According to Lemma 2.13, if $j,k,q$ and $r$ are positive integers such that $k=jq+r$, then $$gcd(j,k)=gcd(r,j)$$

  This means that the greatest common divisor of $q$ and $k$ is equal to the greatest common divisor of $r$ and $q$.
\end{solution}
%%%%%%%%%%%%%%%

%%%%%%%%%%%%%%%
\begin{problem}[CS 2.2-16]
\end{problem}

\begin{solution}
  如果 $m<0,-m=q n+r, r=0$, 那么
  $$
    m=-q n
  $$
  令 $q^{\prime}=-q, r^{\prime}=0$.
  如果 $m<0,-m=q n+r, r>0$, 那么
  $$
    m=-q n-r=-(q+1) n+(n-r)
  $$
  令 $q^{\prime}=-(q+1), r^{\prime}=n-r$.
\end{solution}
%%%%%%%%%%%%%%%

%%%%%%%%%%%%%%%
\begin{problem}[CS 2.2-19]
\end{problem}

\begin{solution}
  $$xy=gcd(x,y)\cdot lcm(x,y)$$

  令$$x=p_1^{a_1}p_2^{a_2}...p_t^{a_t},a_i \geq 0$$

  $$y=p_1^{b_1}p_2^{b_2}...p_t^{b_t},b_i \geq 0$$

  然后
  $$gcd(x,y)=p_1^{min(a_1,b_1)}p_2^{min(a_2,b_2)}...p_t^{min(a_t,b_t)}$$
  $$lcm(x,y)=p_1^{max(a_1,b_1)}p_2^{max(a_2,b_2)}...p_t^{max(a_t,b_t)}$$

  所以
  $$
    \begin{aligned}
      x y & =p_1^{a_1+b_1} p_2^{a_2+b_2} \cdots p_t^{a_t+b_t}                                                                                                                                           \\
          & =p_1^{\min \left(a_1, b_1\right)+\max \left(a_1, b_1\right)} p_2^{\min \left(a_2, b_2\right)+\max \left(a_2, b_2\right)} \cdots p_t^{\min \left(a_t, b_t\right)+\max \left(a_t, b_t\right)} \\
          & =\operatorname{gcd}(x, y) \cdot \operatorname{lcm}(x, y)
    \end{aligned}
  $$
\end{solution}
%%%%%%%%%%%%%%%
%%%%%%%%%%%%%%%%%%%%
\beginoptional


%%%%%%%%%%%%%%%%%%%%
\beginot
%%%%%%%%%%%%%%%
\begin{ot}[Lucas定理]
  \begin{itemize}
    \item 参考资料:\href{https://brilliant.org/wiki/lucas-theorem/}{https://brilliant.org/wiki/lucas-theorem/}
  \end{itemize}
\end{ot}

% \begin{solution}
% \end{solution}
%%%%%%%%%%%%%%%

%%%%%%%%%%%%%%%
\begin{ot}[Miller-Rabin Algorithm]
\end{ot}


% \begin{solution}
% \end{solution}
%%%%%%%%%%%%%%%


% \vspace{0.50cm}
%%%%%%%%%%%%%%%
% \begin{ot}[]
% 
%   \noindent 参考资料:
%   \begin{itemize}
%     \item 
%   \end{itemize}
% \end{ot}

% \begin{solution}
% \end{solution}
%%%%%%%%%%%%%%%

%%%%%%%%%%%%%%%%%%%%
% 如果没有需要订正的题目,可以把这部分删掉

% \begincorrection
%%%%%%%%%%%%%%%%%%%%

%%%%%%%%%%%%%%%%%%%%
% 如果没有反馈,可以把这部分删掉
\beginfb

% 你可以写
% ~\footnote{优先推荐 \href{problemoverflow.top}{ProblemOverflow}}:
% \begin{itemize}
%   \item 对课程及教师的建议与意见
%   \item 教材中不理解的内容
%   \item 希望深入了解的内容
%   \item $\cdots$
% \end{itemize}
%%%%%%%%%%%%%%%%%%%%
% \bibliography{2-5-solving-recurrence}
% \bibliographystyle{plainnat}
%%%%%%%%%%%%%%%%%%%%
\end{document}