% 2-15-rb-tree.tex

%%%%%%%%%%%%%%%%%%%%
\documentclass[a4paper, justified]{tufte-handout}

% hw-preamble.tex

% geometry for A4 paper
% See https://tex.stackexchange.com/a/119912/23098
\geometry{
  left=20.0mm,
  top=20.0mm,
  bottom=20.0mm,
  textwidth=130mm, % main text block
  marginparsep=5.0mm, % gutter between main text block and margin notes
  marginparwidth=50.0mm % width of margin notes
}

% for colors
\usepackage{xcolor} % usage: \color{red}{text}
% predefined colors
\newcommand{\red}[1]{\textcolor{red}{#1}} % usage: \red{text}
\newcommand{\blue}[1]{\textcolor{blue}{#1}}
\newcommand{\teal}[1]{\textcolor{teal}{#1}}

\usepackage{todonotes}

% heading
\usepackage{sectsty}
\setcounter{secnumdepth}{2}
\allsectionsfont{\centering\huge\rmfamily}

% for Chinese
\usepackage{xeCJK}
\usepackage{zhnumber}
\setCJKmainfont[BoldFont=FandolSong-Bold.otf]{FandolSong-Regular.otf}

% for fonts
\usepackage{fontspec}
\newcommand{\song}{\CJKfamily{song}} 
\newcommand{\kai}{\CJKfamily{kai}} 

% To fix the ``MakeTextLowerCase'' bug:
% See https://github.com/Tufte-LaTeX/tufte-latex/issues/64#issuecomment-78572017
% Set up the spacing using fontspec features
\renewcommand\allcapsspacing[1]{{\addfontfeature{LetterSpace=15}#1}}
\renewcommand\smallcapsspacing[1]{{\addfontfeature{LetterSpace=10}#1}}

% for url
\usepackage{hyperref}
\hypersetup{colorlinks = true, 
  linkcolor = teal,
  urlcolor  = teal,
  citecolor = blue,
  anchorcolor = blue}

\newcommand{\me}[4]{
    \author{
      {\bfseries 姓名:}\underline{#1}\hspace{2em}
      {\bfseries 学号:}\underline{#2}\hspace{2em}\\[10pt]
      {\bfseries 评分:}\underline{#3\hspace{3em}}\hspace{2em}
      {\bfseries 评阅:}\underline{#4\hspace{3em}}
  }
}

% Please ALWAYS Keep This.
\newcommand{\noplagiarism}{
  \begin{center}
    \fbox{\begin{tabular}{@{}c@{}}
      请独立完成作业,不得抄袭。\\
      若得到他人帮助, 请致谢。\\
      若参考了其它资料,请给出引用。\\
      鼓励讨论,但需独立书写解题过程。
    \end{tabular}}
  \end{center}
}

\newcommand{\goal}[1]{
  \begin{center}{\fcolorbox{blue}{yellow!60}{\parbox{0.50\textwidth}{\large 
    \begin{itemize}
      \item 体会``思维的乐趣''
      \item 初步了解递归与数学归纳法 
      \item 初步接触算法概念与问题下界概念
    \end{itemize}}}}
  \end{center}
}

% Each hw consists of four parts:
\newcommand{\beginrequired}{\hspace{5em}\section{作业 (必做部分)}}
\newcommand{\beginoptional}{\section{作业 (选做部分)}}
\newcommand{\beginot}{\section{Open Topics}}
\newcommand{\begincorrection}{\section{订正}}
\newcommand{\beginfb}{\section{反馈}}

% for math
\usepackage{amsmath, mathtools, amsfonts, amssymb}
\newcommand{\set}[1]{\{#1\}}

% define theorem-like environments
\usepackage[amsmath, thmmarks]{ntheorem}

\theoremstyle{break}
\theorempreskip{2.0\topsep}
\theorembodyfont{\song}
\theoremseparator{}
\newtheorem{problem}{题目}[subsection]
\renewcommand{\theproblem}{\arabic{problem}}
\newtheorem{ot}{Open Topics}

\theorempreskip{3.0\topsep}
\theoremheaderfont{\kai\bfseries}
\theoremseparator{:}
\theorempostwork{\bigskip\hrule}
\newtheorem*{solution}{解答}
\theorempostwork{\bigskip\hrule}
\newtheorem*{revision}{订正}

\theoremstyle{plain}
\newtheorem*{cause}{错因分析}
\newtheorem*{remark}{注}

\theoremstyle{break}
\theorempostwork{\bigskip\hrule}
\theoremsymbol{\ensuremath{\Box}}
\newtheorem*{proof}{证明}

% \newcommand{\ot}{\blue{\bf [OT]}}

% for figs
\renewcommand\figurename{图}
\renewcommand\tablename{表}

% for fig without caption: #1: width/size; #2: fig file
\newcommand{\fig}[2]{
  \begin{figure}[htbp]
    \centering
    \includegraphics[#1]{#2}
  \end{figure}
}
% for fig with caption: #1: width/size; #2: fig file; #3: caption
\newcommand{\figcap}[3]{
  \begin{figure}[htbp]
    \centering
    \includegraphics[#1]{#2}
    \caption{#3}
  \end{figure}
}
% for fig with both caption and label: #1: width/size; #2: fig file; #3: caption; #4: label
\newcommand{\figcaplbl}[4]{
  \begin{figure}[htbp]
    \centering
    \includegraphics[#1]{#2}
    \caption{#3}
    \label{#4}
  \end{figure}
}
% for margin fig without caption: #1: width/size; #2: fig file
\newcommand{\mfig}[2]{
  \begin{marginfigure}
    \centering
    \includegraphics[#1]{#2}
  \end{marginfigure}
}
% for margin fig with caption: #1: width/size; #2: fig file; #3: caption
\newcommand{\mfigcap}[3]{
  \begin{marginfigure}
    \centering
    \includegraphics[#1]{#2}
    \caption{#3}
  \end{marginfigure}
}

\usepackage{fancyvrb}

% for algorithms
\usepackage[]{algorithm}
\usepackage[]{algpseudocode} % noend
% See [Adjust the indentation whithin the algorithmicx-package when a line is broken](https://tex.stackexchange.com/a/68540/23098)
\newcommand{\algparbox}[1]{\parbox[t]{\dimexpr\linewidth-\algorithmicindent}{#1\strut}}
\newcommand{\hStatex}[0]{\vspace{5pt}}
\makeatletter
\newlength{\trianglerightwidth}
\settowidth{\trianglerightwidth}{$\triangleright$~}
\algnewcommand{\LineComment}[1]{\Statex \hskip\ALG@thistlm \(\triangleright\) #1}
\algnewcommand{\LineCommentCont}[1]{\Statex \hskip\ALG@thistlm%
  \parbox[t]{\dimexpr\linewidth-\ALG@thistlm}{\hangindent=\trianglerightwidth \hangafter=1 \strut$\triangleright$ #1\strut}}
\makeatother

% for footnote/marginnote
% see https://tex.stackexchange.com/a/133265/23098
\usepackage{tikz}
\newcommand{\circled}[1]{%
  \tikz[baseline=(char.base)]
  \node [draw, circle, inner sep = 0.5pt, font = \tiny, minimum size = 8pt] (char) {#1};
}
\renewcommand\thefootnote{\protect\circled{\arabic{footnote}}} % feel free to modify this file
\usepackage{listings}
%%%%%%%%%%%%%%%%%%%%
\title{第4-6讲: 加密算法}
\me{林凡琪 }{211240042 }{}{}
\date{\zhtoday} % or like 2019年9月13日
%%%%%%%%%%%%%%%%%%%%
\begin{document}
\maketitle
%%%%%%%%%%%%%%%%%%%%
\noplagiarism % always keep this line
%%%%%%%%%%%%%%%%%%%%
\begin{abstract}
  % \begin{center}{\fcolorbox{blue}{yellow!60}{\parbox{0.65\textwidth}{\large 
  %   \begin{itemize}
  %     \item 
  %   \end{itemize}}}}
  % \end{center}
\end{abstract}
%%%%%%%%%%%%%%%%%%%%
\beginrequired

%%%%%%%%%%%%%%%
\begin{problem}[TJ 7-7(a,b)]
\end{problem}

\begin{solution}
  (a)$n=3551,E=629,x=31$

  $x^E \mod n=31^{629}\mod 3551 =2791$

  (b)$n=2257, E=46,x=23$

  $x^E \mod n=23^{47}\mod 2257 =769$
\end{solution}
%%%%%%%%%%%%%%%

%%%%%%%%%%%%%%%
\begin{problem}[TJ 7-9(b)]
\end{problem}

\begin{solution}
  $$y^D \mod n =34^{81} \mod 5893=2014$$
\end{solution}
%%%%%%%%%%%%%%%

%%%%%%%%%%%%%%%
\begin{problem}[TJ 7-12]
\end{problem}

\begin{solution}
  极端情况举例:
  $$
    \begin{aligned}
       & n=5 \times 11=55 \\
       & m=4 \times 10=40 \\
       & E=21             \\
       & D=21
    \end{aligned}
  $$
  所以,
  $$
    \forall X, X^E \equiv X \quad \bmod n
  $$

  $$
    X^E \equiv X \quad \bmod n \Rightarrow X\left(X^{E-1}-1\right)=0 \quad \bmod n
  $$
  因为$\operatorname{gcd}(X, n)$ and $\operatorname{gcd}\left(X^{E-1}-1, n\right)$可能有任意一个不是1.
  计算$\operatorname{gcd}(X, n)$ and $\operatorname{gcd}\left(X^{E-1}-1, n\right)$, 我们能得到$n$的因子.
\end{solution}
%%%%%%%%%%%%%%%

%%%%%%%%%%%%%%%
\begin{problem}[TC 31.7-1]
\end{problem}

\begin{solution}
  $\phi(n) = (p - 1) \cdot (q - 1) = 280$.

  $d = e^{-1} \mod \phi(n) = 187$.

  $P(M) = M^e \mod n = 254$.

  $S(C) = C^d \mod n = 254^{187} \mod n = 100$.
\end{solution}
%%%%%%%%%%%%%%%

%%%%%%%%%%%%%%%
\begin{problem}[TC 31.7-2]
\end{problem}

\begin{solution}
  $$ed \equiv 1 \mod \phi(n)$$

  $$ed - 1 = 3d - 1 = k \phi(n)$$

  如果 $p, q < n / 4$, 那么

  $$\phi(n) = n - (p + q) + 1 > n - n / 2 + 1 = n / 2 + 1 > n / 2.$$

  $kn / 2 < 3d - 1 < 3d < 3n$, then $k < 6$, then we can solve $3d - 1 = n - p - n / p + 1$.
\end{solution}
%%%%%%%%%%%%%%%

%%%%%%%%%%%%%%%
\begin{problem}[TC Problem 31-3]
\end{problem}

\begin{solution}
  a. 为了解决 $\text{FIB}(n)$,我们需要计算 $\text{FIB}(n - 1)$ 和 $\text{FIB}(n - 2)$。因此,我们有递归式

  $$T(n) = T(n - 1) + T(n - 2) + \Theta(1).$$

  我们可以得到斐波那契数列的上界为 $O(2^n)$,但这不是紧密的上界。

  斐波那契递推式定义为

  $$F(n) = F(n - 1) + F(n - 2).$$

  这个函数的特征方程将是

  $$ \begin{aligned} x^2 & = x + 1 \ x^2 - x - 1 & = 0. \end{aligned} $$

  我们可以通过二次公式得到根:$x = \frac{1 \pm \sqrt 5}{2}$。

  我们知道线性递归函数的解为

  $$ \begin{aligned} F(n) & = \alpha_1^n + \alpha_2^n \ & = \bigg(\frac{1 + \sqrt 5}{2}\bigg)^n + \bigg(\frac{1 - \sqrt 5}{2}\bigg)^n, \end{aligned} $$

  其中 $\alpha_1$ 和 $\alpha_2$ 是特征方程的根。

  由于 $T(n)$ 和 $F(n)$ 都表示同一件事情,它们在渐近意义下是相同的。

  因此,

  $$ \begin{aligned} T(n) & = \bigg(\frac{1 + \sqrt 5}{2}\bigg)^n + \bigg(\frac{1 - \sqrt 5}{2}\bigg)^n \ & = \bigg(\frac{1 + \sqrt 5}{2}\bigg)^n \ & \approx O(1.618)^n. \end{aligned} $$

  \lstdefinestyle{style1}{
    basicstyle=\ttfamily,
    breaklines=true,
    numbers=left,
    keywordstyle=\color{purple}\bfseries,
    identifierstyle=\color{brown!80!black},
    commentstyle=\color{gray},
    showstringspaces=false,
    frame=trBL,
    frameround=fftt,
    backgroundcolor=\color[RGB]{245,245,244},
  }

  b.
  \begin{lstlisting}[language=C++,style=style1]
    FIBONACCI(n)
    let fib[0..n] be a new array
    fib[0] = 1
    fib[1] = 1
    for i = 2 to n
        fib[i] = fib[i - 1] + fib[i - 2]
    return fib[n]
  \end{lstlisting}

  c. 假设所有整数乘法和加法都可以在 $O(1)$ 的时间内完成。首先,我们要证明

  $$ \begin{pmatrix} 0 & 1 \ 1 & 1 \end{pmatrix}^k = \begin{pmatrix} F_{k - 1} & F_k \ F_k & F_{k + 1} \end{pmatrix} . $$

  通过归纳,

  $$ \begin{aligned} \begin{pmatrix} 0 & 1 \ 1 & 1 \end{pmatrix}^{k + 1} & = \begin{pmatrix} 0 & 1 \ 1 & 1 \end{pmatrix} \begin{pmatrix} 0 & 1 \ 1 & 1 \end{pmatrix}^k \ & = \begin{pmatrix} 0 & 1 \ 1 & 1 \end{pmatrix} \begin{pmatrix} F_{k - 1} & F_k \ F_k & F_{k + 1} \end{pmatrix}^k \ & = \begin{pmatrix} F_k & F_{k + 1} \ F_{k - 1} + F_k & F_k + F_{k + 1} \end{pmatrix} \ & = \begin{pmatrix} F_k & F_{k + 1} \ F_{k + 1} & F_{k + 2} \end{pmatrix}. \end{aligned} $$

  我们证明我们可以在$O(\lg n)$时间内计算给定矩阵的$n-1$次幂,右下角的元素是$F_n$。

  我们应该注意,通过8次乘法和4次加法,我们可以将任何两个$2\times2$矩阵相乘,这意味着矩阵乘法可以在常数时间内完成。因此,我们只需要限制算法中这些操作的数量。

  运行算法$\text{MATRIX-POW}(A,n-1)$需要$O(\lg n)$时间,因为我们在每个步骤中将$n$的值减半,并且在每个步骤中,我们执行恒定数量的计算。

  递归式为

  $$T(n) = T(n / 2) + \Theta(1).$$

  \begin{lstlisting}[language=C++,style=style1]
    MATRIX-POW(A, n)
    if n % 2 == 1
        return A * MATRIX-POW(A^2, (n - 1) / 2)
    return MATRIX-POW(A^2, n / 2)

\end{lstlisting}

  d. 首先,我们应该注意到 $\beta = O(\log n)$。对于第(a)部分,我们朴素地添加一个每次都在指数级增长的 $\beta$ 位数,因此递归变为

  $$ \begin{aligned} T(n) & = T(n - 1) + T(n - 2) + \Theta(\beta) \\ & = T(n - 1) + T(n - 2) + \Theta(\log n), \end{aligned} $$

  因为 $\Theta(\log n)$ 可以被指数时间吸收,所以它的解与 $T(n) = O\Big(\frac{1 + \sqrt 5}{2}\Big)^n$相同,。

  对于第(b)部分,记忆化版本的递归变为

  $$M(n) = M(n - 1) + \Theta(\beta).$$

  这有一个解 $\sum_{i = 2}^n \beta = \Theta(n\beta) = \Theta(2^\beta \cdot \beta)$ or $\Theta(n \log n)$.

  对于第(c)部分,我们执行恒定数量的加法和乘法。递归变为

  $$P(n) = P(n / 2) + \Theta(\beta^2),$$

  它的解为 $\Theta(\log n \cdot \beta^2) = \Theta(\beta^3)$ or $\Theta(\log^3 n)$。
\end{solution}
%%%%%%%%%%%%%%%

%%%%%%%%%%%%%%%%%%%%
\beginoptional

%%%%%%%%%%%%%%%
\begin{problem}[TC Problem 31-4]
\end{problem}

\begin{solution}
\end{solution}
%%%%%%%%%%%%%%%


%%%%%%%%%%%%%%%%%%%%
\beginot
%%%%%%%%%%%%%%%
\begin{ot}[中国剩余定理]
  向同学介绍中国剩余定理及其应用。
\end{ot}

% \begin{solution}
% \end{solution}
%%%%%%%%%%%%%%%

%%%%%%%%%%%%%%%
\begin{ot}[椭圆曲线加密(Elliptic Curve Cryptography, ECC)]
  椭圆曲线加密是基于椭圆曲线数学理论实现的一种非对称加密算法。相比RSA,ECC优势是可以使用更短的密钥,来实现与RSA相当或更高的安全。

  (参考资料-1:\href{https://medium.com/dev-genius/introduction-to-elliptic-curve-cryptography-567e47b0e49e}{https://medium.com/dev-genius/introduction-to-elliptic-curve-cryptography-567e47b0e49e})
  (参考资料-2:\href{https://en.wikipedia.org/wiki/Elliptic-curve_cryptography}{https://en.wikipedia.org/wiki/Elliptic-curve\_cryptography})
  (参考资料-3:\href{https://www.jianshu.com/p/e41bc1eb1d81}{https://www.jianshu.com/p/e41bc1eb1d81})
\end{ot}


% \begin{solution}
% \end{solution}
%%%%%%%%%%%%%%%


% \vspace{0.50cm}
%%%%%%%%%%%%%%%
% \begin{ot}[]
% 
%   \noindent 参考资料:
%   \begin{itemize}
%     \item 
%   \end{itemize}
% \end{ot}

% \begin{solution}
% \end{solution}
%%%%%%%%%%%%%%%

%%%%%%%%%%%%%%%%%%%%
% 如果没有需要订正的题目,可以把这部分删掉

% \begincorrection
%%%%%%%%%%%%%%%%%%%%

%%%%%%%%%%%%%%%%%%%%
% 如果没有反馈,可以把这部分删掉
\beginfb

% 你可以写
% ~\footnote{优先推荐 \href{problemoverflow.top}{ProblemOverflow}}:
% \begin{itemize}
%   \item 对课程及教师的建议与意见
%   \item 教材中不理解的内容
%   \item 希望深入了解的内容
%   \item $\cdots$
% \end{itemize}
%%%%%%%%%%%%%%%%%%%%
% \bibliography{2-5-solving-recurrence}
% \bibliographystyle{plainnat}
%%%%%%%%%%%%%%%%%%%%
\end{document}