% 2-15-rb-tree.tex

%%%%%%%%%%%%%%%%%%%%
\documentclass[a4paper, justified]{tufte-handout}

% hw-preamble.tex

% geometry for A4 paper
% See https://tex.stackexchange.com/a/119912/23098
\geometry{
  left=20.0mm,
  top=20.0mm,
  bottom=20.0mm,
  textwidth=130mm, % main text block
  marginparsep=5.0mm, % gutter between main text block and margin notes
  marginparwidth=50.0mm % width of margin notes
}

% for colors
\usepackage{xcolor} % usage: \color{red}{text}
% predefined colors
\newcommand{\red}[1]{\textcolor{red}{#1}} % usage: \red{text}
\newcommand{\blue}[1]{\textcolor{blue}{#1}}
\newcommand{\teal}[1]{\textcolor{teal}{#1}}

\usepackage{todonotes}

% heading
\usepackage{sectsty}
\setcounter{secnumdepth}{2}
\allsectionsfont{\centering\huge\rmfamily}

% for Chinese
\usepackage{xeCJK}
\usepackage{zhnumber}
\setCJKmainfont[BoldFont=FandolSong-Bold.otf]{FandolSong-Regular.otf}

% for fonts
\usepackage{fontspec}
\newcommand{\song}{\CJKfamily{song}} 
\newcommand{\kai}{\CJKfamily{kai}} 

% To fix the ``MakeTextLowerCase'' bug:
% See https://github.com/Tufte-LaTeX/tufte-latex/issues/64#issuecomment-78572017
% Set up the spacing using fontspec features
\renewcommand\allcapsspacing[1]{{\addfontfeature{LetterSpace=15}#1}}
\renewcommand\smallcapsspacing[1]{{\addfontfeature{LetterSpace=10}#1}}

% for url
\usepackage{hyperref}
\hypersetup{colorlinks = true, 
  linkcolor = teal,
  urlcolor  = teal,
  citecolor = blue,
  anchorcolor = blue}

\newcommand{\me}[4]{
    \author{
      {\bfseries 姓名:}\underline{#1}\hspace{2em}
      {\bfseries 学号:}\underline{#2}\hspace{2em}\\[10pt]
      {\bfseries 评分:}\underline{#3\hspace{3em}}\hspace{2em}
      {\bfseries 评阅:}\underline{#4\hspace{3em}}
  }
}

% Please ALWAYS Keep This.
\newcommand{\noplagiarism}{
  \begin{center}
    \fbox{\begin{tabular}{@{}c@{}}
      请独立完成作业,不得抄袭。\\
      若得到他人帮助, 请致谢。\\
      若参考了其它资料,请给出引用。\\
      鼓励讨论,但需独立书写解题过程。
    \end{tabular}}
  \end{center}
}

\newcommand{\goal}[1]{
  \begin{center}{\fcolorbox{blue}{yellow!60}{\parbox{0.50\textwidth}{\large 
    \begin{itemize}
      \item 体会``思维的乐趣''
      \item 初步了解递归与数学归纳法 
      \item 初步接触算法概念与问题下界概念
    \end{itemize}}}}
  \end{center}
}

% Each hw consists of four parts:
\newcommand{\beginrequired}{\hspace{5em}\section{作业 (必做部分)}}
\newcommand{\beginoptional}{\section{作业 (选做部分)}}
\newcommand{\beginot}{\section{Open Topics}}
\newcommand{\begincorrection}{\section{订正}}
\newcommand{\beginfb}{\section{反馈}}

% for math
\usepackage{amsmath, mathtools, amsfonts, amssymb}
\newcommand{\set}[1]{\{#1\}}

% define theorem-like environments
\usepackage[amsmath, thmmarks]{ntheorem}

\theoremstyle{break}
\theorempreskip{2.0\topsep}
\theorembodyfont{\song}
\theoremseparator{}
\newtheorem{problem}{题目}[subsection]
\renewcommand{\theproblem}{\arabic{problem}}
\newtheorem{ot}{Open Topics}

\theorempreskip{3.0\topsep}
\theoremheaderfont{\kai\bfseries}
\theoremseparator{:}
\theorempostwork{\bigskip\hrule}
\newtheorem*{solution}{解答}
\theorempostwork{\bigskip\hrule}
\newtheorem*{revision}{订正}

\theoremstyle{plain}
\newtheorem*{cause}{错因分析}
\newtheorem*{remark}{注}

\theoremstyle{break}
\theorempostwork{\bigskip\hrule}
\theoremsymbol{\ensuremath{\Box}}
\newtheorem*{proof}{证明}

% \newcommand{\ot}{\blue{\bf [OT]}}

% for figs
\renewcommand\figurename{图}
\renewcommand\tablename{表}

% for fig without caption: #1: width/size; #2: fig file
\newcommand{\fig}[2]{
  \begin{figure}[htbp]
    \centering
    \includegraphics[#1]{#2}
  \end{figure}
}
% for fig with caption: #1: width/size; #2: fig file; #3: caption
\newcommand{\figcap}[3]{
  \begin{figure}[htbp]
    \centering
    \includegraphics[#1]{#2}
    \caption{#3}
  \end{figure}
}
% for fig with both caption and label: #1: width/size; #2: fig file; #3: caption; #4: label
\newcommand{\figcaplbl}[4]{
  \begin{figure}[htbp]
    \centering
    \includegraphics[#1]{#2}
    \caption{#3}
    \label{#4}
  \end{figure}
}
% for margin fig without caption: #1: width/size; #2: fig file
\newcommand{\mfig}[2]{
  \begin{marginfigure}
    \centering
    \includegraphics[#1]{#2}
  \end{marginfigure}
}
% for margin fig with caption: #1: width/size; #2: fig file; #3: caption
\newcommand{\mfigcap}[3]{
  \begin{marginfigure}
    \centering
    \includegraphics[#1]{#2}
    \caption{#3}
  \end{marginfigure}
}

\usepackage{fancyvrb}

% for algorithms
\usepackage[]{algorithm}
\usepackage[]{algpseudocode} % noend
% See [Adjust the indentation whithin the algorithmicx-package when a line is broken](https://tex.stackexchange.com/a/68540/23098)
\newcommand{\algparbox}[1]{\parbox[t]{\dimexpr\linewidth-\algorithmicindent}{#1\strut}}
\newcommand{\hStatex}[0]{\vspace{5pt}}
\makeatletter
\newlength{\trianglerightwidth}
\settowidth{\trianglerightwidth}{$\triangleright$~}
\algnewcommand{\LineComment}[1]{\Statex \hskip\ALG@thistlm \(\triangleright\) #1}
\algnewcommand{\LineCommentCont}[1]{\Statex \hskip\ALG@thistlm%
  \parbox[t]{\dimexpr\linewidth-\ALG@thistlm}{\hangindent=\trianglerightwidth \hangafter=1 \strut$\triangleright$ #1\strut}}
\makeatother

% for footnote/marginnote
% see https://tex.stackexchange.com/a/133265/23098
\usepackage{tikz}
\newcommand{\circled}[1]{%
  \tikz[baseline=(char.base)]
  \node [draw, circle, inner sep = 0.5pt, font = \tiny, minimum size = 8pt] (char) {#1};
}
\renewcommand\thefootnote{\protect\circled{\arabic{footnote}}} % feel free to modify this file
%%%%%%%%%%%%%%%%%%%%
\title{第4-2讲: 置换群与拉格朗日定理}
\me{林凡琪 }{211240042 }{}{}
\date{\zhtoday} % or like 2019年9月13日
%%%%%%%%%%%%%%%%%%%%
\begin{document}
\maketitle
%%%%%%%%%%%%%%%%%%%%
\noplagiarism % always keep this line
%%%%%%%%%%%%%%%%%%%%
\begin{abstract}
	% \begin{center}{\fcolorbox{blue}{yellow!60}{\parbox{0.65\textwidth}{\large 
	%   \begin{itemize}
	%     \item 
	%   \end{itemize}}}}
	% \end{center}
\end{abstract}
%%%%%%%%%%%%%%%%%%%%
\beginrequired

%%%%%%%%%%%%%%%
\begin{problem}[TJ 5-3(d)]
\end{problem}

\begin{solution}
	(14)(15)(12)(17)(13)(12)(14)(12)(13)(16)(14)(15)\\
	even
\end{solution}
%%%%%%%%%%%%%%%

%%%%%%%%%%%%%%%
\begin{problem}[TJ 5-5 (注: 只需列出S4的所有子群, 无需解(a)、(b)、(c))]
\end{problem}

\begin{solution}
	$S_4$的子群有\\
	$N_1=\{(1)\}$\\
	二阶循环群$N_2=\{(1),(12)\}$\\
	二阶循环群$N_3=\{(1),(13)\}$\\
	二阶循环群$N_4=\{(1),(23)\}$\\
	二阶循环群$N_5=\{(1),(24)\}$\\
	二阶循环群$N_6=\{(1),(14)\}$\\
	二阶循环群$N_7=\{(1),(34)\}$\\
	二阶循环群$N_8=\{(1),(12),(34)\}$\\
	二阶循环群$N_9=\{(1),(13),(24)\}$\\
	二阶循环群$N_{10}=\{(1),(14),(23)\}$\\
	三阶循环群$N_{11}=\{(1),(123),(132)\}$\\
	三阶循环群$N_{12}=\{(1),(134),(143)\}$\\
	三阶循环群$N_{13}=\{(1),(124),(142)\}$\\
	三阶循环群$N_{14}=\{(1),(234),(243)\}$\\
	Klein四元群$N_{15}=\{(1),(12),(34),(12)(34)\}$\\
	Klein四元群$N_{16}=\{(1),(13),(24),(13)(24)\}$\\
	Klein四元群$N_{17}=\{(1),(14),(23),(14)(23)\}$\\
	Klein四元群$N_{18}=\{(1),(14)(23),(13)(24),(14)(23)\}$\\
	四阶循环群$N_{19}=\{(1),(1234),(13)(24),(1432)\}$\\
	四阶循环群$N_{20}=\{(1),(1324),(12)(34),(1423)\}$\\
	四阶循环群$N_{21}=\{(1),(1243),(14)(23),(1342)\}$\\
	与$S_3$同构$N_{22}=\{(1),(12),(13),(23),(123),(132)\}$\\
	与$S_3$同构$N_{23}=\{(1),(12),(24),(14),(124),(142)\}$\\
	与$S_3$同构$N_{24}=\{(1),(34),(13),(14),(143),(134)\}$\\
	与$S_3$同构$N_{25}=\{(1),(34),(24),(23),(234),(243)\}$\\
	八阶子群$N_{26}=\{(1),(1234),(13)(24),(1432),(13),(12)(34),(24),(14)(23)\}$\\
	八阶子群$N_{27}=\{(1),(1324),(12)(34),(1423),(12),(13)(24),(34),(14)(32)\}$\\
	八阶子群$N_{28}=\{(1),(1243),(14)(23),(1342),(14),(12)(43),(23),(13)(24)\}$\\
	$N_{29}=\{(1),(123),(132),(134),(143),(124),(142),(234)(243),(12)(34),(13)(24),(14)(23)\}$\\
	十二阶子群$N_{30}=S_4$
\end{solution}
%%%%%%%%%%%%%%%

%%%%%%%%%%%%%%%
\begin{problem}[TJ 5-16]
Find the group of rigid motions of a tetrahedron. Show that this is the same group as $A4$.
\end{problem}

\begin{solution}
	设G为rigid motions的群. 把四面体的顶点标号为1,2,3,4. Rotation由结点1发送的位置(四种可能性)和从该顶点发出边的方向(三种可能性)决定.\\
	所以在G里有12个元素.\\
	通过将给定的旋转映射到它在顶点上诱导的permutation,定义一个从 G 到顶点上的symmetric group的映射$\phi$。\\
	有8个3阶旋转, 固定单个顶点并围绕连接该顶点到对面中心的轴旋转。\\
	$\phi$下的rotation images:$\{(123), (132), (124), (142), (134), (143), (234), (243)\}$.\\
	有3个2阶旋转, 绕着中点和对边的轴旋转.\\
	$\phi$下的rotation images:$\{(12)(34), (13)(24), (14)(23)\}$\\
	再加上identity, 就是所有的12个rotation.\\
	$\phi$的image是$A_4$, 它是单射的,并且保留了群运算(因为运算再这两种情况下都是function composition),所以$\phi$给出了$A_4$和四面体rigid motion的群的同构.
\end{solution}
%%%%%%%%%%%%%%%

%%%%%%%%%%%%%%%
\begin{problem}[TJ 5-26(b)]
\end{problem}

\begin{solution}
	我们通过数学归纳法证明$(1k)$能被写成$(12),(23),...,(n-1,n)$其中$k=2,3...,n$\\
	Base Case:(12)=(12)\\
	Induction Step: (1,k+1)=(1k)(k,k+1)(1k).\\
	由(a)可知,$(12),(13),...,(1n)$能生成$S_n$,所以$(12),(23),...,(n-1,n)$也能.
	(关于(a)的证明:\\
	$S_n$的每一个元素都能写成transpositions的乘积, 并且任意transposition(ab)能被写成(1a)(1b)(1a).所以$(12),(13),...,(1n)$能生成$S_n$)
\end{solution}
%%%%%%%%%%%%%%%

%%%%%%%%%%%%%%%
\begin{problem}[TJ 5-29]
\end{problem}

\begin{solution}
	设$r$是$D_n$中$2\pi / n$的rotation. 那么$D_n$中的rotations是$\{r,r^2,...,r^{n-1},r^n=id\}$并且对于所有的$m,n$, 都有$r^mr^n=r^{m+n}=r^{n+m}=r^nr^m$\\
	设$s$是$D_n$中古丁了顶点1的reflection,那么$D_n$中的reflection是$\{s,sr,sr^2,...,sr^{n-1}\}$\\
	因为$$r^i(sr^j)=r^i(sr^jss)=r^i(srs)^js=r^i(r^{-1})^js=r^ir^{-j}s=r^{i-j}s$$\\
	并且$$(sr^j)r^i=sr^{i+j}ss=r^{-i-j}s$$\\
	所以$r^i(sr^j)=(sr^j)r^i$ iff $i-j\equiv -i-j(\mod n)$ iff $2i \equiv 0(\mod n)$.\\
	所以$Z(D_{2n})=\{id,r^n\}$并且$Z(D_{2n+1})=\{id\}$.\\
	特别地,$Z(Ds)=\{id,r^4\}$并且$Z(D_{10})=\{id,r^5\}$\\

	致谢:https://math.stackexchange.com/questions/1714723/find-the-center-of-d8-d10-and-dn
\end{solution}
%%%%%%%%%%%%%%%

%%%%%%%%%%%%%%%
\begin{problem}[TJ 5-36]
Let r and s be the elements in $D_n$ described in Theorem 5.23
(a) Show that $srs = r^{-1}$.
(b) Show that $r^ks = sr^{-k} in D_n$.
(c) Prove that the order of $r^k \in D_n$ is $n/ gcd(k, n)$.
\end{problem}

\begin{solution}
	(a)证明:

	$$
		r=\left[\begin{array}{cc}
				\cos (\pi / n) & -\sin (\pi / n) \\
				\sin (\pi / n) & \cos (\pi / n)
			\end{array}\right], s=\left[\begin{array}{cc}
				1 & 0  \\
				0 & -1
			\end{array}\right]
	$$
	Then,
	$$
		s r s=\left[\begin{array}{cc}
				\cos (\pi / n)  & \sin (\pi / n) \\
				-\sin (\pi / n) & \cos (\pi / n)
			\end{array}\right]=r^{-1}
	$$\\
	(b)证明:

	数学归纳法

	B:由(a)可知, $k=1,rs=sr^{-1}$;

	H:假设对于所有$k<j$全部r成立

	I:$k=j$

	$$r^js=r^{j-1}rs=r^{j-1}sr^{-1}=sr^{-(j-1)}r^{-1}=sr^{-j}$$
	\\
	(c)证明:

	$\left\langle r \right\rangle$是n阶的cyclic group, 所以根据TH 4.13, 显然证得.
\end{solution}
%%%%%%%%%%%%%%%

%%%%%%%%%%%%%%%
\begin{problem}[TJ 6-11 (注意:(c)中 $\subset$表示$\subseteq$)]

\end{problem}

\begin{solution}
	1.(a)$\Rightarrow$ (c)

	因为$g_1H = g_2H$,所以 $g_1H \subseteq  g_2H$\\
	2.(c)$\Rightarrow$ (d)

	因为$g_1\in g_1H \subseteq g_2H$, 有$h\in H$满足$g_1=h_2h$,然后$g_2=g_1h^{-1}\in g_1H(h^{-1}\in H)$\\
	3.(d)$\Rightarrow$(e)

	因为对于某些$h \in H$, 有$g_2=g_1h$, 所以$g_1^{-1}g_2=h\in H$.\\
	4.(e)$\Rightarrow$(b)

	设 $g_1^{-1} g_2=h \in H$. 我们首先证明$H g_1^{-1} \subseteq H g_2^{-1}$. 假设 $h^{\prime} g_1^{-1} \in H g_1^{-1}$. 因为 $g_1^{-1}=h g_2^{-1}$, 我们有 $h^{\prime} g_1^{-1}=h^{\prime} h g_2^{-1} \in H g_2^{-1}$. 相反地, $h^{\prime \prime} g_2^{-1}=h^{\prime \prime} h^{-1} g_1^{-1} \in H g_1^{-1}$, 所以 $H g_2^{-1} \subseteq H g_1^{-1}$.\\
	5.(b) $\Rightarrow$ (a)

	因为 $H g_1^{-1}=H g_2^{-1}$,存在$h \in H$满足$g_1^{-1}=h g_2^{-1}$, 所以 $g_1^{-1} g_2=h$ 并且$g_2=g_1 h$. 所以对于任意 $h^{\prime} \in H, g_2 h^{\prime}=g_1 h h^{\prime}$, so $g_2 H \subseteq g_1 H$. 相似地, $g_1 h^{\prime}=g_2 h^{-1} h^{\prime}$, so $g_1 H \subseteq g_2 H$.
\end{solution}
%%%%%%%%%%%%%%%


%%%%%%%%%%%%%%%%%%%%
\beginoptional

%%%%%%%%%%%%%%%
\begin{problem}[$3-cycle$]
证明:$A_n$ 中的每个置换皆可表成形如 $(k \text{ } k+1 \text{ } k+2)$ 的 3-cycle 的乘积。
\end{problem}

\begin{solution}
\end{solution}
%%%%%%%%%%%%%%%

%%%%%%%%%%%%%%%
\begin{problem}[SageMath学习]
学习 TJ 第五章, 第六章关于 SageMath 的内容
\end{problem}

\begin{solution}
\end{solution}
%%%%%%%%%%%%%%%

%%%%%%%%%%%%%%%%%%%%
\beginot
%%%%%%%%%%%%%%%
\begin{ot}[二阶魔方]
	请构造出二阶魔方相关的置换群,你能设计一种算法来解二阶魔方复原吗?
\end{ot}

% \begin{solution}
% \end{solution}
%%%%%%%%%%%%%%%

%%%%%%%%%%%%%%%
\begin{ot}[transpositions]
	\begin{itemize}
		\item 	证明:Show that any cycle can be written as the product of transpositions:
		      $$(a_1,a_2,…,a_n)=(a_1 a_n)(a_1 a_{n−1} )⋯(a_1 a_3)(a_1 a_2)$$
		\item 你能很快地给出一个置换的逆是什么吗?
	\end{itemize}


\end{ot}


% \begin{solution}
% \end{solution}
%%%%%%%%%%%%%%%


% \vspace{0.50cm}
%%%%%%%%%%%%%%%
% \begin{ot}[]
% 
%   \noindent 参考资料:
%   \begin{itemize}
%     \item 
%   \end{itemize}
% \end{ot}

% \begin{solution}
% \end{solution}
%%%%%%%%%%%%%%%

%%%%%%%%%%%%%%%%%%%%
% 如果没有需要订正的题目,可以把这部分删掉

% \begincorrection
%%%%%%%%%%%%%%%%%%%%

%%%%%%%%%%%%%%%%%%%%
% 如果没有反馈,可以把这部分删掉
\beginfb

% 你可以写
% ~\footnote{优先推荐 \href{problemoverflow.top}{ProblemOverflow}}:
% \begin{itemize}
%   \item 对课程及教师的建议与意见
%   \item 教材中不理解的内容
%   \item 希望深入了解的内容
%   \item $\cdots$
% \end{itemize}
%%%%%%%%%%%%%%%%%%%%
% \bibliography{2-5-solving-recurrence}
% \bibliographystyle{plainnat}
%%%%%%%%%%%%%%%%%%%%
\end{document}