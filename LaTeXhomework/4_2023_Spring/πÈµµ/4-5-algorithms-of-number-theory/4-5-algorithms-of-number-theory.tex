% 2-15-rb-tree.tex

%%%%%%%%%%%%%%%%%%%%
\documentclass[a4paper, justified]{tufte-handout}

% hw-preamble.tex

% geometry for A4 paper
% See https://tex.stackexchange.com/a/119912/23098
\geometry{
  left=20.0mm,
  top=20.0mm,
  bottom=20.0mm,
  textwidth=130mm, % main text block
  marginparsep=5.0mm, % gutter between main text block and margin notes
  marginparwidth=50.0mm % width of margin notes
}

% for colors
\usepackage{xcolor} % usage: \color{red}{text}
% predefined colors
\newcommand{\red}[1]{\textcolor{red}{#1}} % usage: \red{text}
\newcommand{\blue}[1]{\textcolor{blue}{#1}}
\newcommand{\teal}[1]{\textcolor{teal}{#1}}

\usepackage{todonotes}

% heading
\usepackage{sectsty}
\setcounter{secnumdepth}{2}
\allsectionsfont{\centering\huge\rmfamily}

% for Chinese
\usepackage{xeCJK}
\usepackage{zhnumber}
\setCJKmainfont[BoldFont=FandolSong-Bold.otf]{FandolSong-Regular.otf}

% for fonts
\usepackage{fontspec}
\newcommand{\song}{\CJKfamily{song}} 
\newcommand{\kai}{\CJKfamily{kai}} 

% To fix the ``MakeTextLowerCase'' bug:
% See https://github.com/Tufte-LaTeX/tufte-latex/issues/64#issuecomment-78572017
% Set up the spacing using fontspec features
\renewcommand\allcapsspacing[1]{{\addfontfeature{LetterSpace=15}#1}}
\renewcommand\smallcapsspacing[1]{{\addfontfeature{LetterSpace=10}#1}}

% for url
\usepackage{hyperref}
\hypersetup{colorlinks = true, 
  linkcolor = teal,
  urlcolor  = teal,
  citecolor = blue,
  anchorcolor = blue}

\newcommand{\me}[4]{
    \author{
      {\bfseries 姓名:}\underline{#1}\hspace{2em}
      {\bfseries 学号:}\underline{#2}\hspace{2em}\\[10pt]
      {\bfseries 评分:}\underline{#3\hspace{3em}}\hspace{2em}
      {\bfseries 评阅:}\underline{#4\hspace{3em}}
  }
}

% Please ALWAYS Keep This.
\newcommand{\noplagiarism}{
  \begin{center}
    \fbox{\begin{tabular}{@{}c@{}}
      请独立完成作业,不得抄袭。\\
      若得到他人帮助, 请致谢。\\
      若参考了其它资料,请给出引用。\\
      鼓励讨论,但需独立书写解题过程。
    \end{tabular}}
  \end{center}
}

\newcommand{\goal}[1]{
  \begin{center}{\fcolorbox{blue}{yellow!60}{\parbox{0.50\textwidth}{\large 
    \begin{itemize}
      \item 体会``思维的乐趣''
      \item 初步了解递归与数学归纳法 
      \item 初步接触算法概念与问题下界概念
    \end{itemize}}}}
  \end{center}
}

% Each hw consists of four parts:
\newcommand{\beginrequired}{\hspace{5em}\section{作业 (必做部分)}}
\newcommand{\beginoptional}{\section{作业 (选做部分)}}
\newcommand{\beginot}{\section{Open Topics}}
\newcommand{\begincorrection}{\section{订正}}
\newcommand{\beginfb}{\section{反馈}}

% for math
\usepackage{amsmath, mathtools, amsfonts, amssymb}
\newcommand{\set}[1]{\{#1\}}

% define theorem-like environments
\usepackage[amsmath, thmmarks]{ntheorem}

\theoremstyle{break}
\theorempreskip{2.0\topsep}
\theorembodyfont{\song}
\theoremseparator{}
\newtheorem{problem}{题目}[subsection]
\renewcommand{\theproblem}{\arabic{problem}}
\newtheorem{ot}{Open Topics}

\theorempreskip{3.0\topsep}
\theoremheaderfont{\kai\bfseries}
\theoremseparator{:}
\theorempostwork{\bigskip\hrule}
\newtheorem*{solution}{解答}
\theorempostwork{\bigskip\hrule}
\newtheorem*{revision}{订正}

\theoremstyle{plain}
\newtheorem*{cause}{错因分析}
\newtheorem*{remark}{注}

\theoremstyle{break}
\theorempostwork{\bigskip\hrule}
\theoremsymbol{\ensuremath{\Box}}
\newtheorem*{proof}{证明}

% \newcommand{\ot}{\blue{\bf [OT]}}

% for figs
\renewcommand\figurename{图}
\renewcommand\tablename{表}

% for fig without caption: #1: width/size; #2: fig file
\newcommand{\fig}[2]{
  \begin{figure}[htbp]
    \centering
    \includegraphics[#1]{#2}
  \end{figure}
}
% for fig with caption: #1: width/size; #2: fig file; #3: caption
\newcommand{\figcap}[3]{
  \begin{figure}[htbp]
    \centering
    \includegraphics[#1]{#2}
    \caption{#3}
  \end{figure}
}
% for fig with both caption and label: #1: width/size; #2: fig file; #3: caption; #4: label
\newcommand{\figcaplbl}[4]{
  \begin{figure}[htbp]
    \centering
    \includegraphics[#1]{#2}
    \caption{#3}
    \label{#4}
  \end{figure}
}
% for margin fig without caption: #1: width/size; #2: fig file
\newcommand{\mfig}[2]{
  \begin{marginfigure}
    \centering
    \includegraphics[#1]{#2}
  \end{marginfigure}
}
% for margin fig with caption: #1: width/size; #2: fig file; #3: caption
\newcommand{\mfigcap}[3]{
  \begin{marginfigure}
    \centering
    \includegraphics[#1]{#2}
    \caption{#3}
  \end{marginfigure}
}

\usepackage{fancyvrb}

% for algorithms
\usepackage[]{algorithm}
\usepackage[]{algpseudocode} % noend
% See [Adjust the indentation whithin the algorithmicx-package when a line is broken](https://tex.stackexchange.com/a/68540/23098)
\newcommand{\algparbox}[1]{\parbox[t]{\dimexpr\linewidth-\algorithmicindent}{#1\strut}}
\newcommand{\hStatex}[0]{\vspace{5pt}}
\makeatletter
\newlength{\trianglerightwidth}
\settowidth{\trianglerightwidth}{$\triangleright$~}
\algnewcommand{\LineComment}[1]{\Statex \hskip\ALG@thistlm \(\triangleright\) #1}
\algnewcommand{\LineCommentCont}[1]{\Statex \hskip\ALG@thistlm%
  \parbox[t]{\dimexpr\linewidth-\ALG@thistlm}{\hangindent=\trianglerightwidth \hangafter=1 \strut$\triangleright$ #1\strut}}
\makeatother

% for footnote/marginnote
% see https://tex.stackexchange.com/a/133265/23098
\usepackage{tikz}
\newcommand{\circled}[1]{%
  \tikz[baseline=(char.base)]
  \node [draw, circle, inner sep = 0.5pt, font = \tiny, minimum size = 8pt] (char) {#1};
}
\renewcommand\thefootnote{\protect\circled{\arabic{footnote}}} % feel free to modify this file
%%%%%%%%%%%%%%%%%%%%
\title{第4-5讲: 数论算法}
\me{朱宇博}{ 191220186}{}{}
\date{\zhtoday} % or like 2019年9月13日
%%%%%%%%%%%%%%%%%%%%
\begin{document}
\maketitle
%%%%%%%%%%%%%%%%%%%%
\noplagiarism % always keep this line
%%%%%%%%%%%%%%%%%%%%
\begin{abstract}
  % \begin{center}{\fcolorbox{blue}{yellow!60}{\parbox{0.65\textwidth}{\large 
  %   \begin{itemize}
  %     \item 
  %   \end{itemize}}}}
  % \end{center}
\end{abstract}
%%%%%%%%%%%%%%%%%%%%
\beginrequired

%%%%%%%%%%%%%%%
\begin{problem}[TC 31.1-12]
\end{problem}

\begin{solution}
\begin{algorithm}
\caption{div}\label{euclid}
\begin{algorithmic}[1]
\Procedure{div}{$A[1,...,\beta],B[1,...,\alpha]$}\Comment{$A$ is Dividend and $B$ is divisor}
	\State $C[1...\beta] = 0$
		\For {$i \gets \beta - \alpha + 1 \to 1$}
			\While{$A[\beta,...,i]\geq B[\alpha,...,1]$}
				\State $A[\beta,...,i]\gets A[\beta,...,i]-B[\alpha,...,1]$
				\State $C[i]\gets C[i] + 1$
			\EndWhile
		\EndFor
	\State \Return A, C \Comment{$A$ is remainder and $C$ is quotient}
	\EndProcedure
\end{algorithmic}
\end{algorithm}
\end{solution}
%%%%%%%%%%%%%%%

%%%%%%%%%%%%%%%
\begin{problem}[TC 31.2-5]
\end{problem}

\begin{solution}
假设该算法执行了$k$次迭代,则由引理3.10可知,$b\geq F_{k+1}$,即$b\geq \frac{\phi^{k+1}}{\sqrt{5}}$\\
从而得到$k+1\leq \log_{\phi}\sqrt{5}+ \log_{\phi}b$, 推得$k<1+ \log_{\phi}b$\\
设整数$r$和$s$的最大公因数为$d$, 根据欧几里得算法,$gcd(r,s)$调用一次后变成$gcd(s, r\mod s)$, $gcd(rd,sd)$调用一次后变成$gcd(sd, (r\mod s)d)$\\
根据算法终止条件,当第二维为$0$时停止递归。由于$r \mod s =0$ 当且仅当$ (r\mod s)d$为0(d不为0)\\
故$gcd(r,s)$和$gcd(rd,sd)$迭代次数相同。\\
由此引理可知$gcd(\frac{a}{gcd(a,b)}, \frac{b}{gcd(a,b)})$和$gcd(a,b)$的次数相同。\\
将$\frac{b}{gcd(a,b)}$代入,可得$k<1+ \log_{\phi}\frac{b}{gcd(a,b)}$
\end{solution}
%%%%%%%%%%%%%%%

%%%%%%%%%%%%%%%
\begin{problem}[TC 31.3-5]
\end{problem}

\begin{solution}
$\forall x_1,x_2\in \mathbb{Z}_n^{*}, f_a(x_1)=f_a(x_2)\to ax_1\mod n= ax_2\mod n \to x_1=x_2$(群的right and left cancellation laws)\\
$\forall y\in \mathbb{Z}_n^{*}, \exists x  = a^{-1}y \mod n \in \mathbb{Z}_n^{*}, f_a(x)=y$\\
因此$f$是双射函数,可得其为$\mathbb{Z}_n^{*}$的一个置换
\end{solution}
%%%%%%%%%%%%%%%

%%%%%%%%%%%%%%%
\begin{problem}[TC 31.4-2]
\end{problem}

\begin{solution}
当$gcd(a,n)=1$时,$a$存在逆元$a'$使得$aa'\equiv 1 \pmod n$。则
\[
ax\equiv ay \pmod n \to a'ax\equiv a'ay\pmod n\to x\equiv y\pmod n
\]
当$n=4$时,$2*5\equiv 2 *3 \pmod 4$但$5$和$3$在模$4$的意义下不相等
\end{solution}
%%%%%%%%%%%%%%%

%%%%%%%%%%%%%%%
\begin{problem}[TC 31.5-2]
\end{problem}

\begin{solution}
$m_1=56,m_2=63,m_3=72,c_1=280, c_2=441, c_3=288$\\
\[
\begin{aligned}
a&\equiv 1*c_1+2*c_2+3*c_3\pmod{9*8*7}\\
&\equiv 2026\pmod {504}\\
&\equiv 10 \pmod {504}
\end{aligned}
\]
\end{solution}
%%%%%%%%%%%%%%%
\newpage
%%%%%%%%%%%%%%%
\begin{problem}[TC 31.6-2]
\end{problem}

\begin{solution}
\begin{algorithm}
\caption{quick-mod}\label{euclid}
\begin{algorithmic}[1]
\Procedure{quick\_mod}{$a,b,n$}\Comment{$ab\%n$}
	\State $ans \gets 1$
	\While {$b \neq 0$}
		\If {$b\mod2 == 1$}
			\State $ans \gets ans *a \%n$
		\EndIf
		\State $a\gets a*a\%n$
		\State $b\gets \lfloor \frac{b}{2}\rfloor$
	\EndWhile
	\State \Return ans 
	\EndProcedure
\end{algorithmic}
\end{algorithm}
\end{solution}
%%%%%%%%%%%%%%%

%%%%%%%%%%%%%%%
\begin{problem}[TC 31.1-13(有勘误)]
Give an efficient algorithm to convert a given $\beta$-bit (binary) integer to a decimal representation. 
Argue that if multiplication or division of integers whose length is at most $\beta$ takes time \red{ $M(\beta)=\Omega(\beta)$}, 
then we can convert binary to decimal in time \red{$O(M(\beta)\lg \beta)$}. (Hint: Use a divide-and-conquer approach, obtaining the top and
bottom halves of the result with separate recursions.)

勘误详细参见: \href{https://www.cs.dartmouth.edu/~thc/clrs-bugs/bugs-3e.php}{https://www.cs.dartmouth.edu/~thc/clrs-bugs/bugs-3e.php}
\end{problem}

\begin{solution}
采用分治算法解决该问题,每次将二进制串分为两部分,分而治之,递归处理计算每部分转成十进制后的结果。\\
在合并时,将高位所在区间的结果,乘上对应位数的位权。根据题设已知乘法时间\\
故$T(\beta)=2T(\frac{\beta}{2})+M(\beta)$,解得$T(\beta)=O(M(\beta)\lg\beta)$
\end{solution}
%%%%%%%%%%%%%%%

%%%%%%%%%%%%%%%
\begin{problem}[TC 31.2-9]
\end{problem}

\begin{solution}
$\textbf{Theorem 31.6}$\\
For any integers $a$, $b$, and $p$, if both $gcd(a,p)=1$ and $gcd(b,p)=1$, then $gcd(ab,p)=1$.\\
(1)\\
充分性:\\
若$gcd(n_1n_2,n_3n_4)=1$,即可推出$n_1,n_2$中任意整数,与$n_3,n_4$中任意整数互质。\\
证明: 反证。不矢一般性地假设$gcd(n_1,n_3)=k(k>1)$,则$k|gcd(n_1n_2,n_3n_4)$,与题设相悖。故得证。\\
同理,$gcd(n_1n_3,n_2n_4)=1$,即可推出$n_1,n_3$中任意整数,与$n_2,n_4$中任意整数互质。\\
综上可得出$n_1,n_2,n_3,n_4$互质。即$gcd(n_1n_2,n_3n_4)=1\land gcd(n_1n_3,n_2n_4)=1\to n_1,n_2,n_3,n_4$互质。\\
必要性:\\
若 $n_1,n_2,n_3,n_4$互质,由TH31.6,可得$gcd(n_1n_2,n_3)=1,gcd(n_1n_2,n_4)=1$,再用TH31.6,可得$gcd(n_1n_2,n_3n_4)=1$。
同理,$gcd(n_1n_3,n_2n_4)=1$。得证\\
故为充要条件。\\
(2)\\
充分性:\\
我们需要找到一种划分方法,使得经过$\lceil\lg k\rceil$对互质条件后,推出每两个数之间互质。\\
为简便起见,以下讨论\textbf{不考虑奇偶对“均分”带来的影响}。\\
初始时,所有的数均标号为$1$, 若要推出所有元素之间互质,我们需要证相同标号的数字之间互质。\\
第一次:将原标号为$1$的数\textbf{均分}成两部分,分别标号$1$,$2$。则将$\{\text{标号为1}\}$,$\{\text{标号为2}\}$作为新数对。显然这之后,我们仍只需证相同标号的元素之间互质。。\\
第二次:将原标号为$1$的数\textbf{均分}成两部分,分别标号$1$,$2$。将原标号为$2$的数均分成两部分,分别标号$3$,$4$。则将$\{\text{标号为1和3}\}$,$\{\text{标号为2和4}\}$作为新数对。显然这之后,我们仍只需证相同标号的元素之间互质。\\
第$k$次:将原标号为$k$的数\textbf{均分}成两部分,分别标号$2k-1$,$2k$。则将$\{\text{标号为奇数}\}$,\\$\{\text{标号为偶数}\}$作为新数对。显然这之后,我们仍只需证相同标号的元素之间互质。\\
当划分到每个标号都只有1个数时,即可说明两两互质。\\
因为标号相同的数在每一轮开始时,都会被\textbf{均分}成两组标号不同的数,故经过$\lceil\lg k\rceil$次后,即可满足每个标号都只有1个数。\\
故我们找到了一种经过$\lceil\lg k\rceil$对互质条件后,推出每两个数之间互质的方法。充分性得证。\\

\noindent 必要性:\\
若 $n_1,n_2,n_3,n_4$互质,由TH31.6即其拓展,显然可得其中导出的$\lceil \lg k \rceil$对整数互质。\\

\end{solution}
%%%%%%%%%%%%%%%




%%%%%%%%%%%%%%%
\begin{problem}[TC 31.5-3]
\end{problem}

\begin{solution}
由于$gcd(n,a)=1$,根据群的性质,在$\mathbb{Z}_n$中,有
\[
a^{-1}\leftrightarrow a
\]
由中国剩余定理,有
\[
a\leftrightarrow (a_1,a_2,...,a_n)
\]
由于$gcd(n,a)=1$,可推得$gcd(n_i,a)=1$。\\
由欧几里得定理,$gcd(n_i, a_i)=gcd(n_i, a \mod n_i)=gcd(n_i,a)=1$。\\
根据群的性质,在$\mathbb{Z}_{n_1}\times \mathbb{Z}_{n_2}\times...\times\mathbb{Z}_{n_k}$中,有
\[
(a_1,a_2,...,a_k)\leftrightarrow (a_1^{-1},a_2^{-1},...,a_k^{-1})
\]
由以上三组关系的传递性,可得
\[
a^{-1}\leftrightarrow (a_1^{-1},a_2^{-1},...,a_k^{-1})
\]
即
\[
a^{-1}\mod n\leftrightarrow (a_1^{-1}\mod n_1,a_2^{-1}\mod n_2,...,a_k^{-1}\mod n_k)
\]
得证。
\end{solution}
%%%%%%%%%%%%%%%

%%%%%%%%%%%%%%%
\begin{problem}[TC 31.6-3]
\end{problem}

\begin{solution}
由欧拉定理,利用该算法计算$a^{\phi(n)-1}\mod n$即可,该数即为$a$的模$n$乘法逆元。
\end{solution}
%%%%%%%%%%%%%%%

%%%%%%%%%%%%%%%%%%%%
\beginoptional

%%%%%%%%%%%%%%%
\begin{problem}[同余方程组]
解同余方程组:
$$\begin{array}{ll}
	x &= 3 (\mod 8),\\
	x &= 11 (\mod 20),\\
	x &=1 (\mod 15). 
\end{array}
$$
\end{problem}

\begin{solution}
\end{solution}
%%%%%%%%%%%%%%%


%%%%%%%%%%%%%%%%%%%%
\beginot
%%%%%%%%%%%%%%%
\begin{ot}[乘法算法]	
	请给出n位整数相乘的算法
	\begin{itemize}
	\item $O(n^2)$?
	\item $O(n^{\lg 3})$?
	\item 更快的其他算法?
	\end{itemize}	
	(参考资料:\href{https://en.wikipedia.org/wiki/Multiplication_algorithm}{https://en.wikipedia.org/wiki/Multiplication\_algorithm})
\end{ot}

% \begin{solution}
% \end{solution}
%%%%%%%%%%%%%%%

%%%%%%%%%%%%%%%
\begin{ot}[Pollard's rho algorithm]	
	Pollard's rho algorithm is an algorithm for integer factorization.
	
	(参考资料:\href{https://en.wikipedia.org/wiki/Pollard's_rho_algorithm}{https://en.wikipedia.org/wiki/Pollard's\_rho\_algorithm})
\end{ot}


% \begin{solution}
% \end{solution}
%%%%%%%%%%%%%%%


% \vspace{0.50cm}
%%%%%%%%%%%%%%%
% \begin{ot}[]
% 
%   \noindent 参考资料:
%   \begin{itemize}
%     \item 
%   \end{itemize}
% \end{ot}

% \begin{solution}
% \end{solution}
%%%%%%%%%%%%%%%

%%%%%%%%%%%%%%%%%%%%
% 如果没有需要订正的题目,可以把这部分删掉

% \begincorrection
%%%%%%%%%%%%%%%%%%%%

%%%%%%%%%%%%%%%%%%%%
% 如果没有反馈,可以把这部分删掉
\beginfb

% 你可以写
% ~\footnote{优先推荐 \href{problemoverflow.top}{ProblemOverflow}}:
% \begin{itemize}
%   \item 对课程及教师的建议与意见
%   \item 教材中不理解的内容
%   \item 希望深入了解的内容
%   \item $\cdots$
% \end{itemize}
%%%%%%%%%%%%%%%%%%%%
% \bibliography{2-5-solving-recurrence}
% \bibliographystyle{plainnat}
%%%%%%%%%%%%%%%%%%%%
\end{document}