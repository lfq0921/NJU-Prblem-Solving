% 2-15-rb-tree.tex

%%%%%%%%%%%%%%%%%%%%
\documentclass[a4paper, justified]{tufte-handout}

% hw-preamble.tex

% geometry for A4 paper
% See https://tex.stackexchange.com/a/119912/23098
\geometry{
  left=20.0mm,
  top=20.0mm,
  bottom=20.0mm,
  textwidth=130mm, % main text block
  marginparsep=5.0mm, % gutter between main text block and margin notes
  marginparwidth=50.0mm % width of margin notes
}

% for colors
\usepackage{xcolor} % usage: \color{red}{text}
% predefined colors
\newcommand{\red}[1]{\textcolor{red}{#1}} % usage: \red{text}
\newcommand{\blue}[1]{\textcolor{blue}{#1}}
\newcommand{\teal}[1]{\textcolor{teal}{#1}}

\usepackage{todonotes}

% heading
\usepackage{sectsty}
\setcounter{secnumdepth}{2}
\allsectionsfont{\centering\huge\rmfamily}

% for Chinese
\usepackage{xeCJK}
\usepackage{zhnumber}
\setCJKmainfont[BoldFont=FandolSong-Bold.otf]{FandolSong-Regular.otf}

% for fonts
\usepackage{fontspec}
\newcommand{\song}{\CJKfamily{song}} 
\newcommand{\kai}{\CJKfamily{kai}} 

% To fix the ``MakeTextLowerCase'' bug:
% See https://github.com/Tufte-LaTeX/tufte-latex/issues/64#issuecomment-78572017
% Set up the spacing using fontspec features
\renewcommand\allcapsspacing[1]{{\addfontfeature{LetterSpace=15}#1}}
\renewcommand\smallcapsspacing[1]{{\addfontfeature{LetterSpace=10}#1}}

% for url
\usepackage{hyperref}
\hypersetup{colorlinks = true, 
  linkcolor = teal,
  urlcolor  = teal,
  citecolor = blue,
  anchorcolor = blue}

\newcommand{\me}[4]{
    \author{
      {\bfseries 姓名:}\underline{#1}\hspace{2em}
      {\bfseries 学号:}\underline{#2}\hspace{2em}\\[10pt]
      {\bfseries 评分:}\underline{#3\hspace{3em}}\hspace{2em}
      {\bfseries 评阅:}\underline{#4\hspace{3em}}
  }
}

% Please ALWAYS Keep This.
\newcommand{\noplagiarism}{
  \begin{center}
    \fbox{\begin{tabular}{@{}c@{}}
      请独立完成作业,不得抄袭。\\
      若得到他人帮助, 请致谢。\\
      若参考了其它资料,请给出引用。\\
      鼓励讨论,但需独立书写解题过程。
    \end{tabular}}
  \end{center}
}

\newcommand{\goal}[1]{
  \begin{center}{\fcolorbox{blue}{yellow!60}{\parbox{0.50\textwidth}{\large 
    \begin{itemize}
      \item 体会``思维的乐趣''
      \item 初步了解递归与数学归纳法 
      \item 初步接触算法概念与问题下界概念
    \end{itemize}}}}
  \end{center}
}

% Each hw consists of four parts:
\newcommand{\beginrequired}{\hspace{5em}\section{作业 (必做部分)}}
\newcommand{\beginoptional}{\section{作业 (选做部分)}}
\newcommand{\beginot}{\section{Open Topics}}
\newcommand{\begincorrection}{\section{订正}}
\newcommand{\beginfb}{\section{反馈}}

% for math
\usepackage{amsmath, mathtools, amsfonts, amssymb}
\newcommand{\set}[1]{\{#1\}}

% define theorem-like environments
\usepackage[amsmath, thmmarks]{ntheorem}

\theoremstyle{break}
\theorempreskip{2.0\topsep}
\theorembodyfont{\song}
\theoremseparator{}
\newtheorem{problem}{题目}[subsection]
\renewcommand{\theproblem}{\arabic{problem}}
\newtheorem{ot}{Open Topics}

\theorempreskip{3.0\topsep}
\theoremheaderfont{\kai\bfseries}
\theoremseparator{:}
\theorempostwork{\bigskip\hrule}
\newtheorem*{solution}{解答}
\theorempostwork{\bigskip\hrule}
\newtheorem*{revision}{订正}

\theoremstyle{plain}
\newtheorem*{cause}{错因分析}
\newtheorem*{remark}{注}

\theoremstyle{break}
\theorempostwork{\bigskip\hrule}
\theoremsymbol{\ensuremath{\Box}}
\newtheorem*{proof}{证明}

% \newcommand{\ot}{\blue{\bf [OT]}}

% for figs
\renewcommand\figurename{图}
\renewcommand\tablename{表}

% for fig without caption: #1: width/size; #2: fig file
\newcommand{\fig}[2]{
  \begin{figure}[htbp]
    \centering
    \includegraphics[#1]{#2}
  \end{figure}
}
% for fig with caption: #1: width/size; #2: fig file; #3: caption
\newcommand{\figcap}[3]{
  \begin{figure}[htbp]
    \centering
    \includegraphics[#1]{#2}
    \caption{#3}
  \end{figure}
}
% for fig with both caption and label: #1: width/size; #2: fig file; #3: caption; #4: label
\newcommand{\figcaplbl}[4]{
  \begin{figure}[htbp]
    \centering
    \includegraphics[#1]{#2}
    \caption{#3}
    \label{#4}
  \end{figure}
}
% for margin fig without caption: #1: width/size; #2: fig file
\newcommand{\mfig}[2]{
  \begin{marginfigure}
    \centering
    \includegraphics[#1]{#2}
  \end{marginfigure}
}
% for margin fig with caption: #1: width/size; #2: fig file; #3: caption
\newcommand{\mfigcap}[3]{
  \begin{marginfigure}
    \centering
    \includegraphics[#1]{#2}
    \caption{#3}
  \end{marginfigure}
}

\usepackage{fancyvrb}

% for algorithms
\usepackage[]{algorithm}
\usepackage[]{algpseudocode} % noend
% See [Adjust the indentation whithin the algorithmicx-package when a line is broken](https://tex.stackexchange.com/a/68540/23098)
\newcommand{\algparbox}[1]{\parbox[t]{\dimexpr\linewidth-\algorithmicindent}{#1\strut}}
\newcommand{\hStatex}[0]{\vspace{5pt}}
\makeatletter
\newlength{\trianglerightwidth}
\settowidth{\trianglerightwidth}{$\triangleright$~}
\algnewcommand{\LineComment}[1]{\Statex \hskip\ALG@thistlm \(\triangleright\) #1}
\algnewcommand{\LineCommentCont}[1]{\Statex \hskip\ALG@thistlm%
  \parbox[t]{\dimexpr\linewidth-\ALG@thistlm}{\hangindent=\trianglerightwidth \hangafter=1 \strut$\triangleright$ #1\strut}}
\makeatother

% for footnote/marginnote
% see https://tex.stackexchange.com/a/133265/23098
\usepackage{tikz}
\newcommand{\circled}[1]{%
  \tikz[baseline=(char.base)]
  \node [draw, circle, inner sep = 0.5pt, font = \tiny, minimum size = 8pt] (char) {#1};
}
\renewcommand\thefootnote{\protect\circled{\arabic{footnote}}} % feel free to modify this file
%%%%%%%%%%%%%%%%%%%%
\title{第4-9讲: NP完全理论初步}
\me{朱宇博}{191220186 }{}{}
\date{\zhtoday} % or like 2019年9月13日
%%%%%%%%%%%%%%%%%%%%
\begin{document}
\maketitle
%%%%%%%%%%%%%%%%%%%%
\noplagiarism % always keep this line
%%%%%%%%%%%%%%%%%%%%
\begin{abstract}
  % \begin{center}{\fcolorbox{blue}{yellow!60}{\parbox{0.65\textwidth}{\large 
  %   \begin{itemize}
  %     \item 
  %   \end{itemize}}}}
  % \end{center}
\end{abstract}
%%%%%%%%%%%%%%%%%%%%
\beginrequired

%%%%%%%%%%%%%%%
\begin{problem}[TC  34.1-5]
\end{problem}

\begin{solution}
(1)\\
假设执行$k$次子例程。\\
则第一次输入的规模为$O(n)$,输出的规模为$O(n^c)$,时间为$O(n^c)$。\\
则第$k$次输入的规模为$O(n^{c^{k-1}})$,输出的规模为$O(n^{c^k})$,时间为$O(n^{c^k})$。\\
其中$c^k$为常数,则显然为多项式时间。\\

\noindent 若执行$n$次子例程。\\
则第$k$次输入的规模为$O(n^{c^{n-1}})$,输出的规模为$O(n^{c^n})$,时间为$O(n^{c^n})$。\\
显然不为多项式时间。
\end{solution}
%%%%%%%%%%%%%%%

%%%%%%%%%%%%%%%
\begin{problem}[TC 34.2-3]
\end{problem}

\begin{solution}
将图中顶点标号为$v_1,..,v_n$\\
从$v_1$开始依次做如下操作:	\\
设和$v_1$相连的边集为$E_1$,则任取$e_1,e_2\in E_1$,检验图$G`=(V,(E-E_1)\lor(e_1,e_2))$是否有哈密顿回路。\\
因为图$G$存在哈密顿回路,则总存在上述$G`$使得$G`$中存在哈密顿回路。\\
找到存在哈密顿回路的$G`$后,令$G=G`$,继续对顶点$v_2,v_3,...,v_n$做上述操作。\\
显然若哈密顿回路问题是$P$问题,则上述算法可在多项式时间内结束,并找到顶点集。\\
\end{solution}
%%%%%%%%%%%%%%%

%%%%%%%%%%%%%%%
\begin{problem}[TC 34.2-4]
\end{problem}

\begin{solution}
假设$L_1, L_2\in NP$,$A_1$, $A_2$为 $L_1, L_2$ 的一个多项式验证算法。\\
对于union: \\
对于输入的(x,y),算法A: 若$A_1(x,y)$可接受,或$A_2(x,y)$可接受,则判定为接受,否则拒绝。\\
显然,其可在多项式时间验证,属于$NP$.\\
对于intersection: \\
对于输入的(x,y),算法A: 若$A_1(x,y)$可接受,并且$A_2(x,y)$可接受,则判定为接受,否则拒绝。\\
显然,其可在多项式时间验证,属于$NP$.\\
对于concatenation: \\
对于输入的(x,y),算法A: 若$A_1(x[1,...,i],y[1,...,j])$可接受,并且$A_2(x[i+1,...,n],y[j+1,...,m])$可接受,则判定为接受,否则拒绝。其中$n=|x|, m = |y|$。\\
显然,其可在多项式时间验证,属于$NP$.\\
 Kleene star:\\
 在concatenation中,我们将$(x,y)$各分成$2$段。在该算法中,我们参照该做法,将$x$分为不超过
 在concatenation的验证中,我们将$x$各分成不超过$|x|$段,$y$分成不超过$|y|$段,若每段都可接受,则判定为接受,否则拒绝。\\
 讨论:\\
 若$L\in P$,则$\overline{L}\in P$。 若$L\in NPC$,则目前没有确定的答案,对$\overline{L}$给出明确的界定。
\end{solution}
%%%%%%%%%%%%%%%

%%%%%%%%%%%%%%%
\begin{problem}[TC 34.3-2]
\end{problem}

\begin{solution}
若$L_1\leq L_2$,则存在多项式时间的转换函数$f(x)$,其中$x\in L_1, f(x) \in L_2$\\
若$L_2\leq L_3$,则存在多项式时间的转换函数$g(x)$,其中$x\in L_2, g(x) \in L_3$\\
从而有$g(f(x))$为多项式时间的转换函数,其中$x\in L_1, g(f(x)) \in L_3$\\
故$L_1\leq L_3$
\end{solution}
%%%%%%%%%%%%%%%

%%%%%%%%%%%%%%%
\begin{problem}[TC 34.4-3]
\end{problem}

\begin{solution}
对于有 $n$个变量的布尔公式, 列出其真值表的时间复杂度为$\Omega(2^n)$,故该方法不能多项式时间规约
\end{solution}
%%%%%%%%%%%%%%%

%%%%%%%%%%%%%%%
\begin{problem}[TC 34.2-11]
\end{problem}

\begin{solution}
数学归纳法。\\
若 n = 3,则 其为$K_3$,显然存在哈密顿回路\\
假设对于所有点数为$k$ $(3 \leq k < n)$的连通图都成立\\
则当$k=n$时\\
令 T 表示 G 的任意生成树,取任意一点$x$,并且移除,可以得到 T 的若干个连通块 $G_1, G_2, · · · , G_l$\\
对于任意 $|G_i|>1$,则选出$u, v\in G_i$,满足$e(u,x)\in G_i$,$e(u,v)\in G_i$。根据归纳假设$G_i$ 包含一条由$u$ 到 $v$ 的哈密顿通路。\\
若$|G_i|=1$,则将该点视为该联通块内的哈密顿通路。\\
显然,从$v$出发,依次遍历每个联通块,在每个联通块内按哈密顿通路遍历,即可找到哈密顿回路。\\
综上得证
\end{solution}
%%%%%%%%%%%%%%%

%%%%%%%%%%%%%%%
\begin{problem}[TC 34.4-7]
\end{problem}

\begin{solution}
设有$n$个变量,$m$个子句,对于变量$x_1,x_2,...,x_n$,每个变量创建两个顶点$x_i$和$\neg x_i$。\\
则若有子句$a\lor b$,则添加$ (\neg a, b),(\neg b, a)$两条边。\\
tarjan缩点,若存在$x_i.\neg x_i$在同一强连通分量里,则不可满足。否则,可满足。\\
复杂度$O(n+m)$
\end{solution}
%%%%%%%%%%%%%%%

%%%%%%%%%%%%%%%
\begin{problem}[TC 34.5-6]
\end{problem}

\begin{solution}
(1)\\
首先,我们证明哈密顿路径问题是NP问题。若给出证书$y$,我们只需检验$y$是否构成排列,并且$y$中相领点之间是否存在边即可。显然可多项式时间验证,故为NP问题。\\
(2)\\
我们证明可将哈密顿回路问题规约到哈密顿通路。\\
对于图中的每一条边$e(u,v)\in G$\\
构造$G_1$, $V(G_1)=V(G)\lor \{u_{new}, v_{new}\}$,$E(G_1) = E(G)\lor\{(u_{new}, u), (v_{new}, v)\}/\{(u,v)\}$\\
下证$G$中存在哈密顿回路当且仅当存在符合构造的$G_1$,其存在哈密顿通路。\\
若在原图$G$中存在哈密顿回路,则显然若$(u,v)$属于哈密顿圈,在删去$(u,v)$构造的$G_1$中,存在一条$u_{new}$出发,$v_{new}$结束的哈密顿路径。\\
若在$G_1$中存在哈密顿通路,则一定是条$u_{new}$出发,$v_{new}$结束的哈密顿路径。故原图$G$中显然存在哈密顿回路。\\
调用需要$O(m)$次,显然调用为多项式时间规约。\\
故得证。
\end{solution}
%%%%%%%%%%%%%%%
%%%%%%%%%%%%%%%%%%%%
\beginoptional
%%%%%%%%%%%%%%%

%%%%%%%%%%%%%%%
\begin{problem}[TC 34.5-2]
\end{problem}

\begin{solution}
\end{solution}
%%%%%%%%%%%%%%%

%%%%%%%%%%%%%%%%%%%%
\beginot
%%%%%%%%%%%%%%%
\begin{ot}[NP]	
	The name ``NP'' stands for ``nondeterministic polynomial time.'' The class NP was originally studied	in the context of nondeterminism, but this book uses the somewhat simpler yet equivalent notion of verification. 
	Hopcroft and Ullman [180] give a good presentation of NP-completeness in terms of nondeterministic models of computation.
	
	阅读TC参考文献[180],介绍Hopcroft等人的NP问题定义,并说说两种定义方法是一致的吗?为什么?
	
	\href{http://ce.sharif.edu/courses/94-95/1/ce414-2/resources/root/Text\%20Books/Automata/John\%20E.\%20Hopcroft,\%20Rajeev\%20Motwani,\%20Jeffrey\%20D.\%20Ullman-Introduction\%20to\%20Automata\%20Theory,\%20Languages,\%20and\%20Computations-Prentice\%20Hall\%20(2006).pdf}{John E. Hopcroft and Jeffrey D. Ullman. Introduction to Automata Theory, Languages, and	Computation. Addison-Wesley, 1979.
}
\end{ot}

% \begin{solution}
% \end{solution}
%%%%%%%%%%%%%%%

%%%%%%%%%%%%%%%
\begin{ot}[TSP is NP-hard]	
证明旅行商问题是NPC问题.
\end{ot}


% \begin{solution}
% \end{solution}
%%%%%%%%%%%%%%%


% \vspace{0.50cm}
%%%%%%%%%%%%%%%
% \begin{ot}[]
% 
%   \noindent 参考资料:
%   \begin{itemize}
%     \item 
%   \end{itemize}
% \end{ot}

% \begin{solution}
% \end{solution}
%%%%%%%%%%%%%%%

%%%%%%%%%%%%%%%%%%%%
% 如果没有需要订正的题目,可以把这部分删掉

% \begincorrection
%%%%%%%%%%%%%%%%%%%%

%%%%%%%%%%%%%%%%%%%%
% 如果没有反馈,可以把这部分删掉
\beginfb

% 你可以写
% ~\footnote{优先推荐 \href{problemoverflow.top}{ProblemOverflow}}:
% \begin{itemize}
%   \item 对课程及教师的建议与意见
%   \item 教材中不理解的内容
%   \item 希望深入了解的内容
%   \item $\cdots$
% \end{itemize}
%%%%%%%%%%%%%%%%%%%%
% \bibliography{2-5-solving-recurrence}
% \bibliographystyle{plainnat}
%%%%%%%%%%%%%%%%%%%%
\end{document}