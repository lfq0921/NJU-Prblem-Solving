% 2-15-rb-tree.tex

%%%%%%%%%%%%%%%%%%%%
\documentclass[a4paper, justified]{tufte-handout}

% hw-preamble.tex

% geometry for A4 paper
% See https://tex.stackexchange.com/a/119912/23098
\geometry{
  left=20.0mm,
  top=20.0mm,
  bottom=20.0mm,
  textwidth=130mm, % main text block
  marginparsep=5.0mm, % gutter between main text block and margin notes
  marginparwidth=50.0mm % width of margin notes
}

% for colors
\usepackage{xcolor} % usage: \color{red}{text}
% predefined colors
\newcommand{\red}[1]{\textcolor{red}{#1}} % usage: \red{text}
\newcommand{\blue}[1]{\textcolor{blue}{#1}}
\newcommand{\teal}[1]{\textcolor{teal}{#1}}

\usepackage{todonotes}

% heading
\usepackage{sectsty}
\setcounter{secnumdepth}{2}
\allsectionsfont{\centering\huge\rmfamily}

% for Chinese
\usepackage{xeCJK}
\usepackage{zhnumber}
\setCJKmainfont[BoldFont=FandolSong-Bold.otf]{FandolSong-Regular.otf}

% for fonts
\usepackage{fontspec}
\newcommand{\song}{\CJKfamily{song}} 
\newcommand{\kai}{\CJKfamily{kai}} 

% To fix the ``MakeTextLowerCase'' bug:
% See https://github.com/Tufte-LaTeX/tufte-latex/issues/64#issuecomment-78572017
% Set up the spacing using fontspec features
\renewcommand\allcapsspacing[1]{{\addfontfeature{LetterSpace=15}#1}}
\renewcommand\smallcapsspacing[1]{{\addfontfeature{LetterSpace=10}#1}}

% for url
\usepackage{hyperref}
\hypersetup{colorlinks = true, 
  linkcolor = teal,
  urlcolor  = teal,
  citecolor = blue,
  anchorcolor = blue}

\newcommand{\me}[4]{
    \author{
      {\bfseries 姓名:}\underline{#1}\hspace{2em}
      {\bfseries 学号:}\underline{#2}\hspace{2em}\\[10pt]
      {\bfseries 评分:}\underline{#3\hspace{3em}}\hspace{2em}
      {\bfseries 评阅:}\underline{#4\hspace{3em}}
  }
}

% Please ALWAYS Keep This.
\newcommand{\noplagiarism}{
  \begin{center}
    \fbox{\begin{tabular}{@{}c@{}}
      请独立完成作业,不得抄袭。\\
      若得到他人帮助, 请致谢。\\
      若参考了其它资料,请给出引用。\\
      鼓励讨论,但需独立书写解题过程。
    \end{tabular}}
  \end{center}
}

\newcommand{\goal}[1]{
  \begin{center}{\fcolorbox{blue}{yellow!60}{\parbox{0.50\textwidth}{\large 
    \begin{itemize}
      \item 体会``思维的乐趣''
      \item 初步了解递归与数学归纳法 
      \item 初步接触算法概念与问题下界概念
    \end{itemize}}}}
  \end{center}
}

% Each hw consists of four parts:
\newcommand{\beginrequired}{\hspace{5em}\section{作业 (必做部分)}}
\newcommand{\beginoptional}{\section{作业 (选做部分)}}
\newcommand{\beginot}{\section{Open Topics}}
\newcommand{\begincorrection}{\section{订正}}
\newcommand{\beginfb}{\section{反馈}}

% for math
\usepackage{amsmath, mathtools, amsfonts, amssymb}
\newcommand{\set}[1]{\{#1\}}

% define theorem-like environments
\usepackage[amsmath, thmmarks]{ntheorem}

\theoremstyle{break}
\theorempreskip{2.0\topsep}
\theorembodyfont{\song}
\theoremseparator{}
\newtheorem{problem}{题目}[subsection]
\renewcommand{\theproblem}{\arabic{problem}}
\newtheorem{ot}{Open Topics}

\theorempreskip{3.0\topsep}
\theoremheaderfont{\kai\bfseries}
\theoremseparator{:}
\theorempostwork{\bigskip\hrule}
\newtheorem*{solution}{解答}
\theorempostwork{\bigskip\hrule}
\newtheorem*{revision}{订正}

\theoremstyle{plain}
\newtheorem*{cause}{错因分析}
\newtheorem*{remark}{注}

\theoremstyle{break}
\theorempostwork{\bigskip\hrule}
\theoremsymbol{\ensuremath{\Box}}
\newtheorem*{proof}{证明}

% \newcommand{\ot}{\blue{\bf [OT]}}

% for figs
\renewcommand\figurename{图}
\renewcommand\tablename{表}

% for fig without caption: #1: width/size; #2: fig file
\newcommand{\fig}[2]{
  \begin{figure}[htbp]
    \centering
    \includegraphics[#1]{#2}
  \end{figure}
}
% for fig with caption: #1: width/size; #2: fig file; #3: caption
\newcommand{\figcap}[3]{
  \begin{figure}[htbp]
    \centering
    \includegraphics[#1]{#2}
    \caption{#3}
  \end{figure}
}
% for fig with both caption and label: #1: width/size; #2: fig file; #3: caption; #4: label
\newcommand{\figcaplbl}[4]{
  \begin{figure}[htbp]
    \centering
    \includegraphics[#1]{#2}
    \caption{#3}
    \label{#4}
  \end{figure}
}
% for margin fig without caption: #1: width/size; #2: fig file
\newcommand{\mfig}[2]{
  \begin{marginfigure}
    \centering
    \includegraphics[#1]{#2}
  \end{marginfigure}
}
% for margin fig with caption: #1: width/size; #2: fig file; #3: caption
\newcommand{\mfigcap}[3]{
  \begin{marginfigure}
    \centering
    \includegraphics[#1]{#2}
    \caption{#3}
  \end{marginfigure}
}

\usepackage{fancyvrb}

% for algorithms
\usepackage[]{algorithm}
\usepackage[]{algpseudocode} % noend
% See [Adjust the indentation whithin the algorithmicx-package when a line is broken](https://tex.stackexchange.com/a/68540/23098)
\newcommand{\algparbox}[1]{\parbox[t]{\dimexpr\linewidth-\algorithmicindent}{#1\strut}}
\newcommand{\hStatex}[0]{\vspace{5pt}}
\makeatletter
\newlength{\trianglerightwidth}
\settowidth{\trianglerightwidth}{$\triangleright$~}
\algnewcommand{\LineComment}[1]{\Statex \hskip\ALG@thistlm \(\triangleright\) #1}
\algnewcommand{\LineCommentCont}[1]{\Statex \hskip\ALG@thistlm%
  \parbox[t]{\dimexpr\linewidth-\ALG@thistlm}{\hangindent=\trianglerightwidth \hangafter=1 \strut$\triangleright$ #1\strut}}
\makeatother

% for footnote/marginnote
% see https://tex.stackexchange.com/a/133265/23098
\usepackage{tikz}
\newcommand{\circled}[1]{%
  \tikz[baseline=(char.base)]
  \node [draw, circle, inner sep = 0.5pt, font = \tiny, minimum size = 8pt] (char) {#1};
}
\renewcommand\thefootnote{\protect\circled{\arabic{footnote}}} % feel free to modify this file
%%%%%%%%%%%%%%%%%%%%
\title{第4-2讲: 置换群与拉格朗日定理}
\me{朱宇博 }{191220186 }{}{}
\date{\zhtoday} % or like 2019年9月13日
%%%%%%%%%%%%%%%%%%%%
\begin{document}
\maketitle
%%%%%%%%%%%%%%%%%%%%
\noplagiarism % always keep this line
%%%%%%%%%%%%%%%%%%%%
\begin{abstract}
  % \begin{center}{\fcolorbox{blue}{yellow!60}{\parbox{0.65\textwidth}{\large 
  %   \begin{itemize}
  %     \item 
  %   \end{itemize}}}}
  % \end{center}
\end{abstract}
%%%%%%%%%%%%%%%%%%%%
\beginrequired

%%%%%%%%%%%%%%%
\begin{problem}[TJ 5-3(d)]
\end{problem}

\begin{solution}
(17254)(1423)(154632)=(17254)(24615)=(14672)=(12)(17)(16)(14)\\
It is an even permutation.
\end{solution}
%%%%%%%%%%%%%%%

%%%%%%%%%%%%%%%
\begin{problem}[TJ 5-5 (注: 只需列出S4的所有子群, 无需解(a)、(b)、(c))]
\end{problem}

\begin{solution}
The subgroup of $S4$:\\
$N_1=\{1\}$\\
$N_2=\{1),(12)\}$\\
$N_3=\{1),(13)\}$\\
$N_4=\{1),(23)\}$\\
$N_5=\{1),(24)\}$\\
$N_6=\{1),(14)\}$\\
$N_7=\{1),(34)\}$\\
$N_8=\{1),(12),(34)\}$\\
$N_9=\{1),(13),(24)\}$\\
$N_{10}=\{1),(14),(23)\}$\\
$N_{11}=\{1),(123),(132)\}$\\
$N_{12}=\{1),(134),(143)\}$\\
$N_{13}=\{1),(124),(142)\}$\\
$N_{14}=\{1),(234),(243)\}$\\
$N_{15}=\{1),(1234),(13)(24),(1432)\}$\\
$N_{16}=\{1),(1234),(12)(34),(1432)\}$\\
$N_{17}=\{1),(1243),(14)(23),(1342)\}$\\
$N_{18}=\{1),(12),(34),(12)(34)\}$\\
$N_{19}=\{1),(13),(24),(13)(24)\}$\\
$N_{20}=\{1),(14),(23),(14)(23)\}$\\
$N_{21}=\{1),(12)(34),(13)(24),(14)(23)\}$\\
$N_{22}=\{(1),(1234),(13)(24),(1432),(13),(12)(34),(24),(14)(23)\}$\\
$N_{23}=\{(1),(1324),(12)(34),(1423),(12),(13)(24),(34),(14)(32)\}$\\
$N_{24}=\{(1,(1243),(14)(23),(1342),(14),(12)(43),(34),(14)(32)\}$\\
$N_{25}=S_4$\\
$N_{26}=\{(1),(12),(13),(23),(123),(132)\}$\\
$N_{27}=\{(1),(12),(24),(14),(124),(142)\}$\\
$N_{28}=\{(1),(34),(13),(14),(143),(134)\}$\\
$N_{29}=\{(1),(34),(24),(23),(234),(243)\}$\\
$N_{30}=\{(1),(123),(132),(134),(143),(124),(142),(234),(243),(12)(34),(13)(24),(14)(23)\} $
\end{solution}
%%%%%%%%%%%%%%%

%%%%%%%%%%%%%%%
\begin{problem}[TJ 5-16]
\end{problem}

\begin{solution}
For a tetrahedron,  we mark its vertex as A, B, C and D.\\
Then $\{(id), (ABC),(ACB), (ABD), (ADB), (ACD), (ADC), (BCD), (BDC), (AB)(CD), (AC)(BD), (AD)(BC)\}$ form the sports group.\\
There is a bijective function $f$ between $A_4$ and tetrahedron apparently by $\sigma(A)\to 1, \sigma(B)\to 2, \sigma(C)\to 3, \sigma(D)\to 4$.\\


\end{solution}
%%%%%%%%%%%%%%%

%%%%%%%%%%%%%%%
\begin{problem}[TJ 5-26(b)]
\end{problem}

\begin{solution}
$\forall (a_1,a_2,…,a_n)\in A_4$
$$(a_1,a_2,…,a_n)=(a_1 a_n)(a_1 a_{n−1} )⋯(a_1 a_3)(a_1 a_2)$$
For $(a_1, a_2)$, we assume that $a_1 \leq a_2$.\\
We have that $(a_1,a_2)=(a_1,a_1+1)(a_1+1,a_1+2)...(a_2-2,a_2-1)(a_2-1,a_2)(a_2-2,a_2-1)...(a_1+1,a_1+2)(a_1,a_1+1)$\\
It is the same as $(a_1,a_3),...,(a_1,a_n)$.\\
Therefore, any element in $S_4$ can be written as a finite product of $(12),(23),...,(n-1,n)$
\end{solution}
%%%%%%%%%%%%%%%

%%%%%%%%%%%%%%%
\begin{problem}[TJ 5-29]
\end{problem}

\begin{solution}
$Z(D_8)=\{1,r^4\}, Z(D_{10})=\{1, r^5\}$\\
$Z(D_n)=\{1,\frac{n}{2}\}$(n is even),  $Z(D_n)=\{1\}$(n is odd).  
\end{solution}
%%%%%%%%%%%%%%%

%%%%%%%%%%%%%%%
\begin{problem}[TJ 5-36]
\end{problem}

\begin{solution}
(a)\\
$$s^2=1\to s=s^{-1}$$
$$ srs = r^{-1}\Leftrightarrow rsr=s$$
So we only need to show $rsr=s$.\\
We assume that $s=s_1$. Observe that the graph after operations $rsr$.\\
At the beginning, the first vertex is $1$, and the second vertex is $2$.\\
After the operation $r$,  the first vertex is $2$, and the second vertex is $3$.\\
After the operation $s$,  the first vertex is $2$, and the second vertex is $1$.\\
At the end, after the operation $r$,  the first vertex is $1$, and the second vertex is $n$.\\
It is the same with the single operation $s$. So $rsr=s$ and $srs=r^{-1}$\\
(b)\\
$$ srs = r^{-1}\Leftrightarrow (srs)^k = r^{-k}\Leftrightarrow srss^{-1}rs...s^{-1}rs = r^{-1}\Leftrightarrow  sr^ks = r^{-k} \Leftrightarrow r^ks=sr^{-k}$$\\
(c)\\
Let $C_n=\{r^{k} | r^{k}\in D_n\}$. $C_n$ is a subgroup of $D_n$.\\
Obviously, $C_n$ is a cyclic group with the generator $r$. The order of $r$ is $n$.\\
Due to \textbf{Theorem 4.13} in TJ, the order of $r_k$ is $\frac{n}{gcd(n,k)}$ in the group $C_n$.\\
Since $C_n$ is a cyclic subgroup of $D_n$, the order of $r^k$ is also $\frac{n}{gcd(n,k)}$.
\end{solution}
%%%%%%%%%%%%%%%

%%%%%%%%%%%%%%%
\begin{problem}[TJ 6-11 (注意:(c)中 $\subset$表示$\subseteq$)]
\end{problem}

\begin{solution}
$1. (a)\to (c):\\$
$g_1H=g_2H$\\
$\to (x\in g_1H\to x\in g_2 H)$\\
$\to g_1H\subset g_2H$\\

\noindent $2. (c)\to (e):\\$
$ g_1H\subset g_2H$\\
$\to (x\in g_1H\to x\in g_2 H)$\\
$\to\forall h_1\in H, (\exists h_2\in H, g_1h_1=g_2h_2)$\\
$\to \forall h_1\in H, (\exists h_2\in H, h_1h_2^{-1}=g_1^{-1}g_2)$\\
$\to g_1^{-1}g_2\in H$\\

\noindent $3. (e)\to (d):\\$
$g_1^{-1}g_2\in H$\\
$\to \exists h\in H, g_1^{-1}g_2=h$\\
$\to \exists h\in H, g_2=g_1h$\\
$g_2\in g_1H$\\

\noindent $4. (d)\to (b):\\$
$g_2\in g_1H$\\
$\to \exists h\in H, g_2=g_1h$\\
$\to \exists h\in H, g_1^{-1}=hg_2^{-1}$\\
$\to \exists h\in H, (\forall h_1\in H,h_1g_1^{-1}=h_1hg_2^{-1}\in Hg_2^{-1})$\\
$\to Hg_1^{-1}\subset Hg_2^{-1}$\\
similarly, we have $Hg_2^{-1}\subset Hg_1^{-1}$\\
So $Hg_1^{-1}= Hg_2^{-1}$\\

\noindent $5. (b)\to (a):\\$
$Hg_1^{-1}= Hg_2^{-1}$\\
$\to Hg_1^{-1}\subset Hg_2^{-1}$\\
$\to \forall h_1\in H, \exists h_2\in H, h_1g_1^{-1}= h_2g_2^{-1}$\\
$\to \exists h_3\in H, g_1=g_2h_3$\\
$\to \forall h_4\in H, \exists h_3\in H, g_1h_4=g_2h_3h_4\in g_2H$\\
$\to g_1H\subset g_2H$\\
similarly, we have $g_2H\subset g_1H$\\
So $g_1H= g_2H$\\
\end{solution}
%%%%%%%%%%%%%%%


%%%%%%%%%%%%%%%%%%%%
\beginoptional

%%%%%%%%%%%%%%%
\begin{problem}[$Z_p$]
证明:$A_n$ 中的每个置换皆可表成形如 $(k \text{ } k+1 \text{ } k+2)$ 的 3-cycle 的乘积。
\end{problem}

\begin{solution}
\end{solution}
%%%%%%%%%%%%%%%

%%%%%%%%%%%%%%%
\begin{problem}[SageMath学习]
学习 TJ 第五章, 第六章关于 SageMath 的内容
\end{problem}

\begin{solution}
\end{solution}
%%%%%%%%%%%%%%%

%%%%%%%%%%%%%%%%%%%%
\beginot
%%%%%%%%%%%%%%%
\begin{ot}[二阶魔方]	
	请构造出二阶魔方相关的置换群,你能设计一种算法来解二阶魔方复原吗?
\end{ot}

% \begin{solution}
% \end{solution}
%%%%%%%%%%%%%%%

%%%%%%%%%%%%%%%
\begin{ot}[transpositions]	
	证明:Show that any cycle can be written as the product of transpositions:
	$$(a_1,a_2,…,a_n)=(a_1 a_n)(a_1 a_{n−1} )⋯(a_1 a_3)(a_1 a_2)$$
	
\end{ot}


% \begin{solution}
% \end{solution}
%%%%%%%%%%%%%%%


% \vspace{0.50cm}
%%%%%%%%%%%%%%%
% \begin{ot}[]
% 
%   \noindent 参考资料:
%   \begin{itemize}
%     \item 
%   \end{itemize}
% \end{ot}

% \begin{solution}
% \end{solution}
%%%%%%%%%%%%%%%

%%%%%%%%%%%%%%%%%%%%
% 如果没有需要订正的题目,可以把这部分删掉

% \begincorrection
%%%%%%%%%%%%%%%%%%%%

%%%%%%%%%%%%%%%%%%%%
% 如果没有反馈,可以把这部分删掉
\beginfb

% 你可以写
% ~\footnote{优先推荐 \href{problemoverflow.top}{ProblemOverflow}}:
% \begin{itemize}
%   \item 对课程及教师的建议与意见
%   \item 教材中不理解的内容
%   \item 希望深入了解的内容
%   \item $\cdots$
% \end{itemize}
%%%%%%%%%%%%%%%%%%%%
% \bibliography{2-5-solving-recurrence}
% \bibliographystyle{plainnat}
%%%%%%%%%%%%%%%%%%%%
\end{document}